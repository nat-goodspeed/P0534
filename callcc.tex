\newpage
\abschnitt{Call with current continuation}

\cc (abbreviation of 'call with current continuation') is a universal control
operator (well-known from languages like Scheme, Ruby, Lisp ...) that captures
the \currcont (the sequence of instructions after \cc returns) as a
\bfs{first-class object} and passes it to a function that is
executed in a newly-created execution context.\\

\callcc is the C++ equivalent to \cc, preserving the \bfs{call state} and the
\bfs{program state} (variables).\\

When code running in some \emph{original} context calls \resume on some \cont
instance \cpp{target}, the \emph{original} context is saved and
the \cpp{target} continuation is restored in its place, so that program flow
will continue at the point at which the \cpp{target} continuation was
originally captured. The captured \emph{original} continuation then becomes
the \emph{return value} of the \callcc invocation in \cpp{target}.\\

\cont is a \bfs{one-shot continuation}: it can be resumed at most once, is only
move-constructible and move-assignable.
\cppf{loop}

The \cpp{std::callcc(foo)} call at (0) captures the \currcont, entering function
\cpp{foo()} while passing the captured continuation as argument \cpp{caller}.\\

As long as continuation \cpp{caller} is valid, \cpp{"foo"} is passed to standard
output.\\

The expression \cpp{caller.resume()} at (1) resumes the original
continuation represented within \cpp{foo()} by \cpp{caller} and transfers back the
control of execution to \cpp{main()}. On return from \cpp{std::callcc(foo)},
the assignment at (2) sets \cpp{foo\_ct} to the \currcont as of (1).\\

The call to \cpp{foo\_ct.resume()} at (3)
resumes function \cpp{foo}, returning from the \resume call at (1) and executing
the assignment at (4). Here we replace the \cont instance \cpp{caller}
invalidated by the \resume call at (1) with the new instance returned by that
same \resume call.\\

Function \callcc captures the \currcont and enters the given function immediately,
while \resume returns control back to the continuation saved in \cpp{*this}.\\

The presented code prints out \cpp{"foo"} and \cpp{"bar"} in a endless loop.\\

In order to transfer data, \callcc and \resume accept arguments. These are
stored on the stack of the captured \currcont. Function \dataavail tests whether data
have been passed, and with \getdata the data can be retrieved.
\cppf{fibonacci}

The invocation of \callcc at (0) immediately enters the lambda, passing no data
but the \currcont. The lambda calculates the fibonacci number using local
variables \cpp{a}, \cpp{b} and \cpp{next}. The calculated fibonacci number is
transferred via \resume at (1). The execution control returns; \cpp{lambda} now
represents the continuation of the lambda. With \getdata at (2) the fibonacci
number is transferred to the current context while at (3) the lambda is entered
again in order to compute the next fibonacci number -- without passing any
parameter to the lambda.
