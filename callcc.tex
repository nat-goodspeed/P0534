\abschnitt{Call with current continuation}

\cc (abbreviation of 'call with current continuation') is a universal control
operator (well-known from languages like Scheme, Ruby, Lisp ...) that captures
the \currcont (instructions that happen next after \cc returns) as a first-class
object and pass it as an argument to a function that is executed in the newly
created execution context.\\

\call is the C++ equivalent to \cc, preserving the \bfs{call state} and the
\bfs{program state} (variables).\\

When a continuation object is applied to \resume, the existing continuation is
eliminated and the applied continuation is restored in its place, so that the
program flow will continue at the point at which the continuation was captured
and the argument of the continuation then becomes the \emph{return value} of
the \call invocation (see \nameref{mechanism}).\\

\cont is a one-shot continuation - it can be used only once, is only
move-constructible and move-assignable.
\cppf{loop}

The \cpp{std::callcc(foo)} call at (a) captures the \currcont, enters function
\cpp{foo()} while passing the captured continuation as argument \cpp{c1}.\\
As long as continuation \cpp{c1} is valid, \cpp{"foo"} is passed to standard
output.\\
The expression \cpp{std::resume(std::move(c1))} at (b) resumes the original
continuation represented within \cpp{foo()} by \cpp{c1} and transfers back the
control of execution to \cpp{main()}. On return from \cpp{std::callcc(foo)},
the assignment at (c) sets \cpp{c2} to the \currcont by (b).\\
The call to \cpp{resume(std::move(c2))} at (d) (the \emph{updated} \cpp{c2})
resumes function \cpp{foo}, returning from the \cpp{resume()} call at (b) and
executing the assignment at (e). Here, too, we replace the \cont instance
\cpp{c1} invalidated by the \resume call at (b) with the new instance
returned by that same \resume call.\\
Function \call captures \currcont and enters the given function immediately,
while \resume returns the control back to the continuation passed as argument.\\
The presented code prints out \cpp{"foo"} and \cpp{"bar"} in a endless loop.\\

In order to transfer data, \call as well as \resume accept arguments, that are
stored on the stack of the \currcont. Function \davail tests if data have been
passed and with \dtransfer the data can be accessed.
\cppf{fibonacci}

The invocation of \call at (a) immediately enters the lambda, passing no data
but the \currcont. The lambda calculates the fibonacci number on behalf of local
variables \cpp{a}, \cpp{b} and \cpp{next}. The calculated fibonacci number is
transferred via \resume at (b). The execution control returns and \cpp{c2}
represents the continuation of the lambda. With \dtransfer at (c) the fibonacci
number is transferred to the current context while at (d) the lambda is entered
again in order to compute the next fibonacci number (without passing parameter
to the lambda).
