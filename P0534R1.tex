%//////////////////////////////////////////////////////////////////////////////

\documentclass[fontsize=10pt,paper=A4,pagesize,DIV=15]{scrartcl}

\usepackage[T1]{fontenc}
\usepackage[latin1]{inputenc}
\usepackage[british]{babel}

%\usepackage{fixltx2e}
\usepackage{ellipsis}
\usepackage{ragged2e}
\usepackage[final]{microtype}

\usepackage{palatino}

\usepackage{overcite}
\usepackage{booktabs}
\usepackage{fancyhdr}
\usepackage{listings}
\usepackage{perpage}
\usepackage{rotating}
\usepackage{svg}
\usepackage{tikz}
\usetikzlibrary{arrows,automata}
\usepackage{xcolor}
\usepackage{xspace}
\usepackage[colorlinks=true,
            urlcolor=blue,
            pdftex,
            pdfsubject  = {},
            pdfauthor   = {Oliver Kowalke},
            pdfkeywords = {C++,callcc,call/cc,context,continuation,coroutine,execution,fiber,P0099,P0534},
            pdftitle    = {call/cc: A low-level API for stackful context switching}]{hyperref}

%//////////////////////////////////////////////////////////////////////////////

\setlength{\parindent}{0pt} 
\renewcommand\sfdefault{phv}

\makeatletter
    \renewcommand*\l@subsection{\@dottedtocline{2}{0em}{2.3em}}
    \renewcommand*\l@subsection{\@dottedtocline{3}{0em}{3.2em}}
    \renewcommand{\tableofcontents}{\@starttoc{toc}}
\makeatother

\MakePerPage{footnote}
\renewcommand*{\thefootnote}{\fnsymbol{footnote}}

\newcommand{\pdfimg}[1]{\pdfximage{pics/#1}\pdfrefximage\pdflastximage}
\newcommand{\img}[1]{\mbox{\pdfimg{#1}}}
\newcommand{\imgc}[1]{\begin{center}\img{#1}\end{center}}
\newcommand{\graph}[1]{\input{graphs/#1}}
\newcommand{\graphc}[1]{\begin{center}\graph{#1}\end{center}}
\newcommand{\bfs}[1]{{\bfseries #1}}

\newcommand{\cpp}[1]{{\lstinline[
		basicstyle=\ttfamily\small\color{black},
        breakatwhitespace=true,
        breaklines=true,
        captionpos=b,
        commentstyle=\ttfamily\color{red},
        keywordstyle=\ttfamily\color{blue},
        language={C++},
        morekeywords={co\_await,from,noexcept,resumable,co\_yield},
        showspaces=false,
        showstringspaces=false,
        showtabs=false,
        stringstyle=\ttfamily\color{magenta}
] !#1!}\xspace}
\newcommand{\cppf}[1]{\lstinputlisting[
		basicstyle=\ttfamily\small\color{black},
        breakatwhitespace=true,
        breaklines=true,
        captionpos=b,
        commentstyle=\ttfamily\color{red},
        keywordstyle=\ttfamily\color{blue},
        language={C++},
        morekeywords={co\_await,from,noexcept,resumable,co\_yield},
        showspaces=false,
        showstringspaces=false,
        showtabs=false,
        stringstyle=\ttfamily\color{magenta}
] {code/#1.cpp}}

\newcommand{\callcc}{\cpp{std::callcc()}\xspace}
\newcommand{\resume}{\cpp{resume()}\xspace}
\newcommand{\resumewith}{\cpp{resume\_with()}}
\newcommand{\getdata}{\cpp{get\_data()}\xspace}
\newcommand{\dataavail}{\cpp{data\_available()}\xspace}
\newcommand{\cont}{\cpp{std::continuation}\xspace}
\newcommand{\main}{\cpp{main()}\xspace}
\newcommand{\nullptr}{\cpp{nullptr}\xspace}
\newcommand{\opbool}{\cpp{operator bool()}\xspace}
\newcommand{\unwindfn}{\cpp{unwind_context()}\xspace}
\newcommand{\unwindex}{\cpp{std::unwind\_exception}\xspace}

\newcommand{\cc}{\emph{call/cc}\xspace}
\newcommand{\uc}{\emph{ucontext}\xspace}
\newcommand{\lj}{\emph{longjmp}\xspace}
\newcommand{\sj}{\emph{setjmp}\xspace}
\newcommand{\entryfn}{\emph{entry-function}\xspace}

\newcommand{\abschnitt}[1]{\addcontentsline{toc}{subsection}{#1}\subsection*{#1}}
\newcommand{\uabschnitt}[1]{\paragraph*{#1}}


%//////////////////////////////////////////////////////////////////////////////

\begin{document}
\small
\begin{tabbing}
    Document number: \= P0534R1\\
    Date:            \> 2017-06-16\\
    Reply-to:        \> Oliver Kowalke (oliver.kowalke@gmail.com)\\
    Audience:        \> SG1/LEWG\\
\end{tabbing}

\section*{call/cc (call-with-current-continuation): A low-level API for stackful context switching}

%//////////////////////////////////////////////////////////////////////////////

\tableofcontents

%//////////////////////////////////////////////////////////////////////////////

\abschnitt{Abstract}
This document proposes a C++ equivalent to the well-known concept
\bfs{call-with-current-continuation} (abbreviated \bfs{call/cc}). This
facility permits a program written in portable C++ to subdivide processing into
distinct \bfs{contexts:} units smaller than a thread.

Within this proposal, the unadorned term ``thread'' means a \cpp{std::thread}.
When the Standard's more general term ``thread of execution'' is intended, it
is spelled out in full.

With \cc, processing in a given thread may be further subdivided into multiple
contexts. Each such context qualifies as a ``thread of execution'' according
to the definition in the Standard. However, within a given thread, control is
cooperatively passed from one context to another.

This has a couple of important implications:

\begin{itemize}
\item In each thread in a process, exactly one context is running at any given
  time. All others are \bfs{suspended.}
\item The running context on a thread continues running until it explicitly
  \bfs{resumes} some other context. The act of resuming another context
  suspends the previously-running context. This transfer of control, in which
  one context suspends and another resumes, is \bfs{context-switching.}
\end{itemize}

The kind of context-switching presented in this proposal is called
\bfs{stackful} because each context requires some implementation of the C++
stack. C++ code running on a particular context may transparently call
ordinary C++ functions. In contrast to the \bfs{\cpp{co\_await}}
facility (proposed separately\cite{N4649}), this permits encapsulation. A
function that suspends (by resuming some other context) needs no special
signature. Its caller need not be aware that it might suspend. It need not
call that function in any special way.

This supports use cases that cannot be addressed with \cpp{co\_await} alone.

Also in contrast to the \cpp{co\_await} facility, this proposal requires no
changes to the core C++ language. \cc is presented as a library facility,
albeit a library that cannot be implemented in portable C++. This is why it is
desirable to incorporate it into the International Standard.

The proposed \cc facility is intended to be foundational. While of course
application coders are free to use the \cc API, its real promise is in
supporting higher-level abstractions.

This proposal describes the basic \cc facility, presents some illustrative use
cases and explains why the API is set at its present level.

\abschnitt{Revision History}
This document supersedes P0534R0.\cite{P0534R0}\\
\newline
Changes since P0534R0:

\begin{itemize}
    \item API modified
    \begin{itemize}
        \item \cpp{operator()} renamed to \resume
        \item \cpp{operator(invoke\_ontop\_arg\_t)} renamed to \resumewith
        \item \cpp{data\_available()}, \cpp{get\_data()} and \cpp{any\_thread()}
              are now member functions instead of free functions
        \item stack unwinding now explicitly requires specific \cpp{unwind\_exception} 
              exception
    \end{itemize}
    \item \cc compared with \uc and \lj
    \item why \cc is a low-level implementation
    \item details of stack destruction
\end{itemize}

\abschnitt{Continuations}

A continuation is an abstract concept that represents the context state at a
given point during the execution of a program. That implies that a continuation
represents the remaining steps of a computation.\\

As a \bfs{basic, low-level primitive} it can be used to implement control
structures like coroutines,  generators, lightweight threads, cooperative
multitasking (fibers), backtracking, non-deterministic choice. In classic
event-driven programs, organized around a main loop that fetches and dispatches
incoming I/O events, certain asynchronous I/O sequences are logically
sequential, and for those the written and maintained code can look and act
sequential while using continuations.\\

C and C++ already use implicit continuations: if a routine calls a sub-routine,
then a (hidden) continuation (the remaining steps after the sub-routine call) is
created. This continuation is resumed when the sub-routine returns. For
instance the x86 architecture stores the (hidden) continuation as return address
on the stack\footnote{Other (RISC) architectures use a special micro-processor
register for this purpose.}.\\

Continuations exposed as \bfs{first-class continuations} can be passed to and
returned from functions, assigned to variables or stored into containers. With
first-class continuations, a language can explicitly \bfs{control the flow of
execution} via suspending and resuming continuations, enabling control to pass
into a function at exactly the point where it previously suspended.\\
Making the program state visible via first-class continuations is known as
\bfs{reification}.\\

The continuation of the computation step derived from the current point in a
program's execution is called the \bfs{current continuation}. \cc captures the
\bfs{current continuation} and passes it as the argument of the function invoked by
\cc.\\

Continuations that can be called multiple times are named
\bfs{full continuations}.\\
\bfs{One-shot continuations} can only resumed once (a resumed 
\bfs{one-shot continuation} becomes invalidated); control is transferred to
an execution context where the continuation is no longer in scope.\\
Class \cpp{std::execution_context<>}, proposed in \bfs{P0099R1}\cite{P0099R1}
already represents a \bfs{one-shot-continuation}.\\
\bfs{Full continuations} are \bfs{not} considered in this proposal because of
their nature, problematic in C++. Full continuations would require copies of
the stack (including the variables), which would violate C++'s RAII pattern.\\

In contrast to \cc that captures the \bfs{entire remaining} continuation, the
operators \shift and \reset create a so called \bfs{delimited continuation}. A
delimited continuation represents a slice of the program context. Operator
\reset delimits the continuation, i.e. it determines where the continuation
starts and ends, while \shift \bfs{reifies} the continuation.\\
\bfs{Delimited continuations} are \bfs{not part} of this proposal. However,
delimited continuation functionality can be built on \cc, as shown in
\nameref{delimited}.


\abschnitt{Call with current continuation}

\cc (abbreviation of 'call with current continuation') is a universal control
operator (well-known from languages like Scheme, Ruby, Lisp ...) that captures
the \currcont (instructions that happen next after \cc returns) as a first-class
object and pass it as an argument to a function that is executed in the newly
created execution context.\\

\call is the C++ equivalent to \cc, preserving the \bfs{call state} and the
\bfs{program state} (variables).\\

When a continuation object is applied to \resume, the existing continuation is
eliminated and the applied continuation is restored in its place, so that the
program flow will continue at the point at which the continuation was captured
and the argument of the continuation then becomes the \emph{return value} of
the \call invocation (see \nameref{mechanism}).\\

\cont is a one-shot continuation - it can be used only once, is only
move-constructible and move-assignable.
\cppf{loop}

The \cpp{std::callcc(foo)} call at (a) captures the \currcont, enters function
\cpp{foo()} while passing the captured continuation as argument \cpp{c1}.\\
As long as continuation \cpp{c1} is valid, \cpp{"foo"} is passed to standard
output.\\
The expression \cpp{std::resume(std::move(c1))} at (b) resumes the original
continuation represented within \cpp{foo()} by \cpp{c1} and transfers back the
control of execution to \cpp{main()}. On return from \cpp{std::callcc(foo)},
the assignment at (c) sets \cpp{c2} to the \currcont by (b).\\
The call to \cpp{resume(std::move(c2))} at (d) (the \emph{updated} \cpp{c2})
resumes function \cpp{foo}, returning from the \cpp{resume()} call at (b) and
executing the assignment at (e). Here, too, we replace the \cont instance
\cpp{c1} invalidated by the \resume call at (b) with the new instance
returned by that same \resume call.\\
Function \call captures \currcont and enters the given function immediately,
while \resume returns the control back to the continuation passed as argument.\\
The presented code prints out \cpp{"foo"} and \cpp{"bar"} in a endless loop.\\

In order to transfer data, \call as well as \resume accept arguments, that are
stored on the stack of the \currcont. Function \davail tests if data have been
passed and with \dtransfer the data can be accessed.
\cppf{fibonacci}

The invocation of \call at (a) immediately enters the lambda, passing no data
but the \currcont. The lambda calculates the fibonacci number on behalf of local
variables \cpp{a}, \cpp{b} and \cpp{next}. The calculated fibonacci number is
transferred via \resume at (b). The execution control returns and \cpp{c2}
represents the continuation of the lambda. With \dtransfer at (c) the fibonacci
number is transferred to the current context while at (d) the lambda is entered
again in order to compute the next fibonacci number (without passing parameter
to the lambda).

\abschnitt{The switch mechanism}\label{mechanism}

Modern \bfs{micro-processors} are \bfs{registers machines}; the content of
processor registers represent a execution context of the program at a given
point in time.\\
\bfs{Operating systems} simulate parallel execution of programs on a single
processor by switching between programs (\bfs{context switch}) by
\bfs{preserving} and \bfs{restoring} the content of \bfs{all registers}.\\

For \cc no all registers have to be preserved because the calling convention,
part of the ABI, that covers how function's arguments and return values are
passed, determines which subset of CPU registers have to be preserved by the
called subroutine (scratch and callee-saved registers).\\

As a consequence a continuation preserves the execution context, e.g. state of
the register machine (including the stack as well as the instruction pointer).

The \bfs{calling convention}\cite{SYSVABI} of \bfs{SYSV ABI} for \bfs{x86\_64}
architecture determines that general purpose registers R12, R13, R14, R15, RBX
and RBP have to be preserved by the sub-routine - the first arguments are passed
to functions via RDI, RSI, RDX, RCX, R8 and R9 and values returned from
functions in RAX, RDX.\\

\asmfn{jump_x86_64.S}

The code fragment, taken from boost.context\cite{bcontext}, shows how the
context switch might be implemented for \bfs{SYSV ABI}/\bfs{x86\_64}.\\
Line (1) prepares the stack of the current context to hold the content of
registers R12-R15, RBX ad RBP. The address of the stack pointer is preserved in
register RAX at line (10) for further use.\\
The \bfs{return address}, e.g. the address of the instruction that has to be
executed after this function returns, is left on the stack. Other architectures
store the return address in a special register (link register) instead on the
stack. In this case the link register must be preserved too.\\
At line (12) the stack pointer gets assigned to the address of the
continuation that as to be resumed - in fact, the continuation represents a
stack address (the stack pointer was passed in RDI as first argument).\\
The return address is loaded into register R8 at line (14); with the indirect
jump at line (28) the \bfs{continuation} is \bfs{resumed}.\\
As required by the calling convention, registers R12-R15, RBX and RBP are
restored at lines (16) - (21).\\
The stack address, preserved in RAX at line (10), of the
\bfs{suspended continuation} is \bfs{returned} as a \bfs{one-shot continuation}.\\

In fact, \cc is simply the \bfs{exchange} of the \bfs{stack pointer} and certain
\bfs{general purpose registers} - for the surrounding code \cc looks like a
ordinary function.

\abschnitt{Design}

\call expects a \entryfn (callable that is executed in the new continuation)
with signature \cpp{continuation(continuation &&,Arg ...)}.\\

On return the \entryfn has to specify an \cont to which the execution control is
transferred.\\

If an instance with valid state goes out of scope and the \entryfn has not yet
returned, the stack is traversed  and continuation's stack is deallocated.\\


\uabschnitt{Passing data}\label{subsec:data}

\uabschnitt{Toplevel functions: main() and thread functions}\label{subsec:main}

\main as well as the \entryfn of a thread can be represented by an continuation.
That \cont instance is synthesized when the running context suspends, and is
passed into the newly-resumed continuation.
\cppf{simple}
The \cpp{ctx1()} call at (a) enters the lambda in context ctx1.\\
The \cont\ \cpp{ctx2} at (b) represents the execution context of \main.\\
Returning \cpp{ctx2} at (c) resumes the original context (switch back to
\main).

\uabschnitt{\callcc and std::thread}
Any contiunation represented by a valid \cont instance is necessarily suspended.\\
It is valid to resume a \cont instance on any thread -- \emph{except} that you
must not attempt to resume a \cont instance representing \main, or
the \emph{entry-function} of some other \cpp{std::thread}, on any thread other
than its own.
\cont provides a method to test for this.
If \cpp{std::contiunation<>::any\_thread()} returns \cpp{false}, it is
only valid to resume that
\cont instance on the thread on which it was initially launched.


\uabschnitt{Termination}
When that toplevel callable returns a \cont instance, the continuation is
terminated. Control switches to the continuation indicated by the returned \cont
instance.\\
Returning an invalid \cont instance (\opbool returns \cpp{false}) invokes
undefined behavior.\\
If the toplevel callable returns the same \cont instance it was originally
passed (or rather, the most recently updated instance returned from the
previous instance's \call or \resume), control returns to the context that most
recently resumed the running callable. However, the callable may return (switch
to) any reachable valid \cont instance with the correct type signature.


\uabschnitt{Exceptions}\label{subsec:exceptions}
If an uncaught exception escapes from the \entryfn, \cpp{std::terminate} is called.


\uabschnitt{Invoke function on top of a continuation}
Sometimes it is useful to invoke a new function (for instance, to trigger
unwinding the stack) on top of a continuation. For this purpose you may
pass to \cpp{resume(continuation &&,invoke\_ontop\_arg\_t,Fn &&,Args ...)}:

\begin{itemize}
  \item the special argument \cpp{invoke\_ontop\_arg}
  \item the function to execute
  \item any additional arguments.
\end{itemize}

The function passed in this case must accept as parameters the \cont.
It must return the same set of arguments as the \resume specialization.
\footnote{But in the case of passing no arguments, the return type is simply
\cpp{void}.}\\

For purposes of discussion, consider two \cont instances: \cpp{mc} and \cpp{fc}.
Suppose that code running on the program's main context instantiates \cpp{fc}
with function \cpp{f()} and calls \cpp{calcc( fc)}, thereby entering \cpp{f()}.
This is the point at which \cpp{mc} is synthesized and passed into \cpp{f()}.

Suppose further that after doing some work, \cpp{f()} calls \cpp{resume(mc)},
thereby switching context back to the main context. \cpp{f()} remains
suspended in the call to \cpp{resume(mc)}.

At this point the main context calls \cpp{iresume(fc,invoke\_ontop\_arg, g);}
where \cpp{g()} is declared as \cpp{void g(continuation &);} \cpp{g()} is
entered in the context of \cpp{f()}. It is as if \cpp{f()}'s call to
\cpp{resume(mc)} directly called \cpp{g()}.

Function \cpp{g()} has the same range of possibilities as any function called
on \cpp{f()}'s context. Its special invocation only matters when control
leaves it in either of two ways:

\begin{enumerate}
  \item If \cpp{g()} throws an exception, that exception unwinds all previous
        stack entries in that context (such as \cpp{f()}'s) as well, back to a
        matching \cpp{catch} clause.\footnote{As stated in
        \nameref{subsec:exceptions}, if there is no matching \cpp{catch} clause
        in that context, \cpp{std::terminate()} is called.}
  \item If \cpp{g()} returns, its return value becomes the value returned by
        \cpp{f()}'s suspended \cpp{resume(mc)} call. This is why \cpp{g()}'s
        return type must be the same as that of \resume, rather than that of an
        ordinary toplevel context function.
\end{enumerate}

Consider the following example:

\cppf{ontop}

Control passes from (a) to (b) to (c), and so on.

The \cpp{resume(c,invoke\_ontop\_arg, f2, data+1)} call at (l) passes control
to \cpp{f2()} on the context originally created for \cpp{f1()}.

The \cpp{return} statement at (n) causes the \op call at (i) to return,
executing the assignment at (o). The \cpp{int} returned by \cpp{f2()}
is directly returned to that assignment at (o).

So in this example, the call at (l) synthesizes a \cont instance representing
the main context and updates \cpp{mctx} internally. \cpp{f2()} returns its
return value \emph{-1}. Finally, \cpp{f1()} returns its own \cpp{ctx}
variable, switching back to the main context.


\uabschnitt{Stack destruction}\label{subsec:destruction}
On construction of \cpp{execution\_context} a stack is allocated. If the
toplevel context-function returns, the stack will be destroyed. If the
context-function has not yet returned and the \nameref{subpara:destructor} of
a valid \cpp{execution\_context} instance (\opbool
returns \cpp{true}) is called, the stack will be unwound and
destroyed.\footnote{An implementation is free to unwind the stack without
throwing an exception. However, if an exception is thrown, it should be
named \cpp{std::execution\_context\_unwind}. Portable consumer
code \emph{must} permit \cpp{std::execution\_context\_unwind} exceptions to
propagate, even if all other exceptions are caught with \cpp{catch(...)}.}

The stack on which \cpp{main()} is executed, as well as the stack implicitly
created by \cpp{std::thread}'s constructor, is allocated by the operating
system. Such stacks are recognized by \cont, and are not deallocated by its
destructor.

\uabschnitt{Stack allocators}\label{subsec:stackalloc}
are used to create stacks.\footnote{This concept, along with the \cont
constructor accepting \cpp{std::allocator\_arg\_t}, is an optional part of the
proposal. It might be that implementations can reliably infer the optimal
stack representation.} Stack allocators might implement arbitrary stack
strategies. For instance, a stack allocator might append a guard page at the
end of the stack, or cache stacks for reuse, or create stacks that grow on
demand.\\
Because stack allocators are provided by the implementation, and are only used
as parameters of\\
\cont's constructor, the StackAllocator concept is an implementation detail,
used only by the internal mechanisms of the \cont implementation. Different
implementations might use different StackAllocator concepts.\\
However, when an implementation provides a stack allocator matching one of
the descriptions below, it should use the specified name.\\
Possible types of stack allocators:
\begin{itemize}
    \item \cpp{protected\_fixedsize}: The constructor accepts a \cpp{size\_t}
        parameter. This stack allocator constructs a contiguous stack of
        specified size, appending a guard page at the end to protect against
        overflow. If the guard page is accessed (read or write operation), a
        segmentation fault/access violation is generated by the operating
        system.
    \item \cpp{fixedsize}: The constructor accepts a \cpp{size\_t} parameter.
        This stack allocator constructs a contiguous stack of specified size.
        In contrast to \cpp{protected\_fixedsize}, it does not append a guard
        page. The memory is simply managed by \cpp{std::malloc()}
        and \cpp{std::free()}, avoiding kernel involvement.
    \item \cpp{segmented}: The constructor accepts a \cpp{size\_t} parameter.
        This stack allocator creates a segmented stack with the specified
        initial size, which grows on demand.
\end{itemize}


\uabschnitt{std::continuation}
declaration of class \cont
\cppf{cont}
\paragraph*{member functions}
\subparagraph*{(constructor)}
constructs new continuation\\

\begin{tabular}{ l l }
    \midrule

    \cpp{continuation() noexcept} & (1)\\

    \midrule

    \cpp{continuation(continuation&& other)} & (2)\\

    \midrule

    \cpp{continuation(const continuation& other)=delete} & (3)\\

    \midrule
\end{tabular}

\begin{description}
    \item[1)] This constructor instantiates an invalid \cont. Its \opbool
              returns \cpp{false}; its \cpp{operator\!()} returns \cpp{true}.
    \item[2)] moves underlying state to new \cont
    \item[3)] copy constructor deleted
\end{description}

{\bfseries Notes}
\newline
When a \cont is constructed using either of the constructors accepting
\cpp{fn}, control is \emph{not} immediately passed to \cpp{fn}. Resuming
(entering) \cpp{fn} is performed by calling \cpp{operator()()} on the new
\cont instance.\\

\subparagraph*{(destructor)}\label{subpara:destructor}
destroys an continuation\\

\begin{tabular}{ l l }
    \midrule

    \cpp{\~continuation()} & (1)\\

    \midrule
\end{tabular}

\begin{description}
    \item[1)] destroys a \cont instance. If this instance represents a
              context of execution (\opbool returns \cpp{true}),
              then the context of execution is destroyed too. Specifically,
              the stack is unwound. As noted in \nameref{subsec:destruction},
              an implementation is free to unwind the stack either by
              throwing \cpp{std::execution\_context\_unwind} or by intrinsics
              not requiring \cpp{throw}.\\
\end{description}

\subparagraph*{operator=}
moves the continuation object\\

\begin{tabular}{ l l }
    \midrule

    \cpp{continuation& operator=(continuation&& other)} & (1)\\

    \midrule

    \cpp{continuation& operator=(const continuation& other)=delete} & (2)\\

    \midrule
\end{tabular}

\begin{description}
    \item[1)] assigns the state of \cpp{other} to \cpp{*this} using move semantics
    \item[2)] copy assignment operator deleted
\end{description}

{\bfseries Parameters}
\begin{description}
    \item[other]   another execution context to assign to this object\\
\end{description}

{\bfseries Return value}
\begin{description}
    \item[*this]
\end{description}

\subparagraph*{operator bool}
test whether contiunation is valid\\

\begin{tabular}{ l l }
    \midrule

    \cpp{explicit operator bool() const noexcept} & (1)\\

    \midrule
\end{tabular}

\begin{description}
    \item[1)] returns \cpp{true} if \cpp{*this} represents a context of
              execution, \cpp{false} otherwise.
\end{description}

{\bfseries Notes}
\newline
A \cont instance might not represent a context of execution for any of a
number of reasons.
\begin{itemize}
    \item It might have been default-constructed.
    \item It might have been assigned to another instance, or passed into a
          function.\\
          \cont instances are move-only.
    \item It might already have been resumed (\resume called). Calling \resume
          invalidates the instance.
    \item The toplevel context-function might have voluntarily terminated the
          context by returning.
\end{itemize}
The essential points:
\begin{itemize}
    \item Regardless of the number of \cont declarations, exactly one\\
          \cont instance represents each suspended context.
    \item No \cont instance represents the currently-running context.
\end{itemize}

\subparagraph*{operator!}
test whether contiunation is invalid\\

\begin{tabular}{ l l }
    \midrule

    \cpp{bool operator\!() const noexcept} & (1)\\

    \midrule
\end{tabular}

\begin{description}
    \item[1)] returns \cpp{false} if \cpp{*this} represents a context of
              execution, \cpp{true} otherwise.
\end{description}

{\bfseries Notes}
\newline
See {\bfseries Notes} for \opbool.

\subparagraph*{any\_thread}
test whether suspended contiunation may be resumed on a different thread\\

\begin{tabular}{ l l }
    \midrule

    \cpp{bool any\_thread() const noexcept} & (1)\\

    \midrule
\end{tabular}

\begin{description}
    \item[1)] returns \cpp{false} if \cpp{*this} must be resumed on the same
              thread on which it previously ran, \cpp{true} otherwise
\end{description}

{\bfseries Notes}
\newline
As stated in \nameref{subsec:main}, a \cont instance can represent the initial
context on which the operating system runs \main, or the context created by
the operating system for a new \cpp{std::thread}.

It is not permitted to attempt to resume such a \cont instance on any thread
other than its original thread. \cpp{any\_thread()} allows consumer code to
distinguish this case.

\subparagraph*{(comparisons)}
establish an arbitrary total ordering for \cont instances\\

\begin{tabular}{ l l }
    \midrule

    \cpp{bool operator==(const continuation& other) const noexcept} & (1)\\

    \midrule

    \cpp{bool operator\!=(const continuation& other) const noexcept} & (1)\\

    \midrule

    \cpp{bool operator<(const continuation& other) const noexcept} & (2)\\

    \midrule

    \cpp{bool operator>(const continuation& other) const noexcept} & (2)\\

    \midrule

    \cpp{bool operator<=(const continuation& other) const noexcept} & (2)\\

    \midrule

    \cpp{bool operator>=(const continuation& other) const noexcept} & (2)\\

    \midrule
\end{tabular}

\begin{description}
    \item[1)] Every invalid \cont instance compares equal to every other
              invalid instance. But because the running context is never
              represented by a valid \cont instance, and because every
              suspended context is represented by exactly one valid
              instance, \emph{no valid instance can ever compare equal to any
              other valid instance.}
    \item[2)] These comparisons establish an arbitrary total ordering of \cont
              instances, for example to store in ordered containers. (However,
              key lookup is meaningless, since you cannot construct a search
              key that would compare equal to any entry.) There is no
              significance to the relative order of two instances.
\end{description}

\subparagraph*{swap}
swaps two \cont instances\\

\begin{tabular}{ l l }
    \midrule

    \cpp{void swap(continuation& other) noexcept} & (1)\\

    \midrule
\end{tabular}

\begin{description}
    \item[1)] Exchanges the state of \cpp{*this} with \cpp{other}.\\
\end{description}


\uabschnitt{std::callcc()}
declaration of free function \call
\cppf{callcc}

\subparagraph*{std::callcc()}
create and enter new continuation\\

\begin{tabular}{ l l }
    \midrule

    \cpp{template< typename Fn, typename ...Arg >}\\
    \cpp{continuation callcc( Fn &&, Arg ...)} & (1)\\

    \midrule

    \cpp{template< typename StackAlloc, typename Fn, typename ...Arg >}\\
    \cpp{continuation callcc( std::allocator_arg_t, StackAlloc, Fn &&, Arg ...)} & (2)\\

    \midrule
\end{tabular}

\begin{description}
    \item[1)] suspends the active context, resumes the execution context
    \item[2)] specialization of (1) for \cpp{execution\_context<void>}
    \item[3)] suspends the active context, resumes the execution context but
        executes \cpp{fn(args ...)} in the resumed context (on top of the
        last stack frame)
    \item[4)] specialization of (3) for \cpp{execution\_context<void>}
\end{description}

{\bfseries Parameters}
\begin{description}
    \item[...args] If the toplevel context-function represented
                   by \cpp{*this} has not yet been entered, the arguments
                   you pass are passed to the context-function as its
                   parameters, following the \cont first parameter. \\
                   If the context represented by \cpp{*this} suspended by
                   calling \op, the arguments you pass
                   are constructed into a \contargstup and returned by
                   that suspended \op call. \\
                   See section \nameref{subsec:ectxdata}.\\
    \item[fn]      The \cpp{fn} passed to (3) must accept \cpp{Args...}. It
                   must return \cpp{std::tuple<Args...>} or simply \cpp{Arg} in
                   the case of a single argument.\\
                   The \cpp{fn} passed to (4) must accept and return \cpp{void}.\\
\end{description}

{\bfseries Return value}
\begin{description}
    \item[void]     When called on a \contvoid instance, \op returns
                    a different \contvoid instance. This new instance
                    represents the context that suspended in order to resume
                    the current context. That may or may not be the same
                    context that was previously represented by \cpp{*this},
                    depending on other context switches executed in the
                    meantime.
    \item[tuple]    When called on a \contargs instance, \op returns a\\
                    \contargstup containing a different\\
                    \contargs instance. This new instance represents the
                    context that suspended in order to resume the current
                    context, as above.\\
                    If the context represented by the new \contargs instance
                    suspended by calling \op, the arguments passed to \op are
                    used to populate the rest of the members of the
                    returned \cpp{std::tuple}.\\
                    See section \nameref{subsec:ectxdata}.\\
                    If the context represented by the new \contargs instance
                    voluntarily terminated by returning from its toplevel
                    context-function, the rest of the members of the
                    returned \cpp{std::tuple} are indeterminate.\\
\end{description}

{\bfseries Exceptions}
\begin{description}
    \item[1)] calls \cpp{std::terminate} if an exception escapes toplevel \cpp{fn}\\
\end{description}

{\bfseries Preconditions}
\begin{description}
    \item[1)] \cpp{*this} represents a context of execution (\opbool
              returns \cpp{true})
    \item[2)] \cpp{any\_thread()} returns \cpp{true}, or the running thread is
              the same thread on which \cpp{*this} ran previously.
\end{description}

{\bfseries Postcondition}
\begin{description}
    \item[1)] \cpp{*this} is invalidated (\opbool returns \cpp{false})
\end{description}

{\bfseries Notes}
\newline
The \emph{prologue} preserves the execution context of the calling context as
well as stack parts like \emph{parameter list} and \emph{return
address}.\footnote{required only by some x86 ABIs}
Those data are restored by the \emph{epilogue} if the calling context is
resumed.
\newline
A suspended \cpp{execution\_context} can be destroyed. Its resources will be
cleaned up at that time.
\newline
The returned \cpp{execution\_context} indicates whether the suspended context
has terminated (returned from toplevel context-function) via \opbool. If the
returned \cpp{execution\_context} has terminated, no data are transferred in
the returned tuple.
\newline
Because \op invalidates the instance on which it is called, \emph{no
valid \cont instance ever represents the currently-running context.}
\newline
When calling \op, it is conventional to replace the newly-invalidated 
instance -- the instance on which \op was called -- with the new instance
returned by that \op call. This helps to avoid inadvertent calls to \op on the
old, invalidated instance.
\newline
For any \cont specialization other than \contvoid,\\
when \op returns, it is important to test the returned \contargs instance using
\opbool or \cpp{operator\!()} before referencing any of the \cpp{args...} in the
returned \contargstup. If that context voluntarily terminated by returning
from the toplevel context-function, only the \contargs member of
the \cpp{std::tuple} will be populated. The rest of the members will have
indeterminate values.


\subparagraph*{std::resume()}
resume a continuation\\

\begin{tabular}{ l l }
    \midrule

    \cpp{template< typename ...Arg >}\\
    \cpp{continuation resume( continuation &&, Arg ...)} & (1)\\

    \midrule

    \cpp{template< typename Fn, typename ...Arg >}\\
    \cpp{continuation resume( continuation &&, invoke\_ontop\_arg_t, Fn &&, Arg ...)} & (2)\\

    \midrule
\end{tabular}

\begin{description}
    \item[1)] suspends the active context, resumes the execution context
    \item[2)] specialization of (1) for \cpp{execution\_context<void>}
    \item[3)] suspends the active context, resumes the execution context but
        executes \cpp{fn(args ...)} in the resumed context (on top of the
        last stack frame)
    \item[4)] specialization of (3) for \cpp{execution\_context<void>}
\end{description}

{\bfseries Parameters}
\begin{description}
    \item[...args] If the toplevel context-function represented
                   by \cpp{*this} has not yet been entered, the arguments
                   you pass are passed to the context-function as its
                   parameters, following the \cont first parameter. \\
                   If the context represented by \cpp{*this} suspended by
                   calling \op, the arguments you pass
                   are constructed into a \contargstup and returned by
                   that suspended \op call. \\
                   See section \nameref{subsec:ectxdata}.\\
    \item[fn]      The \cpp{fn} passed to (3) must accept \cpp{Args...}. It
                   must return \cpp{std::tuple<Args...>} or simply \cpp{Arg} in
                   the case of a single argument.\\
                   The \cpp{fn} passed to (4) must accept and return \cpp{void}.\\
\end{description}

{\bfseries Return value}
\begin{description}
    \item[void]     When called on a \contvoid instance, \op returns
                    a different \contvoid instance. This new instance
                    represents the context that suspended in order to resume
                    the current context. That may or may not be the same
                    context that was previously represented by \cpp{*this},
                    depending on other context switches executed in the
                    meantime.
    \item[tuple]    When called on a \contargs instance, \op returns a\\
                    \contargstup containing a different\\
                    \contargs instance. This new instance represents the
                    context that suspended in order to resume the current
                    context, as above.\\
                    If the context represented by the new \contargs instance
                    suspended by calling \op, the arguments passed to \op are
                    used to populate the rest of the members of the
                    returned \cpp{std::tuple}.\\
                    See section \nameref{subsec:ectxdata}.\\
                    If the context represented by the new \contargs instance
                    voluntarily terminated by returning from its toplevel
                    context-function, the rest of the members of the
                    returned \cpp{std::tuple} are indeterminate.\\
\end{description}

{\bfseries Exceptions}
\begin{description}
    \item[1)] calls \cpp{std::terminate} if an exception escapes toplevel \cpp{fn}\\
\end{description}

{\bfseries Preconditions}
\begin{description}
    \item[1)] \cpp{*this} represents a context of execution (\opbool
              returns \cpp{true})
    \item[2)] \cpp{any\_thread()} returns \cpp{true}, or the running thread is
              the same thread on which \cpp{*this} ran previously.
\end{description}

{\bfseries Postcondition}
\begin{description}
    \item[1)] \cpp{*this} is invalidated (\opbool returns \cpp{false})
\end{description}

{\bfseries Notes}
\newline
The \emph{prologue} preserves the execution context of the calling context as
well as stack parts like \emph{parameter list} and \emph{return
address}.\footnote{required only by some x86 ABIs}
Those data are restored by the \emph{epilogue} if the calling context is
resumed.
\newline
A suspended \cpp{execution\_context} can be destroyed. Its resources will be
cleaned up at that time.
\newline
The returned \cpp{execution\_context} indicates whether the suspended context
has terminated (returned from toplevel context-function) via \opbool. If the
returned \cpp{execution\_context} has terminated, no data are transferred in
the returned tuple.
\newline
Because \op invalidates the instance on which it is called, \emph{no
valid \cont instance ever represents the currently-running context.}
\newline
When calling \op, it is conventional to replace the newly-invalidated 
instance -- the instance on which \op was called -- with the new instance
returned by that \op call. This helps to avoid inadvertent calls to \op on the
old, invalidated instance.
\newline
For any \cont specialization other than \contvoid,\\
when \op returns, it is important to test the returned \contargs instance using
\opbool or \cpp{operator\!()} before referencing any of the \cpp{args...} in the
returned \contargstup. If that context voluntarily terminated by returning
from the toplevel context-function, only the \contargs member of
the \cpp{std::tuple} will be populated. The rest of the members will have
indeterminate values.


\subparagraph*{std::transfer\_data<>()}
transfer of data\\

\abschnitt{Why \cc is preferred over \uc}

\uabschnitt{stack represents the continuation}

In contrast to \uc, \cc uses the stack as storage for the suspended
execution context (the content of the registers, see P0534R0\cite{P0534R0}).

\begin{itemize}
    \item only the target has to be provided at resumption
        (\cpp{swap\_ucontext()} required source and target)
    \item current execution context is already represented by the
        stack to which the stack-pointer points to
    \item suspended execution context is passed as continuation (parameter) 
        to the resumed execution context
    \item no need for a global pointer that points to the current execution context
    \item \main and thread's \entryfn do fit seamlessly into the concept of \cc
        because the stack of \main or thread already represents the continuation of \main
        and thread
\end{itemize}

\uabschnitt{aggregation of stack address}

A instance of \cont aggregates the stack address of a suspended execution
- \cont:

\begin{itemize}
    \item represents the continuation of an suspended context
    \item prevents accidentaly copying the stack
    \item prevents accidentaly resuming an already resumed/running execution
        context
    \item prevents accidentaly resuming an execution context that has already
        terminated (compuation has already been finished)
    \item manages lifespan of an explictly-allocated stack, the stack gets
        deallocated if \cont goes out of scope
    \item context switch and data transfer via one function call
\end{itemize}

Ofcourse a \uc-like API would be possible but it would be unsafe and error
prone.

\abschnitt{Why \cc is low-level}

\cc is a low-level implementation. \cont has the memory footprint of a pointer
because it contains only a pointer to the stack of the managed continuation.\\
\newline
The assembly code generated for a function consists of a \emph{function prologue} at the
beginning and \emph{function epilogue} at the end.
\cppf{function}

The \emph{prologue} (a few lines of assembler) prepares the stack and
registers for use inside the function. The \emph{epilogue} restores the stack
and registers\footnote{callee-saved registers as defined by the calling
convention} to the state they were before the function was called.\\
Between prologue and epilogue the computation and calls to sub-routines are
done.\\
\newline
Like ordinary functions, \resume and \resumewith contain prologue and
epilogue. The only difference from ordinary functions is that \resume and
\resumewith additionally exchange the stack- and instruction pointer
\footnote{In fact on x86-architecture the instruction-pointer return-address
is already stored on the stack. On some RISC-architectures the link-register
must be preserved on the stack while the context is suspended, and is loaded
into the instruction-pointer on resumption.} between prologue and epilogue.\\
The prologue and epilogue of \resume and \resumewith neither consume stack space
nor do they call sub-routines; only the stack-pointer is exchanged.
\cppf{implementation}

\abschnitt{Stack destruction}

When an instance of \cont goes out of scope and the contained stack-pointer is
not \nullptr,\footnote{That means that the computation was not finished, hence
the execution context is still suspended.} the stack gets
unwound and deallocated.\\
For this purpose member-function \resumewith is called with a special function
(call it \unwindfn) as argument. The execution context will be
resumed and \unwindfn is invoked. Function \unwindfn throws exception
\unwindex.\footnote{\unwindex binds an instance of \cont that represents the
continuation that called the destructor.} The exception
is caught by the first frame on the stack: the one created by
\callcc. Control is switched back to the context that called
\cpp{\~continuation()} and the stack gets deallocated.


\abschnitt{Performance of \cc}

On modern architectures suspending/resuming continuations takes very few CPU cycles.
\footnote{\cpp{callcc()} from boost.context takes 18 CPU cycles on Intel E5 2620 v4,
SYS V.}

\abschnitt{Changed API}

As requested at the C++ meeting in Kona; free-functions \getdata and \dataavail
became member-functions of \cont. Additionally \cpp{operator(arg...)} and
\cpp{operator(invoke_ontop_arg,arg...)} have been renamed to \resume and
\resumewith.\\
\newline
Keeping \callcc as free-function is preferred because \cont represents a
continuation \footnote{A continuation is an abstract concept that represents the
context state at a given point during the execution of a program (for more
details see P0534R0, section \emph{Continuation}\cite{P0534R0})}.
In this context \callcc acts as a factory-method, it creates and resumes a new
execution context (stack etc.) and if this execution context is suspended
\callcc returns a continuation that represents the rest of the computation.\\
The rest of the API remains a proposed in P0534R0\cite{P0534R0}.
\cppf{continuation}

\abschnitt{Additional notes}
\uabschnitt{Interaction with other ToE}
\uabschnitt{Thread-local-storage}
\uabschnitt{Debugging}
\uabschnitt{Migration between threads}


%//////////////////////////////////////////////////////////////////////////////

\addcontentsline{toc}{subsection}{References}
\begin{thebibliography}{99}

    \bibitem{P0099R1}
        \href{http://www.open-std.org/jtc1/sc22/wg21/docs/papers/2016/p0099r1.pdf}
        {P0099R1: A low-level API for stackful context switching}

    \bibitem{schemecallcc}
        \href{http://community.schemewiki.org/?call-with-current-continuation}
        {call/cc in Scheme}

    \bibitem{rubycallcc}
        \href{http://gnuu.org/2009/03/21/demystifying-continuations-in-ruby}
        {call/cc in Ruby}

    \bibitem{SYSVABI}
        {System V Application Binary Interface, AMD64 Architecture Processor Supplement,
        Draft Version 0.96}

    \bibitem{N3708}
        \href{http://www.open-std.org/jtc1/sc22/wg21/docs/papers/2013/n3708.pdf}
        {N3708: A proposal to add coroutines to the C++ standard library}

    \bibitem{gccsplit}
        \href{http://gcc.gnu.org/wiki/SplitStacks}
        {Split Stacks / GCC}

    \bibitem{bcontext}
        \href{http://www.boost.org/doc/libs/release/libs/context/doc/html/index.html}
        {Library \emph{Boost.Context}} (\cc available int boost-1.64)

    \bibitem{bcoroutine2}
        \href{http://www.boost.org/doc/libs/release/libs/coroutine2/doc/html/index.html}
        {Library \emph{Boost.Coroutine2}}

    \bibitem{bfiber}
        \href{http://www.boost.org/doc/libs/release/libs/fiber/doc/html/index.html}
        {Library \emph{Boost.Fiber}}

    \bibitem{Ferguson}
        {Darrell Ferguson and Dwight Deugo, Call with Current Continuation Patterns.
        in 8th Conference on Pattern Languages of Programs (PLoP 2001) , 2001}

\end{thebibliography}


%//////////////////////////////////////////////////////////////////////////////

\end{document}
