%//////////////////////////////////////////////////////////////////////////////

\documentclass[fontsize=10pt,paper=A4,pagesize,DIV=15]{scrartcl}

\usepackage[T1]{fontenc}
\usepackage[latin1]{inputenc}
\usepackage[british]{babel}

\usepackage{amsmath}
\usepackage{ellipsis}
\usepackage{ragged2e}
\usepackage[final]{microtype}

\usepackage{palatino}

\usepackage{overcite}
\usepackage{booktabs}
\usepackage{fancyhdr}
\usepackage{listings}
\usepackage{perpage}
\usepackage{rotating}
\usepackage{svg}
\usepackage{tikz}
\usetikzlibrary{arrows,automata}
\usepackage{xcolor}
\usepackage{xspace}
\usepackage[colorlinks=true,
            urlcolor=blue,
            pdftex,
            pdfsubject  = {},
            pdfauthor   = {Oliver Kowalke},
            pdfkeywords = {C++,callcc,call/cc,context,continuation,coroutine,execution,fiber,P0099,P0534},
            pdftitle    = {call/cc: A low-level API for stackful context switching}]{hyperref}

%//////////////////////////////////////////////////////////////////////////////

\setlength{\parindent}{0pt} 
\renewcommand\sfdefault{phv}

\makeatletter
    \renewcommand*\l@subsection{\@dottedtocline{2}{0em}{2.3em}}
    \renewcommand*\l@subsection{\@dottedtocline{3}{0em}{3.2em}}
    \renewcommand{\tableofcontents}{\@starttoc{toc}}
\makeatother

\MakePerPage{footnote}
\renewcommand*{\thefootnote}{\fnsymbol{footnote}}

\newcommand{\pdfimg}[1]{\pdfximage{pics/#1}\pdfrefximage\pdflastximage}
\newcommand{\img}[1]{\mbox{\pdfimg{#1}}}
\newcommand{\imgc}[1]{\begin{center}\img{#1}\end{center}}
\newcommand{\graph}[1]{\input{graphs/#1}}
\newcommand{\graphc}[1]{\begin{center}\graph{#1}\end{center}}

\lstdefinelanguage
   [arm]{Assembler}                     % add a "arm" dialect of Assembler
   {morekeywords={mov,pop,push,subi,stw,mfcr,mflr,mr,lwz,mtcr,mtlr,mtctr,addi,bctr,str}}    % with these extra keywords:
\lstset{
        language=[arm]Assembler,
        numbers=none,
        numberstyle=\tiny,
        numberblanklines=false,
        stepnumber=1,
        numbersep=10pt
}

\newcommand{\cpp}[1]{{\lstinline[
		basicstyle=\ttfamily\small\color{black},
        breakatwhitespace=true,
        breaklines=true,
        captionpos=b,
        commentstyle=\ttfamily\color{gray},
        keywordstyle=\ttfamily\color{blue},
        language={C++},
        morekeywords={co\_await,from,noexcept,resumable,co\_yield},
        showspaces=false,
        showstringspaces=false,
        showtabs=false,
        stringstyle=\ttfamily\color{red}
] !#1!}\xspace}
\newcommand{\cppf}[1]{\lstinputlisting[
		basicstyle=\ttfamily\small\color{black},
        breakatwhitespace=true,
        breaklines=true,
        captionpos=b,
        commentstyle=\ttfamily\color{gray},
        keywordstyle=\ttfamily\color{blue},
        language={C++},
        morekeywords={co\_await,from,noexcept,resumable,co\_yield},
        showspaces=false,
        showstringspaces=false,
        showtabs=false,
        stringstyle=\ttfamily\color{red}
] {code/#1.cpp}}


\newcommand{\asm}[1]{
    \lstinline[
        basicstyle=\ttfamily\color{black},
        keywordstyle=\color{blue},
        commentstyle=\color{red},
        stringstyle=\color{green}
    ] {#1}
}
\newcommand{\asmf}[1]{
    \lstinputlisting[
%        numbers=left,
        basicstyle=\ttfamily\color{black},
        keywordstyle=\color{blue},
        commentstyle=\color{red},
        stringstyle=\color{green}
    ] {code/#1}
}

\newcommand{\call}{\cpp{std::callcc()}}
\newcommand{\resume}{\cpp{std::resume()}}
\newcommand{\cont}{\cpp{std::contiunation}}
\newcommand{\ectx}{\cpp{std::execution\_context<>}}
\newcommand{\main}{\cpp{main()}}

\newcommand{\callcc}{\emph{call-with-current-continuation}}
\newcommand{\cc}{\emph{call/cc}}
\newcommand{\contfn}{\emph{continuation-function}}

\newcommand{\abschnitt}[1]{\addcontentsline{toc}{subsection}{#1}\subsection*{#1}}
\newcommand{\uabschnitt}[1]{\addcontentsline{toc}{subsubsection}{#1}\paragraph*{#1}}


%//////////////////////////////////////////////////////////////////////////////

\begin{document}
\small
\begin{tabbing}
    Document number: \= P0534R1\\
    Date:            \> 2017-06-18\\
    Reply-to:        \> Oliver Kowalke (oliver.kowalke@gmail.com)\\
    Authors:         \> Oliver Kowalke (oliver.kowalke@gmail.com), Nat Goodspeed (nat@lindenlab.com)\\
    Audience:        \> SG1/LEWG\\
\end{tabbing}

\section*{call/cc (call-with-current-continuation): A low-level API for stackful context switching}

%//////////////////////////////////////////////////////////////////////////////

\tableofcontents

%//////////////////////////////////////////////////////////////////////////////

\abschnitt{Abstract}
This document proposes a C++ equivalent to the well-known concept
\bfs{call-with-current-continuation} (abbreviated \bfs{call/cc}). This
facility permits a program written in portable C++ to subdivide processing into
distinct \bfs{contexts:} units smaller than a thread.\\

Within this proposal, the unadorned term ``thread'' means a \cpp{std::thread}
(or \bfs{kernel thread}).
When the Standard's more general term ``thread of execution'' is intended, it
is spelled out in full.\\

With \cc, processing in a given thread may be further subdivided into multiple
contexts. Each such context qualifies as a ``thread of execution'' according
to the definition in the Standard. However, within a given thread, control is
cooperatively passed from one context to another.\\

This has a couple of important implications:

\begin{itemize}
\item In each thread in a process, exactly one context is running at any given
  time. All others are \bfs{suspended.}
\item The running context on a thread continues running until it explicitly
  \bfs{resumes} some other context. The act of resuming another context
  suspends the previously-running context. This transfer of control, in which
  one context suspends and another resumes, is \bfs{context-switching.}
\item There are no data races between contexts running on the same thread.
\end{itemize}

The kind of context-switching presented in this proposal is called
\bfs{stackful} because each context requires some implementation of the C++
stack. C++ code running on a particular context may transparently call
ordinary C++ functions. In contrast to the \bfs{\coawait}
facility (proposed separately\cite{N4649}), this permits encapsulation. A
function that suspends (by resuming some other context) needs no special
signature. Its caller need not be aware that it might suspend. It need not
call that function in any special way.\\

This supports use cases that cannot be addressed with \coawait alone.\\

Also in contrast to the \coawait facility, this proposal requires no
changes to the core C++ language. \cc is presented as a library facility,
albeit a library that cannot be implemented in portable C++. This is why it is
desirable to incorporate it into the International Standard.\\

Consider the following bullets from P0559R0:\cite{P0559R0}

\begin{itemize}
\item Avoid 'compiler magic' when possible
\item Prefer library solutions over language changes if feasible
\end{itemize}

The proposed \cc facility is intended to be foundational. While of course
application coders are free to use the \cc API, its real promise is in
supporting higher-level abstractions.\\

This proposal describes the basic \cc facility, presents some illustrative use
cases and explains why the API is set at its present level.

\abschnitt{Why should \cc be standardized?}

The \cc facility cannot itself be implemented in portable C++. The present
implementation,\cite{bcontext} maintained by a single author, supports a
small set of current platforms available to that author. Should \cc be
integrated into the Standard, it will become universally available.\\

Moreover, correct support for certain platforms might involve undocumented
complexity. The runtime vendor is best positioned to implement the specified
functionality.\\

Compiler awareness of this facility could enable certain optimizations as
well:

\begin{itemize}
  \item The compiler might be able to analyze the code to be launched on a new
    \cc context and determine an optimal stack size for that context.
  \item The compiler might be able to determine that not all registers need be
    preserved across a context switch.
  \item For certain use cases, the compiler might be able to optimize away
    context-switching altogether. Promising work has been done in this area
    for the \coawait facility.\cite{N4649}
\end{itemize}

\abschnitt{Revision History}
This document supersedes P0534R0.\cite{P0534R0}\\
\newline
Changes since P0534R0:

\begin{itemize}
    \item API modified
    \begin{itemize}
        \item \cpp{operator()} renamed to \resume
        \item \cpp{operator(invoke\_ontop\_arg\_t)} renamed to \resumewith
        \item \dataavail, \getdata and \anythread are now member functions
          instead of free functions
        \item stack unwinding now explicitly requires specific \cpp{unwind\_exception} 
              exception
    \end{itemize}
    \item added example use cases
    \item \cc compared with \uc and \lj
    \item why \cc is a low-level implementation
    \item details of stack destruction
\end{itemize}

\newpage
\abschnitt{Continuations}

A continuation is an abstract concept that represents the context state at a
given point during the execution of a program. That implies that a continuation
represents the remaining steps of a computation.\\

As a \bfs{basic, low-level primitive} it can be used to implement control
structures like coroutines,  generators, lightweight threads, cooperative
multitasking (fibers), backtracking, non-deterministic choice. In classic
event-driven programs, organized around a main loop that fetches and dispatches
incoming I/O events, certain asynchronous I/O sequences are logically
sequential. Use of continuations permits writing and maintaining code that
looks and acts sequential, even though from time to time it may suspend while
asynchronous I/O is pending.\\

C and C++ already use implicit continuations: when running code calls a
function, then a (hidden) continuation (the remaining steps after the function
call) is created. This continuation is resumed when the function returns. For
instance the x86 architecture stores the (hidden) continuation as a return
address on the stack.\footnote{Other (RISC) architectures use a special
micro-processor register for this purpose.}\\

Continuations exposed as \bfs{first-class continuations} can be passed to and
returned from functions, assigned to variables or stored into containers. With
first-class continuations, a language can explicitly \bfs{control the flow of
execution} by suspending and resuming continuations, enabling control to pass
into a function at exactly the point where it previously suspended.\\
Making the program state visible via first-class continuations is known as
\bfs{reification}.\\

The remainder of the computation derived from the current point in a
program's execution is called the \bfs{current continuation}. \cc captures the
\bfs{current continuation} and passes it to the function invoked by
\cc.\\

Continuations that can be called multiple times are named
\bfs{full continuations}.\\
\bfs{One-shot continuations} can only resumed once: once resumed, a
\bfs{one-shot continuation} is invalidated.\\
\bfs{Full continuations} are \bfs{not} considered in this proposal because of
their nature, which is problematic in C++. Full continuations would require copies of
the stack (including all stack variables), which would violate C++'s RAII pattern.\\

In contrast to \cc that captures the \bfs{entire remaining} continuation, the
operators \shift and \reset create a so-called \bfs{delimited continuation}. A
delimited continuation represents a slice of the program context. Operator
\reset delimits the continuation, i.e. it determines where the continuation
starts and ends, while \shift\ \bfs{reifies} the continuation.\\
\bfs{Delimited continuations} are \bfs{not} part of this proposal. However,
delimited continuation functionality can be built on \cc.


\abschnitt{Call with current continuation}

\cc (abbreviation of 'call with current continuation') is a universal control
operator (well-known from languages like Scheme, Ruby, Lisp ...) that captures
the \currcont (instructions that happen next after \cc returns) as a
\bfs{first-class object} and pass it as an argument to a function that is
executed in the newly created execution context.\\

\call is the C++ equivalent to \cc, preserving the \bfs{call state} and the
\bfs{program state} (variables).\\

When a continuation object is applied to \resume, the existing continuation is
eliminated and the applied continuation is restored in its place, so that the
program flow will continue at the point at which the continuation was captured
and the currnt continuation then becomes the \emph{return value} of the \call
invocation (see \nameref{mechanism}).\\

\cont is a \bfs{one-shot continuation} - it can be used only once, is only
move-constructible and move-assignable.
\cppf{loop}

The \cpp{std::callcc(foo)} call at (a) captures the \currcont, enters function
\cpp{foo()} while passing the captured continuation as argument \cpp{c1}.\\
As long as continuation \cpp{c1} is valid, \cpp{"foo"} is passed to standard
output.\\
The expression \cpp{std::resume(std::move(c1))} at (b) resumes the original
continuation represented within \cpp{foo()} by \cpp{c1} and transfers back the
control of execution to \cpp{main()}. On return from \cpp{std::callcc(foo)},
the assignment at (c) sets \cpp{c2} to the \currcont by (b).\\
The call to \cpp{resume(std::move(c2))} at (d) (the \emph{updated} \cpp{c2})
resumes function \cpp{foo}, returning from the \cpp{resume()} call at (b) and
executing the assignment at (e). Here, too, we replace the \cont instance
\cpp{c1} invalidated by the \resume call at (b) with the new instance
returned by that same \resume call.\\
Function \call captures \currcont and enters the given function immediately,
while \resume returns the control back to the continuation passed as argument.\\
The presented code prints out \cpp{"foo"} and \cpp{"bar"} in a endless loop.\\

In order to transfer data, \call as well as \resume accept arguments, that are
stored on the stack of the \currcont. Function \davail tests if data have been
passed and with \dget the data can be accessed.
\cppf{fibonacci}

The invocation of \call at (a) immediately enters the lambda, passing no data
but the \currcont. The lambda calculates the fibonacci number using local
variables \cpp{a}, \cpp{b} and \cpp{next}. The calculated fibonacci number is
transferred via \resume at (b). The execution control returns and \cpp{c2}
represents the continuation of the lambda. With \dget at (c) the fibonacci
number is transferred to the current context while at (d) the lambda is entered
again in order to compute the next fibonacci number (without passing parameter
to the lambda).

\abschnitt{Design}\label{design}

Because \cont represents a continuation (it \bfs{contains} only its
\bfs{stack pointer} as member variable) it is proposed as a
\bfs{pure data structure} (no behaviour exposed via member functions,
only status queries).


\uabschnitt{Passing data}\label{subsec:data}

Data are passed to another context as additional arguments of \call and
\resume.\\
With functions \davail and \dget the code can test for data and if desired
transfer the data.
\cppf{passing_single}
The \cpp{callcc()} call at (a) enters the lambda and passes \cpp{j=1} into the
new context. The value is transferred as shown by (b). The expression
\cpp{callcc(c2,j+1)} at (c) resumes the original context (represented
within the lambda by \cpp{c2}) and transfers back an integer of \cpp{j+1}.
The assignment at (d) sets \cpp{i} to \cpp{j+1}.\\
The call to \cpp{callcc(c1)} (note that no data is passsed) at (e) resumes the
\cpp{c1} lambda, returning from the \cpp{callcc(c2,j+1)} call at (c). Here, too,
we replace the \cont instance \cpp{c2} invalidated by the \resume call at (c)
with the new instance returned by that same \resume call.\\
Finally the lambda returns (the updated) \cpp{c2} at (f), terminating its
context.\\
Since the updated \cpp{c2} represents the continuation suspended by the call at
(e), control returns to \main.\\
However, since context \cpp{c1} has now terminated, the updated \cpp{c1} is
invalid. Its \opbool returns \cpp{false}; its \cpp{operator\!()} returns
\cpp{true}.\\
It may seem tricky to keep track of which \cont instance is currently valid,
representing the state of the suspended context. Please bear in mind that this
facility is intended as a high-performance foundation for higher-level
libraries. It is not intended to be directly consumed by applications.\\
Multiple arguments can be transferred into another continuation too.
\cppf{passing_multiple}
\call takes the parameters \cpp{i} and \cpp{j} that are returned from \dget
as \cpp{std::tuple} of \cpp{int}.


\uabschnitt{main() and thread functions}\label{subsec:main}

\main as well as the \entryfn of a thread can be represented by a continuation.
That \cont instance is synthesized when the running context suspends, and is
passed into the newly-resumed continuation.
\cppf{simple}
The \cpp{callcc()} call at (a) enters the lambda. The \cont\ \cpp{c2} at (b)
represents the execution context of \main. Returning \cpp{c2} at (c) resumes the
original context (switch back to \main).


\uabschnitt{\cc and std::thread}
Any continuation represented by a valid \cont instance is necessarily suspended.\\
It is valid to resume a \cont instance on any thread -- \emph{except} that
since the operating system is responsible for the stack allocated for \main,
as well as each \cpp{std::thread}, you must not attempt to resume a \cont
instance representing any such context on any thread other
than its own. \cpp{any\_thread()} tests for this.\\
If \cpp{std::any\_thread()} returns \cpp{false}, it is
only valid to resume that \cont instance on the thread on which it was initially
launched.


\uabschnitt{Termination}
When the \entryfn invoked by \cc returns a valid \cont instance,
the running context is terminated. Control switches to the continuation
indicated by the returned \cont instance.\\
Returning an invalid \cont instance (\opbool returns \cpp{false}) invokes
undefined behavior.\\
If the \entryfn returns the same \cont instance it was originally
passed (or rather, the most recently updated instance returned from the
previous instance's \call or \resume), control returns to the context that most
recently resumed the running callable. However, the callable may return (switch
to) any reachable valid \cont instance.


\uabschnitt{Exceptions}\label{subsec:exceptions}
If an uncaught exception escapes from the \entryfn, \cpp{std::terminate} is
called.


\uabschnitt{Invoke function on top of a continuation}
Sometimes it is useful to invoke a new function (for instance, to throw an
exception) on top of a continuation. For this purpose you may pass to\\
\cpp{resume(continuation &&,invoke\_ontop\_arg\_t,Fn &&,Args ...)}:

\begin{itemize}
  \item the special argument \cpp{invoke\_ontop\_arg}
  \item the function to execute
  \item any additional arguments.
\end{itemize}

The function passed in this case must accept a reference of \cont (in order to
pass the continuation in a thrown exception - otherwise the continuation gets
deallocated).\\

Suppose that code running on the program's main context calls \cpp{callcc(f)},
thereby entering \cpp{f()}. This is the point at which \cpp{mc} is synthesized
and passed into \cpp{f()}.\\
Suppose further that after doing some work, \cpp{f()} calls \cpp{resume(mc)},
thereby switching context back to the main context. \cpp{f()} remains suspended
in the call to \cpp{resume(mc)}.\\
At this point the main context calls \cpp{resume(fc,invoke\_ontop\_arg, g);}
where \cpp{g()} is declared as\\
\cpp{void g(continuation &);} \cpp{g()} is entered in the context of \cpp{f()}.
It is as if \cpp{f()}'s call to \cpp{resume(mc)} directly called \cpp{g()}.\\
Function \cpp{g()} has the same range of possibilities as any function called on
\cpp{f()}'s context. Its special invocation only matters when control leaves it
in either of two ways:

\begin{enumerate}
  \item If \cpp{g()} throws an exception, that exception unwinds all previous
        stack entries in that context (such as \cpp{f()}'s) as well, back to a
        matching \cpp{catch} clause.\footnote{As stated in \nameref{subsec:exceptions},
        if there is no matching \cpp{catch} clause in that context,
        \cpp{std::terminate()} is called.}
  \item If \cpp{g()} returns, its return value becomes the value returned by
        \cpp{f()}'s suspended \cpp{resume(mc)} call.
\end{enumerate}

\cppf{ontop}

Control passes from (a) to (b) to (c), and so on.\\
The \cpp{resume(c,invoke\_ontop\_arg, f2, data+1)} call at (i) passes control
to \cpp{f2()} on the continuation of \cpp{f1()}.\\
The \cpp{return} statement at (j) causes the \resume call at (g) to return,
executing the assignment at (k). The \cpp{int} returned by \cpp{f2()} is
accessed at (l).\\
Finally, \cpp{f1()} returns its own \cpp{c} variable, switching back to the main
context.


\uabschnitt{Stack destruction}\label{subsec:destruction}
On construction of a continuation with \call a stack is allocated. If the
\entryfn returns, the stack will be destroyed. If the function has not
yet returned and the \nameref{subpara:destructor} of a valid \cont instance (\opbool
returns \cpp{true}) is called, the stack will be unwound and destroyed.
\footnote{An implementation is free to unwind the stack without throwing an
exception.}\\
The stack on which \cpp{main()} is executed, as well as the stack implicitly
created by \cpp{std::thread}'s constructor, is allocated by the operating
system. Such stacks are recognized by \cont, and are not deallocated by its
destructor.


\uabschnitt{Stack allocators}\label{subsec:stackalloc}
are used to create stacks.\footnote{This concept, along with \call accepting
\cpp{std::allocator\_arg\_t}, is an optional part of the proposal. It might be
that implementations can reliably infer the optimal stack representation.}
Stack allocators might implement arbitrary stack strategies. For instance, a
stack allocator might append a guard page at the end of the stack, or cache
stacks for reuse, or create stacks that grow on demand.\\
Because stack allocators are provided by the implementation, and are only used
as parameters of \call, the StackAllocator concept is an implementation detail,
used only by the internal mechanisms of the \cc implementation. Different
implementations might use different StackAllocator concepts.\\
However, when an implementation provides a stack allocator matching one of
the descriptions below, it should use the specified name.\\
Possible types of stack allocators:
\begin{itemize}
    \item \cpp{protected\_fixedsize}: The constructor accepts a \cpp{size\_t}
        parameter. This stack allocator constructs a contiguous stack of
        specified size, appending a guard page at the end to protect against
        overflow. If the guard page is accessed (read or write operation), a
        segmentation fault/access violation is generated by the operating
        system.
    \item \cpp{fixedsize}: The constructor accepts a \cpp{size\_t} parameter.
        This stack allocator constructs a contiguous stack of specified size.
        In contrast to \cpp{protected\_fixedsize}, it does not append a guard
        page. The memory is simply managed by \cpp{std::malloc()}
        and \cpp{std::free()}, avoiding kernel involvement.
    \item \cpp{segmented}: The constructor accepts a \cpp{size\_t} parameter.
        This stack allocator creates a segmented stack with the specified
        initial size, which grows on demand.
\end{itemize}


\uabschnitt{std::continuation}
declaration of class \cont
\cppf{cont}
\paragraph*{member functions}
\subparagraph*{(constructor)}
constructs new continuation\\

\begin{tabular}{ l l }
    \midrule

    \cpp{continuation() noexcept} & (1)\\

    \midrule

    \cpp{continuation(continuation&& other)} & (2)\\

    \midrule

    \cpp{continuation(const continuation& other)=delete} & (3)\\

    \midrule
\end{tabular}

\begin{description}
    \item[1)] This constructor instantiates an invalid \cont. Its \opbool
              returns \cpp{false}; its \cpp{operator\!()} returns \cpp{true}.
    \item[2)] moves underlying state to new \cont
    \item[3)] copy constructor deleted
\end{description}

\subparagraph*{(destructor)}\label{subpara:destructor}
destroys a continuation\\

\begin{tabular}{ l l }
    \midrule

    \cpp{\~continuation()} & (1)\\

    \midrule
\end{tabular}

\begin{description}
    \item[1)] destroys a \cont instance. If this instance represents a context
              of execution (\opbool returns \cpp{true}), then the context of
              execution is destroyed too. Specifically, the stack is unwound. As
              noted in \nameref{subsec:destruction}, an implementation is free to
              unwind the stack either by throwing an exception or by
              intrinsics not requiring \cpp{throw}.
\end{description}

\subparagraph*{operator=}
moves the continuation object\\

\begin{tabular}{ l l }
    \midrule

    \cpp{continuation& operator=(continuation&& other)} & (1)\\

    \midrule

    \cpp{continuation& operator=(const continuation& other)=delete} & (2)\\

    \midrule
\end{tabular}

\begin{description}
    \item[1)] assigns the state of \cpp{other} to \cpp{*this} using move semantics
    \item[2)] copy assignment operator deleted
\end{description}

{\bfseries Parameters}
\begin{description}
    \item[other]   another execution context to assign to this object\\
\end{description}

{\bfseries Return value}
\begin{description}
    \item[*this]
\end{description}

\subparagraph*{operator bool}
test whether continuation is valid\\

\begin{tabular}{ l l }
    \midrule

    \cpp{explicit operator bool() const noexcept} & (1)\\

    \midrule
\end{tabular}

\begin{description}
    \item[1)] returns \cpp{true} if \cpp{*this} represents a context of
              execution, \cpp{false} otherwise.
\end{description}

{\bfseries Notes}
\newline
A \cont instance might not represent a context of execution for any of a
number of reasons.
\begin{itemize}
    \item It might have been default-constructed.
    \item It might have been assigned to another instance, or passed into a
          function.\\
          \cont instances are move-only.
    \item It might already have been resumed (\resume called) - calling \resume
          invalidates the instance.
    \item The \entryfn might have voluntarily terminated the
          context by returning.
\end{itemize}
The essential points:
\begin{itemize}
    \item Regardless of the number of \cont declarations, exactly one\\
          \cont instance represents each suspended context.
    \item No \cont instance represents the currently-running context.
\end{itemize}

\subparagraph*{operator!}
test whether continuation is invalid\\

\begin{tabular}{ l l }
    \midrule

    \cpp{bool operator\!() const noexcept} & (1)\\

    \midrule
\end{tabular}

\begin{description}
    \item[1)] returns \cpp{false} if \cpp{*this} represents a context of
              execution, \cpp{true} otherwise.
\end{description}

{\bfseries Notes}
\newline
See {\bfseries Notes} for \opbool.

\subparagraph*{(comparisons)}
establish an arbitrary total ordering for \cont instances\\

\begin{tabular}{ l l }
    \midrule

    \cpp{bool operator==(const continuation& other) const noexcept} & (1)\\

    \midrule

    \cpp{bool operator\!=(const continuation& other) const noexcept} & (1)\\

    \midrule

    \cpp{bool operator<(const continuation& other) const noexcept} & (2)\\

    \midrule

    \cpp{bool operator>(const continuation& other) const noexcept} & (2)\\

    \midrule

    \cpp{bool operator<=(const continuation& other) const noexcept} & (2)\\

    \midrule

    \cpp{bool operator>=(const continuation& other) const noexcept} & (2)\\

    \midrule
\end{tabular}

\begin{description}
    \item[1)] Every invalid \cont instance compares equal to every other
              invalid instance. But because the running context is never
              represented by a valid \cont instance, and because every
              suspended context is represented by exactly one valid
              instance, \emph{no valid instance can ever compare equal to any
              other valid instance.}
    \item[2)] These comparisons establish an arbitrary total ordering of \cont
              instances, for example to store in ordered containers. (However,
              key lookup is meaningless, since you cannot construct a search
              key that would compare equal to any entry.) There is no
              significance to the relative order of two instances.
\end{description}

\subparagraph*{swap}
swaps two \cont instances\\

\begin{tabular}{ l l }
    \midrule

    \cpp{void swap(continuation& other) noexcept} & (1)\\

    \midrule
\end{tabular}

\begin{description}
    \item[1)] Exchanges the state of \cpp{*this} with \cpp{other}.\\
\end{description}


\uabschnitt{std::callcc()}

create and enter a new context while the current execution context is wrapped
into a continuation and passed as argument to the new context\\

\begin{tabular}{ l l }
    \midrule

    \cpp{template< typename Fn, typename ...Args >}\\
    \cpp{continuation callcc( Fn && fn, Args ...args)} & (1)\\

    \midrule

    \cpp{template< typename StackAlloc, typename Fn, typename ...Args >}\\
    \cpp{continuation callcc( std::allocator\_arg\_t, StackAlloc salloc, Fn && fn, Args ...args)} & (2)\\

    \midrule
\end{tabular}

\begin{description}
    \item[1)] creates and immediately enters the new execution context
              (executing \cpp{fn}). The current execution context is suspended,
              wrapped in a continuation (\cont) and passed as argument to
              \cpp{fn}.
    \item[2)] takes a callable as argument, requirements as for (1). The stack
              is constructed using \emph{salloc}
              (see \nameref{subsec:stackalloc}).\footnote{This constructor, along with
              the \nameref{subsec:stackalloc} section, is an optional part of the
              proposal. It might be that implementations can reliably infer the
              optimal stack representation.}
\end{description}

{\bfseries Parameters}
\begin{description}
    \item[fn]      callable (function, lambda, functor) executed in the new
                   context; expected signature \cpp{continuation(continuation &&)} 
    \item[...args] data transferred to the new context - see section
                   \nameref{subsec:data}\\
\end{description}

{\bfseries Return value}
\begin{description}
    \item[continuation] the returned instance represents the execution context
                        (continuation) that has been suspended in order to
                        resume the current context
\end{description}

{\bfseries Exceptions}
\begin{description}
    \item[1)] calls \cpp{std::terminate} if an exception escapes \entryfn
              \cpp{fn}\\
\end{description}

{\bfseries Notes}
\newline
\call preserves the execution context of the calling context as well as stack
parts like \emph{parameter list} and \emph{return address}.
\footnote{required only by some x86 ABIs} Those data are restored if the calling
context is resumed.\\
A suspended \cpp{continuation} can be destroyed. Its resources will be cleaned
up at that time.\\
On return \cpp{fn} has to specify a \cont to which the execution control is
transferred.\\
If an instance with valid state goes out of scope and the \cpp{fn} has not yet
returned, the stack is traversed  and continuation's stack is deallocated.


\uabschnitt{std::resume()}

resumes a continuation\\

\begin{tabular}{ l l }
    \midrule

    \cpp{template< typename ...Args >}\\
    \cpp{continuation resume( continuation && c, Args ... args)} & (1)\\

    \midrule

    \cpp{template< typename Fn, typename ...Args >}\\
    \cpp{continuation resume( continuation && c, invoke\_ontop\_arg\_t, Fn && fn, Args ... args)} & (2)\\

    \midrule
\end{tabular}

\begin{description}
    \item[1)] suspends the active context, resumes continuation \cpp{c}
    \item[2)] suspends the active context, resumes continuation \cpp{c} but
              invokes \cpp{fn(args ...)} in the resumed context (on top of the
              last stack frame)
\end{description}

{\bfseries Parameters}
\begin{description}
    \item[c]       continuation that gets resumed
    \item[...args] passed to the resumed continuation - see section
                   \nameref{subsec:data}\\
\end{description}

{\bfseries Return value}
\begin{description}
    \item[continuation] the returned instance represents the execution context
                        (continuation) that has been suspended in order to
                        resume the current context
\end{description}

{\bfseries Exceptions}
\begin{description}
    \item[1)] calls \cpp{std::terminate} if an exception escapes \entryfn
              \cpp{fn}\\
\end{description}

{\bfseries Preconditions}
\begin{description}
    \item[1)] \cpp{c} represents a context of execution (\opbool returns
               \cpp{true})
    \item[2)] \cpp{any\_thread(c)} returns \cpp{true}, or the running thread is
              the same thread on which \cpp{c} ran previously.
\end{description}

{\bfseries Postcondition}
\begin{description}
    \item[1)] \cpp{c} is invalidated (\opbool returns \cpp{false})
\end{description}

{\bfseries Notes}
\newline
\resume preserves the execution context of the calling context as well as stack
parts like \emph{parameter list} and \emph{return address}.\footnote{required
only by some x86 ABIs} Those data are restored if the calling context is
resumed.
\newline
A suspended \cpp{continuation} can be destroyed. Its resources will be cleaned
up at that time.
\newline
The returned \cpp{continuation} indicates whether the suspended context
has terminated (returned from \entryfn) via \opbool. If the returned
\cpp{continuation} has terminated, no data may be retrieved.
\newline
Because \resume invalidates the instance on which it is called, \emph{no valid
\cont instance ever represents the currently-running context.}
\newline
When calling \resume, it is conventional to replace the newly-invalidated
instance -- the instance passed to \resume -- with the new instance
returned by that \resume call. This helps to avoid inadvertent calls to \resume
with the old, invalidated instance.


\uabschnitt{std::data\_available()}

test if data are present\\

\begin{tabular}{ l l }
    \midrule

    \cpp{bool data\_available( continuation && c)} & (1)\\

    \midrule
\end{tabular}

\begin{description}
    \item[1)] returns \cpp{true} if \call or \resume have been invoked with
              additional data as argument (\cpp{args})
\end{description}


\uabschnitt{std::get\_data()}

transfer of data\\

\begin{tabular}{ l l }
    \midrule

    \cpp{template< typename Arg >}\\
    \cpp{Arg get\_data( continuation && c)} & (1)\\

    \midrule

    \cpp{template< typename ...Args >}\\
    \cpp{std::tuple< Args... > get\_data( continuation && c)} & (2)\\

    \midrule
\end{tabular}

\begin{description}
    \item[1)] transfers single datum from continuation \cpp{c} into this context
    \item[2)] transfers multiple data from continuation \cpp{c} into this
              context
\end{description}

{\bfseries Notes}
\newline
The template argument(s) passed to \cpp{get\_data()} must match in number and
type the actual argument types passed to \call or \resume.

\uabschnitt{std::any\_thread()}

test whether suspended continuation may be resumed on a different thread\\

\begin{tabular}{ l l }
    \midrule

    \cpp{bool any\_thread( continuation const& c) const noexcept} & (1)\\

    \midrule
\end{tabular}

\begin{description}
    \item[1)] returns \cpp{false} if \cpp{c} must be resumed on the same
              thread on which it previously ran, \cpp{true} otherwise
\end{description}

{\bfseries Notes}
\newline
As stated in \nameref{subsec:main}, a \cont instance can represent the initial
context on which the operating system runs \main, or the context created by
the operating system for a new \cpp{std::thread}.

Attempting to resume such a \cont instance on any thread other than its
original thread invokes undefined behavior. \cpp{any\_thread()} allows
consumer code to distinguish this case by returning \cpp{false}.

\abschnitt{Performance of \cc}

On modern architectures suspending/resuming continuations takes very few CPU cycles.
\footnote{\cpp{callcc()} from boost.context takes 16 CPU cycles on Intel E5 2620 v4,
SYS V.}

\newpage
\abschnitt{Use case: userland threads}
\callcc can be used to implement userland threads. A userland thread resembles
a \cpp{std::thread} in that it is launched and proceeds more or less
independently. Its lifespan is not tied to the code that launched it.\\

The operating system kernel schedules \cpp{std::thread}s. When there are more
threads on the system than processor cores (which is frequently the case), it
gives every such thread a time slice, preemptively and transparently
suspending it once it has consumed that time slice.\\

In contrast, a userland thread does not engage the
operating system kernel to perform context-switching. The term ``userland
thread'' means that context is switched explicitly and cooperatively by code within the \\


For present purposes, we will use the term \bfs{fiber} to mean ``userland
thread.''\\

Asynchronous I/O \\

\uabschnitt{Why not use \cpp{std::thread}s?}
There are two reasons to prefer userland threads over \cpp{std::thread}s:
\begin{description}
  \item[Performance:] The Skynet benchmark results\cite{bfiberperf} illustrate
  that userland context-switching can be three orders of magnitude faster than
  kernel-mediated context-switching.
  \item[Scalability:] You can productively run many more userland threads in a
  single process than you could run kernel threads. The Skynet
  benchmark\cite{bfiberperf} tests a million concurrent tasks. It is not
  reasonable (even when possible) to launch that many kernel threads.
  Kernel-mediated context-switching overhead starts to overwhelm the
  processor.
\end{description}

\uabschnitt{Why not use \coawait?\cite{N4649}}
The \coawait facility is best suited for new code. The caller of
a \coawait function must itself be a \coawait function, and so
on all the way up to the launch point.\\

If you are modifying existing code to use \coawait, you quickly find
that introducing a new \coawait operation into a given function
requires transitively modifying any function that directly calls it, any
function that calls any of \emph{those...}\\

Modification for \coawait requires more than sprinkling \coawait
operations throughout your code base. It also typically requires altering the
signature of every affected function. A function invoked by \coawait
must be able to communicate to the \coawait operation whether it is
suspending or returning with a result. This is often communicated by using a
return type such as \cpp{std::future}, which can express the absence of data
and pass control back to the \coawait operation once data become
available.\\

Of course, as has been pointed out,\cite{N4045} \cpp{std::future} introduces
overhead of its own.\\

It is not usually emphasized that each call to a \coawait function
implies a \cpp{malloc()} call to obtain a heap activation frame;
each \cpp{return} implies a corresponding \cpp{free()} call. Under certain
circumstances -- specifically, the case of a function-local coroutine -- that
overhead can be optimized away. When \coawait is used to emulate
userland threads, it cannot.\\

One might consider the use of a memory pool for activation frames. This is an
excellent idea. A memory pool for activation frames is called a ``stack.''\\

\uabschnitt{Userland threads built on \cpp{callcc()}}
\begin{itemize}
\item Each fiber is reified as an object. That object contains the
  \cpp{continuation} representing its suspended context.
\item The object representing the currently-running fiber contains
  an invalid \cpp{continuation}.
\item The function that launches a fiber creates its context using
  \cpp{callcc()}.
\item Fiber objects are known to a central fiber manager object.
\item The fiber manager keeps fiber objects in separate containers: those
  still waiting for something else versus those that are ready
  to resume.
\item Instead of directly resuming a specific other fiber, the
  running fiber suspends by calling a scheduler to pick one of the ready
  fibers. (Note that the scheduler can execute on the context of the fiber
  about to suspend; we do not need two separate context switches.)
\end{itemize}

\abschnitt{Why not propose userland threads instead?}

Consider the following bullet from P0559R0:\cite{P0559R0}

\begin{itemize}
\item ``Prefer generality over specificity: prefer standardizing general
  building blocks on top of which domain-specific semantics can be layered, as
  opposed to domain-specific facilities on top of which other domain-specific
  semantics can't be layered.''
\end{itemize}

The \callcc facility proposed in this document is very low-level and very
general. With a public implementation of this facility,\cite{bcontext} the
author has built high-performance stackful coroutines\cite{bcoroutine2} and
high-performance userland threads\cite{bfiber}.\\

Both libraries, it should be noted, are built in portable C++ on top of the
\cpp{callcc()} and \cpp{continuation} API. The \cpp{callcc()}-based
implementation gives the best performance yet\cite{bfiberperf} for each of
these libraries.\\

The API permits still other higher-level abstractions too. The author has also
prototyped an implementation of delimited continuations (\shift and \reset
operators).
\newpage
\abschnitt{Use case: stackful coroutines}

\uabschnitt{Stackful coroutines built on \cpp{callcc()}}
\begin{itemize}
\item Symmetric coroutines map very directly to \cpp{callcc()} functionality.
  Each coroutine is reified as an object. The object contains the
  \cpp{continuation} representing its suspended context -- or, if that
  coroutine is currently running, an invalid \cpp{continuation}.
\item A symmetric coroutine suspends by specifying a particular other
  coroutine object to resume. The implementation calls \resume on that other
  coroutine object's \cpp{continuation}.
\item An asymmetric coroutine ``knows'' its invoker: rather than explicitly
  resuming an arbitrary other coroutine, it \emph{yields,} implicitly resuming its
  invoker. (In just the same way, \cpp{return} implicitly resumes a function's
  caller.)
\item An asymmetric coroutine object could contain a reference to its
  invoker's coroutine object, permitting an anonymous \emph{yield} operation.
\end{itemize}

\abschnitt{Why not propose stackful coroutines instead?}

In fact -- we \emph{did!}\cite{N3708}\citecomma\cite{N3985} We were directed
to bring back a lower-level proposal. That lower-level proposal has evolved to
this present form.
\newpage
\abschnitt{Use case: many small stacks, one deep stack}
Proponents of \coawait frequently describe a particular execution
environment: a 32-bit Windows server process supporting millions of clients in
a transiently-stateful way, preserving some amount of state data across some
number of asynchronous operations.\\

It is pointed out that calls to opaque library, runtime and operating-system
functions may consume arbitrary amounts of stack space. The inability to
predict stack consumption in advance leads the Windows operating system to
allocate a 1MB stack for each kernel thread.\\

Of course, that stack memory is not committed until actually used. Still, it
does present a problem: in a 32-bit process, you quickly run out of address
space. You are constrained to no more than\\

$ \frac{2^{32} - (size\ of\ all\ code) - (size\ of\ all\ other\ data)}{stack\ size} $\\

stacks. Even if you set both \cpp{(size of all code)} and \cpp{(size of all other
data)} to zero -- impossible, in practice -- you can allocate no more than 4096
1MB stacks. 4096 is very much smaller than ``millions.''\\

On another operating system, one could use segmented stacks, but those are not
supported on Windows.\\

In a 64-bit process, the limitation would be actual memory consumption rather
than the address space. But it is suggested that many people still use 32-bit
server processes.\\

When considering userland threads, we might not be quite as conservative as
the Windows operating system. We might decide that we need far less stack
space than 1MB per userland thread. We might be so bold as to suggest 16KB
stacks. But that \emph{still} is constrained to a theoretical maximum of
262144 stacks -- and that's without code or any other data. This falls short
of ``millions'' by at least an order of magnitude.\\

We might be certain that our own code requires very little stack space -- even
less than 16KB. But does our code ever call library functions? runtime
functions? operating-system functions? How much stack space do \emph{they}
consume? This brings us back to the original unanswerable question.\\

Proponents of the \coawait facility explain that since each \coawait function
allocates a separate heap activation frame -- in effect, each has its own tiny
stack -- the thread's main stack is left largely untouched. Calls to opaque
library or runtime or operating-system functions that require arbitrary stack
space consume the thread's main stack, which is presumed to be Big Enough.\\

With \cc, we can use a similar trick. We can construct each new context with a
very small stack: just big enough for the function calls in our own code. We
can set aside one ``big enough'' stack as a common resource, shunting function
calls of unknown depth onto the shared ``big enough'' stack.\\

We assume that opaque functions of this kind will not themselves suspend.
Please note that the \coawait scenario requires the same assumption.\\

\cppf{deepstack}


\abschnitt{Why \cc should be preferred over \uc or \lj}

\uabschnitt{stack represents the continuation}

In contrast to \uc, \cc uses the stack as storage for the suspended
execution context (the content of the registers, see P0534R0\cite{P0534R0}).

\begin{itemize}
    \item only the target has to be provided at resumption
        (\cpp{swapcontext()} required source and target)
    \item current execution context is already represented by the
        stack to which the stack-pointer points
    \item suspended execution context is passed as continuation (parameter) 
        to the resume operation
    \item no need for a global pointer that points to the current execution context
    \item data are transferred through a function call, no need for a global pointer
    \item \main and each thread's \entryfn fit seamlessly into the concept of \cc
        because the stack of \main, or the thread, already represents the continuation of
        that context
\end{itemize}

\uabschnitt{aggregation of stack address}

A instance of \cont contains the stack address of a suspended execution.
\cont:

\begin{itemize}
    \item represents the continuation of a suspended context
    \item prevents accidentally copying the stack
    \item prevents accidentally resuming the running execution
        context
    \item prevents accidentally resuming an execution context that has already
        terminated (computation has finished)
    \item manages lifespan of an explicitly-allocated stack: the stack gets
        deallocated when \cont goes out of scope
    \item context switch and data transfer via one function call
\end{itemize}

Of course a \uc-like standard API would be possible, but in C++ we can do much
better with very little abstraction cost.

\uabschnitt{Disadvantages of \uc}

\begin{itemize}
    \item deprecated since POSIX.1-2004d and removed in POSIX.1-2008
    \item \cpp{makecontext} violates C99 standard (function pointer cast and integer arguments)
    \item \cpp{makecontext} arguments in var-arg list are required to be integers; passing pointers
        is not guaranteed to work (especially on platforms where pointers are larger than integers)
    \item \cpp{swapcontext} calls into the kernel, consuming many CPU cycles (two orders
        of magnitude)
    \item does not prevent accidentally copying the stack
    \item does not prevent accidentally resuming the running execution
        context
    \item does not prevent accidentally resuming an execution context that has already
        terminated (computation has finished)
    \item does not manage lifespan of an explicitly-allocated stack
    \item context switch (\cpp{swapcontext}) does not transfer data (requires global pointer)
\end{itemize}

\uabschnitt{Disadvantages of \lj}

\begin{itemize}
    \item C99 defines \sj / \lj for non-local jumps
    \item \lj is not required to preserve the current stack frame, therefore jumping into a function
        which has exited (by return or by a different \lj higher up the stack) is undefined behavior:
        only \lj up the call stack is allowed
    \item does not prevent accidentally copying the stack
    \item does not prevent accidentally resuming the running execution
        context
    \item does not prevent accidentally resuming an execution context that has already
        terminated (computation has finished)
    \item does not manage lifespan of an explicitly-allocated stack
    \item context switch (\lj) does not transfer data (requires global pointer)
\end{itemize}

\abschnitt{Why \cc is low-level}

\cc is a low-level implementation. \cont has the memory footprint of a pointer
because it aggregates a pointer to the stack of the managed continuation.\\
\newline
Functions expressed in assembler, consit of a \emph{function prologue} at the
beginning of a function and \emph{function epilogue} at the end of the function.
\cppf{function}

The \emph{prologue} (few lines of assembler) that prepares the stack and
registers for use inside the function. The \emph{epilogue} restores the stack
and registers\footnote{callee-saved registers as defined by the calling
convention} to the state they were before the function was called.\\
Between prologue and epilogue the computation and calls to sub-routines are
done.\\
\newline
Like ordinary functions \resume and \resumewith consist of prologue and
epilogue. The only difference to oridnary functions is, that \resume and
\resumewith additionally exchange the stack- and instruction pointer
\footnote{In fact on x86-architecture the instructions- pointer return-address
remains already on the stack. On some RISC-architectures the link-register
has to be preserved on the stack while the context is suspended and is loaded
into the instruction-pointer on resumption.} between proloque and epiloque.\\
The prologue and epilogue of \resume and \resumewith neither consume stack space
nor do they call sub-routines, only the stack-pointer is exchanged.\\
\newline
On modern architectures \callcc takes view CPU cycles (6-8 CPU cycles on Intel E5 2620).
\cppf{implementation}

\newpage
\abschnitt{API}\label{api}

\uabschnitt{std::continuation}
declaration of class \cont
\cppf{continuation}
\paragraph*{member functions}

\subparagraph*{(constructor)}
constructs new continuation\\

\begin{tabular}{ l l }
    \midrule

    \cpp{continuation() noexcept} & (1)\\

    \midrule

    \cpp{continuation(continuation&& other)} & (2)\\

    \midrule

    \cpp{continuation(const continuation& other)=delete} & (3)\\

    \midrule
\end{tabular}

\begin{description}
    \item[1)] This constructor instantiates an invalid \cont. Its \opbool
              returns \cpp{false}; its \cpp{operator\!()} returns \cpp{true}.
    \item[2)] moves underlying state to new \cont
    \item[3)] copy constructor deleted
\end{description}

{\bfseries Notes}
\begin{description}
\item Every valid \cont instance is synthesized by the underlying facility -- or
move-constructed, or move-assigned, from another valid instance. There is
no \cont constructor that directly constructs a valid \cont instance.
\item The entry-function \cpp{fn} passed to \callcc is passed a synthesized \cont
instance representing the suspended caller of \callcc.
\item The entry-function \cpp{fn} passed to \resumewith is passed a
synthesized \cont instance representing the suspended caller of \resumewith.
\item \callcc returns a synthesized \cont representing the previously-executing
context, the context that suspended in order to resume the caller of \callcc. The
returned \cont instance \emph{might} represent the context created by \callcc, but
need not: the context created by \callcc might have created (or resumed) yet
another context, which might then have resumed the caller of \callcc.
\item Similarly, \resume returns a synthesized \cont instance representing the
previously-executing context, the context that suspended in order to resume
the caller of \resume.
\item Similarly, \resumewith returns a synthesized \cont instance representing
the previously-executing context, the context that suspended in order to
resume the caller of \resumewith.
\end{description}

\subparagraph*{(destructor)}\label{subpara:destructor}
destroys a continuation\\

\begin{tabular}{ l l }
    \midrule

    \dtor & (1)\\

    \midrule
\end{tabular}

\begin{description}
    \item[1)] destroys a \cont instance. If this instance represents a context
              of execution (\opbool returns \cpp{true}), then the context of
              execution is destroyed too. Specifically, the stack is unwound
              by throwing \unwindex.\footnote{ In a program in which exceptions are thrown, it is
              prudent to code a context's \entryfn with a last-ditch
              \cpp{catch (...)} clause: in general, exceptions must
              \emph{not} leak out of the \entryfn. However, since
              stack unwinding is implemented by throwing an
              exception, a correct \entryfn\ \cpp{try} statement
              must also \cpp{catch (std::unwind\_exception const&)} and rethrow it.}
\end{description}


\subparagraph*{operator=}
moves the continuation object\\

\begin{tabular}{ l l }
    \midrule

    \cpp{continuation& operator=(continuation&& other)} & (1)\\

    \midrule

    \cpp{continuation& operator=(const continuation& other)=delete} & (2)\\

    \midrule
\end{tabular}

\begin{description}
    \item[1)] assigns the state of \cpp{other} to \cpp{*this} using move semantics
    \item[2)] copy assignment operator deleted
\end{description}

{\bfseries Parameters}
\begin{description}
    \item[other]   another execution context to assign to this object\\
\end{description}

{\bfseries Return value}
\begin{description}
    \item[*this]
\end{description}


\subparagraph*{resume()}
resumes a continuation\\

\begin{tabular}{ l l }
    \midrule

    \cpp{template< typename ...Args >}\\
    \cpp{continuation resume( Args ... args)} & (1)\\

    \midrule

    \cpp{template< typename Fn, typename ...Args >}\\
    \cpp{continuation resume\_with( Fn && fn, Args ... args)} & (2)\\

    \midrule
\end{tabular}

\begin{description}
    \item[1)] suspends the active context, resumes continuation \cpp{*this}
    \item[2)] suspends the active context, resumes continuation \cpp{*this} but
              invokes \cpp{fn(args ...)} in the resumed context (on top of the
              last stack frame)
\end{description}

{\bfseries Parameters}
\begin{description}
    \item[...args] passed to the resumed continuation - see section
                   \nameref{subsec:data}
    \item[fn] function invoked ontop of resumed continuation\\
\end{description}

{\bfseries Return value}
\begin{description}
    \item[continuation] the returned instance represents the execution context
                        (continuation) that has been suspended in order to
                        resume the current context
\end{description}

{\bfseries Exceptions}
\begin{description}
    \item[1)] \resume or \resumewith might
              throw \unwindex if, while suspended, the
              calling context is destroyed
    \item[2)] \resume or \resumewith might throw \emph{any}
              exception if, while suspended:
        \begin{itemize}
            \item some other context calls \resumewith to resume
              this suspended context
            \item the function \cpp{fn} passed to \resumewith --
              or some function called by \cpp{fn} -- throws an exception
        \end{itemize}
    \item[3)] any exception thrown by the function \cpp{fn} passed
              to \resumewith, or any function called by \cpp{fn}, is thrown in
              the context referenced by \cpp{*this} rather than in the context
              of the caller of \resumewith
\end{description}

{\bfseries Preconditions}
\begin{description}
    \item[1)] \cpp{*this} represents a context of execution (\opbool returns
               \cpp{true})
    \item[2)] \cpp{any\_thread()} returns \cpp{true}, or the running thread is
              the same thread on which \cpp{*this} ran previously.
\end{description}

{\bfseries Postcondition}
\begin{description}
    \item[1)] \cpp{*this} is invalidated (\opbool returns \cpp{false})
\end{description}

{\bfseries Notes}
\newline
\resume preserves the execution context of the calling context as well as stack
parts like \emph{parameter list} and \emph{return address}.\footnote{required
only by some x86 ABIs} Those data are restored if the calling context is
resumed.
\newline
A suspended \cpp{continuation} can be destroyed. Its resources will be cleaned
up at that time.
\newline
The returned \cpp{continuation} indicates whether the suspended context
has terminated (returned from \entryfn) via \opbool. If the returned
\cpp{continuation} has terminated, no data may be retrieved.
\newline
Because \resume invalidates the instance on which it is called, \emph{no valid
\cont instance ever represents the currently-running context.}
\newline
When calling \resume, it is conventional to replace the newly-invalidated
instance -- the instance on which \resume was called -- with the new instance
returned by that \resume call. This helps to avoid inadvertent calls to \resume
on the old, invalidated instance.


\subparagraph{data\_available()}
test if data are present\\

\begin{tabular}{ l l }
    \midrule

    \cpp{bool data\_available()} & (1)\\

    \midrule
\end{tabular}

\begin{description}
    \item[1)] returns \cpp{true} if \callcc or \resume have been invoked with
              additional data as argument (\cpp{args})
\end{description}


\subparagraph{get\_data()}
transfer of data\\

\begin{tabular}{ l l }
    \midrule

    \cpp{template< typename Arg >}\\
    \cpp{Arg get\_data()} & (1)\\

    \midrule

    \cpp{template< typename ...Args >}\\
    \cpp{std::tuple< Args... > get\_data()} & (2)\\

    \midrule
\end{tabular}

\begin{description}
    \item[1)] transfers single datum from continuation \cpp{c} into this context
    \item[2)] transfers multiple data from continuation \cpp{c} into this
              context
\end{description}

{\bfseries Notes}
\newline
The template argument(s) passed to \cpp{get\_data()} must match in number and
type the actual argument types passed to \callcc or \resume.


\subparagraph{any\_thread()}
test whether suspended continuation may be resumed on a different thread\\

\begin{tabular}{ l l }
    \midrule

    \cpp{bool any\_thread() const noexcept} & (1)\\

    \midrule
\end{tabular}

\begin{description}
    \item[1)] returns \cpp{false} if \cpp{c} must be resumed on the same
              thread on which it previously ran, \cpp{true} otherwise
\end{description}

{\bfseries Notes}
\newline
As stated in \nameref{subsec:main}, a \cont instance can represent the initial
context on which the operating system runs \main, or the context created by
the operating system for a new \cpp{std::thread}.

Attempting to resume such a \cont instance on any thread other than its
original thread invokes undefined behavior. \cpp{any\_thread()} allows
consumer code to distinguish this case by returning \cpp{false}.


\subparagraph*{operator bool}
test whether continuation is valid\\

\begin{tabular}{ l l }
    \midrule

    \cpp{explicit operator bool() const noexcept} & (1)\\

    \midrule
\end{tabular}

\begin{description}
    \item[1)] returns \cpp{true} if \cpp{*this} represents a context of
              execution, \cpp{false} otherwise.
\end{description}

{\bfseries Notes}
\newline
A \cont instance might not represent a context of execution for any of a
number of reasons.
\begin{itemize}
    \item It might have been default-constructed.
    \item It might have been assigned to another instance, or passed into a
          function.\\
          \cont instances are move-only.
    \item It might already have been resumed (\resume called) - calling \resume
          invalidates the instance.
    \item The \entryfn might have voluntarily terminated the
          context by returning.
\end{itemize}
The essential points:
\begin{itemize}
    \item Regardless of the number of \cont declarations, exactly one\\
          \cont instance represents each suspended context.
    \item No \cont instance represents the currently-running context.
\end{itemize}


\subparagraph*{operator!}
test whether continuation is invalid\\

\begin{tabular}{ l l }
    \midrule

    \cpp{bool operator\!() const noexcept} & (1)\\

    \midrule
\end{tabular}

\begin{description}
    \item[1)] returns \cpp{false} if \cpp{*this} represents a context of
              execution, \cpp{true} otherwise.
\end{description}

{\bfseries Notes}
\newline
See {\bfseries Notes} for \opbool.

\subparagraph*{(comparisons)}
establish an arbitrary total ordering for \cont instances\\

\begin{tabular}{ l l }
    \midrule

    \cpp{bool operator==(const continuation& other) const noexcept} & (1)\\

    \midrule

    \cpp{bool operator\!=(const continuation& other) const noexcept} & (1)\\

    \midrule

    \cpp{bool operator<(const continuation& other) const noexcept} & (2)\\

    \midrule

    \cpp{bool operator>(const continuation& other) const noexcept} & (2)\\

    \midrule

    \cpp{bool operator<=(const continuation& other) const noexcept} & (2)\\

    \midrule

    \cpp{bool operator>=(const continuation& other) const noexcept} & (2)\\

    \midrule
\end{tabular}

\begin{description}
    \item[1)] Every invalid \cont instance compares equal to every other
              invalid instance. But because the running context is never
              represented by a valid \cont instance, and because every
              suspended context is represented by exactly one valid
              instance, \emph{no valid instance can ever compare equal to any
              other valid instance.}
    \item[2)] These comparisons establish an arbitrary total ordering of \cont
              instances, for example to store in ordered containers. (However,
              key lookup is meaningless, since you cannot construct a search
              key that would compare equal to any entry.) There is no
              significance to the relative order of two instances.
\end{description}


\subparagraph*{swap}
swaps two \cont instances\\

\begin{tabular}{ l l }
    \midrule

    \cpp{void swap(continuation& other) noexcept} & (1)\\

    \midrule
\end{tabular}

\begin{description}
    \item[1)] Exchanges the state of \cpp{*this} with \cpp{other}.\\
\end{description}


\uabschnitt{std::callcc()}

create and enter a new context, capturing the current execution context (the
{\bfseries current continuation}) in a \cont and passing it to the
specified \entryfn.\\
\callcc acts as a factory-function: it creates and starts a new execution context
(stack etc.) and returns a continuation that represents the rest of the execution
context's computation.\\
\callcc explicitly expresses the creation of a new execution
context and the switch to the other execution path.\\

\begin{tabular}{ l l }
    \midrule

    \cpp{template< typename Fn, typename ...Args >}\\
    \cpp{continuation callcc( Fn && fn, Args ...args)} & (1)\\

    \midrule

    \cpp{template< typename StackAlloc, typename Fn, typename ...Args >}\\
    \cpp{continuation callcc( std::allocator\_arg\_t, StackAlloc salloc, Fn && fn, Args ...args)} & (2)\\

    \midrule
\end{tabular}

\begin{description}
    \item[1)] creates and immediately enters the new execution context
              (executing \cpp{fn}). The current execution context is suspended,
              wrapped in a continuation (\cont) and passed as argument to
              \cpp{fn}.
    \item[2)] takes a callable as argument, requirements as for (1). The stack
              is constructed using \emph{salloc}
              (see \nameref{subsec:stackalloc}).
\end{description}

{\bfseries Parameters}
\begin{description}
    \item[fn]      callable (function, lambda, functor) executed in the new
                   context; expected signature \cpp{continuation(continuation &&)} 
    \item[...args] data transferred to the new context - see section
                   \nameref{subsec:data}\\
\end{description}

{\bfseries Return value}
\begin{description}
    \item[continuation] the returned instance represents the execution context
                        (continuation) that was suspended in order to
                        resume the current context
\end{description}

{\bfseries Exceptions}
\begin{description}
    \item[1)] calls \cpp{std::terminate} if an exception other
              than \unwindex escapes \entryfn\ \cpp{fn}
    \item[2)] \callcc might throw \unwindex if,
              while suspended, the calling context is destroyed
    \item[3)] \callcc might throw \emph{any} exception if, while
              suspended:
        \begin{itemize}
            \item some other context calls \resumewith to resume
              this suspended context
            \item the function \cpp{fn} passed to \resumewith --
              or some function called by \cpp{fn} -- throws an exception
        \end{itemize}
\end{description}

{\bfseries Notes}
\begin{description}
\item \callcc preserves the execution context of the calling context as well as stack
parts like \emph{parameter list} and \emph{return address}.\footnote{required
only by some x86 ABIs} Those data are restored if the calling context is resumed.
\item A suspended \cpp{continuation} can be destroyed. Its resources will be cleaned
up at that time.
\item On return \cpp{fn} must specify a \cont to which execution control is
transferred.
\item If an instance with valid state goes out of scope and its \cpp{fn} has not yet
returned, the stack is unwound and deallocated.
\end{description}

\uabschnitt{std::terminate\_context\_then()}

terminate the current running context, switching to the context represented by
the passed \cont. This is like returning that \cont from the \entryfn, but may
be called from any function on that context.

\begin{tabular}{ l l }
    \midrule

    \cpp{void terminate\_context\_then( continuation && cont )} & (1)\\

    \midrule
\end{tabular}

\begin{description}
    \item[1)] throws \unwindex, binding the passed \cont. The running
              context's first stack entry -- the one created by \callcc --
              catches \unwindex, extracts the bound \cont and terminates the
              current context by returning that \cont.
\end{description}

\bfs{Parameters}
\begin{description}
    \item[cont] the \cont to which to switch once the current context has terminated
\end{description}

\bfs{Preconditions}
\begin{description}
    \item[1)] \cpp{cont} must be valid (\cpp{operator bool()} returns \cpp{true})
\end{description}

\bfs{Return value}
\begin{description}
    \item[1)] None: \termthen does not return
\end{description}

\bfs{Exceptions}
\begin{description}
    \item[1)] throws \unwindex
\end{description}

\newpage
\abschnitt{Additional notes}

\uabschnitt{GPU}

\cc as proposed in this paper does not take GPUs into account. Later revisions
will address this issue, once we have an overarching concept of how the various
kinds of ``lightweight execution agents'' should interact.


\uabschnitt{SIMD}

does not interfere with \cc and can be used as usual (\cc triggers the context
switch at its invocation).\\
Of course, depending on the calling convention, some micro-processor registers,
dedicated to SIMD, might be preserved and restored too
\footnote{\emph{MS Windows x64} calling convention}.


\uabschnitt{TLS}

\cc is TLS-agnostic - best practice related to TLS applies to \cc too.\\
As shown in \nameref{mechanism}, \cc only preserves and restores
micro-processor registers at its invocation.


\uabschnitt{Migration between threads}

\cont can be migrated between threads, except for instances of
\cont representing \main or \entryfn of a thread (see \nameref{subsec:main}).


%//////////////////////////////////////////////////////////////////////////////

\addcontentsline{toc}{subsection}{References}
\begin{thebibliography}{99}

    \bibitem{P0099R1}
        \href{http://www.open-std.org/jtc1/sc22/wg21/docs/papers/2016/p0099r1.pdf}
        {P0099R1: A low-level API for stackful context switching}

    \bibitem{schemecallcc}
        \href{http://community.schemewiki.org/?call-with-current-continuation}
        {call/cc in Scheme}

    \bibitem{rubycallcc}
        \href{http://gnuu.org/2009/03/21/demystifying-continuations-in-ruby}
        {call/cc in Ruby}

    \bibitem{SYSVABI}
        {System V Application Binary Interface, AMD64 Architecture Processor Supplement,
        Draft Version 0.96}

    \bibitem{N3708}
        \href{http://www.open-std.org/jtc1/sc22/wg21/docs/papers/2013/n3708.pdf}
        {N3708: A proposal to add coroutines to the C++ standard library}

    \bibitem{gccsplit}
        \href{http://gcc.gnu.org/wiki/SplitStacks}
        {Split Stacks / GCC}

    \bibitem{bcontext}
        \href{http://www.boost.org/doc/libs/release/libs/context/doc/html/index.html}
        {Library \emph{Boost.Context}} (\cc available int boost-1.64)

    \bibitem{bcoroutine2}
        \href{http://www.boost.org/doc/libs/release/libs/coroutine2/doc/html/index.html}
        {Library \emph{Boost.Coroutine2}}

    \bibitem{bfiber}
        \href{http://www.boost.org/doc/libs/release/libs/fiber/doc/html/index.html}
        {Library \emph{Boost.Fiber}}

    \bibitem{Ferguson}
        {Darrell Ferguson and Dwight Deugo, Call with Current Continuation Patterns.
        in 8th Conference on Pattern Languages of Programs (PLoP 2001) , 2001}

\end{thebibliography}


%//////////////////////////////////////////////////////////////////////////////

\end{document}
