\newpage
\abschnitt{Use case: userland threads}
\callcc can be used to implement userland threads. A userland thread resembles
a \cpp{std::thread} in that it is launched and proceeds more or less
independently. Its lifespan is not tied to the code that launched it.\\

The operating system kernel schedules \cpp{std::thread}s. When there are more
threads on the system than processor cores (which is frequently the case), it
gives every such thread a time slice, preemptively and transparently
suspending it once it has consumed that time slice.\\

In contrast, a userland thread does not create additional work for the kernel
scheduler. As far as the kernel is concerned, that userland thread is simply
the work the kernel thread is choosing to do at this moment. When context
switches from one userland thread to another, the kernel is not engaged.\\

For present purposes, we will use the term \bfs{fiber} to mean ``userland
thread.''\\

A fiber suspends by calling its own (userland) scheduler. The scheduler
decides which of the other fibers are ready to run, picks one and resumes
it. Typically a suspending fiber is waiting \emph{for} something: a timer,
asynchronous I/O, work performed by another fiber or another thread. The
suspending fiber first arranges to be notified when the wait is over. That
notification informs the scheduler that this fiber is once again ready to run.\\

Because each fiber has a C++ stack, the details of arranging notification and
suspending can be encapsulated in a lower-level function. Consuming code calls
that function, which returns with expected results once they become available.
The fact that the lower-level function suspended, allowing other fibers to run
in the meantime, is transparent to the caller.\\

In effect, fibers provide a way to organize code running within a kernel
thread. Fibers provide no more parallelism than setting up a chain of
callbacks, which is the conventional way to orchestrate a sequence of
asynchronous operations. But code running on a fiber looks and behaves as if
the underlying operations were classic blocking operations.\footnote{In a
sense, they are -- but they block only the calling \emph{fiber,} not the
entire host \emph{thread.}}\\

Fibers permit structuring your code into layers of abstraction, even when the
underlying I/O is asynchronous. Chains of callbacks do not permit that.\\

Fibers permit ordinary use of stack variables. Chains of callbacks do not
permit that.\\

C++ code based on asynchronous operations can be (and has been) written in a
startling variety of different ways. But when such code is written using
fibers, it is both easier to write and easier to reason about -- thus easier
to maintain, and therefore more robust.\\

Asynchronous I/O is becoming increasingly pervasive in modern computer
systems. It's clear that we need some better tactic than chains of callbacks,
state machines, big switch statements, Duff's Device, etc. etc.\\

Fibers allow us to use the native control structures provided by C++ itself to
organize and maintain our code.\\

The Boost.Fiber library\cite{bfiber} is an implementation of fibers based on
the \cpp{callcc()} implementation in Boost.Context.\cite{bcontext}

\uabschnitt{Why not use \cpp{std::thread}s?}
There are two reasons to prefer userland threads over \cpp{std::thread}s:
\begin{description}
  \item[Performance:] The Skynet benchmark results\cite{bfiberperf} illustrate
  that userland context-switching can be three orders of magnitude faster than
  kernel-mediated context-switching.
  \item[Scalability:] You can productively run many more userland threads in a
  single process than you could run kernel threads. The Skynet
  benchmark\cite{bfiberperf} tests a million concurrent tasks. It is not
  reasonable (even when possible) to launch that many kernel threads.
  Kernel-mediated context-switching overhead starts to overwhelm the
  processor.
\end{description}

\uabschnitt{Why not use \coawait?\cite{N4649}}
The \coawait facility is best suited for new code. The caller of
a \coawait function must itself be a \coawait function, and so
on all the way up to the launch point.\\

If you are modifying existing code to use \coawait, you quickly find that
introducing a new \coawait operation into a given function requires
transitively modifying every function that directly calls it, then also
modifying every function that calls any of \emph{those...}\\

Modification for \coawait requires more than sprinkling \coawait
operations throughout your code base. It also typically requires altering the
signature of every affected function. A function invoked by \coawait
must be able to communicate to the \coawait operation whether it is
suspending or returning with a result. This is often communicated by using a
return type such as \cpp{std::future}, which can express the absence of data
and pass control back to the \coawait operation once data become
available.\\

Of course, as has been pointed out,\cite{N4045} \cpp{std::future} introduces
overhead of its own.\\

It is not usually emphasized that each call to a \coawait function
implies a \cpp{malloc()} call to obtain a heap activation frame;
each \cpp{return} implies a corresponding \cpp{free()} call. Under certain
circumstances -- specifically, the case of a function-local coroutine -- that
overhead can be optimized away. When \coawait is used to emulate
userland threads, it cannot.\\

One might consider the use of a memory pool for activation frames. This is an
excellent idea. A memory pool for activation frames is called a ``stack.''\\

\uabschnitt{Userland threads built on \cpp{callcc()}}
\begin{itemize}
\item Each fiber is reified as an object. That object contains the
  \cpp{continuation} representing its suspended context.
\item The object representing the currently-running fiber contains
  an invalid \cpp{continuation}.
\item The function that launches a fiber creates its context using
  \cpp{callcc()}.
\item Fiber objects are known to a central fiber manager object.
\item The fiber manager keeps fiber objects in separate containers: those
  still waiting for something else versus those that are ready
  to resume.
\item Instead of directly resuming a specific other fiber, the
  running fiber suspends by calling a scheduler to pick one of the ready
  fibers. (Note that the scheduler can execute on the context of the fiber
  about to suspend; we do not need two separate context switches.)
\end{itemize}

\abschnitt{Why not propose userland threads instead?}

Consider the following bullet from P0559R0:\cite{P0559R0}

\begin{itemize}
\item ``Prefer generality over specificity: prefer standardizing general
  building blocks on top of which domain-specific semantics can be layered, as
  opposed to domain-specific facilities on top of which other domain-specific
  semantics can't be layered.''
\end{itemize}

The \callcc facility proposed in this document is very low-level and very
general. With a public implementation of this facility,\cite{bcontext} the
author has built high-performance stackful coroutines\cite{bcoroutine2} and
high-performance userland threads\cite{bfiber}.\\

Both libraries, it should be noted, are built in portable C++ on top of the
\cpp{callcc()} and \cpp{continuation} API. The \cpp{callcc()}-based
implementation gives the best performance yet\cite{bfiberperf} for each of
these libraries.\\

The API permits still other higher-level abstractions too. The author has also
prototyped an implementation of delimited continuations (\shift and \reset
operators).
\newpage
\abschnitt{Use case: stackful coroutines}

The Boost.Coroutine2 library\cite{bcoroutine2} is an implementation of
stackful coroutines based on the \cpp{callcc()} implementation in
Boost.Context.\cite{bcontext} Both symmetric and asymmetric coroutines are
supported.\\

With this API, a \cpp{pull\_type} asymmetric coroutine functions as a
generator: it is resumed each time its invoker requests the next data item.
The library provides input iterator support for \cpp{pull\_type} coroutines;
such a generator may provide data to any range operation compatible with input
iterators, for example range \cpp{for}.\\

Similarly, a \cpp{push\_type} asymmetric coroutine is a data sink. It is
resumed each time its invoker passes the next data item. The library provides
output iterator support for \cpp{push\_type} coroutines.\\

The API permits chaining\cite{bcoroutinechaining} of coroutines. Indeed, a
function that accepts both \cpp{pull\_type} and \cpp{push\_type} coroutine
endpoints can be chained without having to care whether it's in an ``input
chain'' or an ``output chain.''\\

Invoking code instantiates a coroutine, then engages with it either by
requesting data items or passing data items to it.\\

\uabschnitt{Why not use \coawait?\cite{N4649}}

This kind of use case is actually where \coawait shines. Significant effort
has already been invested in compiler optimizations that can make a small
local \coawait coroutine ``vanish.''\\

Even so, there remain significant cases that cannot readily be handled
by \coawait. Generally speaking, there are times when it's very useful to run
your processing on a whole separate stack.\\

A few examples of such cases:\\

\begin{description}
\item[Recursive SAX parsing] There are two common ways to parse an
  XML document.\\

  With \bfs{DOM} parsing, you read the entire document into memory as a linked
  data structure. You may then traverse that structure any way you want,
  including recursively. Of course, the caveat is that you must have enough
  free memory to represent the entire document.\\

  With \bfs{SAX} parsing, the document is streamed. Application code is
  notified (via callbacks) when the start and end of each element is
  encountered. The benefit is that this can handle XML documents of arbitrary
  size: you need not store the entire document in memory at the same time. The
  drawback is that you must manually track element nesting.\\

  A C++Now 2014 talk\cite{ohmy} illustrates how stackful coroutines permit you
  to use an off-the-shelf SAX parser to structure your processing code as
  recursion over an XML document of arbitrary size being streamed through your
  application.\\

  Essentially, it permits passing callback events from the SAX parser into
  recursive processing code. Having two different stacks allows the SAX parser
  to call application callbacks, which then return to the parser to continue,
  while the application performs recursive processing on the other stack.\\

  Attempting to use \coawait for this would require rewriting the SAX parser.
\item[Lazy visitor-based processing] Even when you do have an entire data
  structure in memory, a particular library might use callbacks -- or visitors
  -- to engage application-specific processing.\\

  The Boost.Graph library\cite{bgraph} provides a number of useful graph
  traversals and processing algorithms. Its API relies heavily on
  visitors.\cite{bgraphvisitor}\\

  The problem with that design is that the Boost.Graph algorithm drives the
  traversal. The visitor is called for every applicable node or edge on the
  graph. What if you want to stop? What if you want to pause? What if you want
  to proceed stepwise?\\

  For example, consider the problem of finding a route from node A to node B
  in a densely-connected graph. You might consider taking a step outward from
  node A, taking a step outward from node B, taking another from node A, and
  so forth, until the two traversals reach the same node.\\

  This is unreasonably difficult using visitors.\\

  A C++Now 2016 talk\cite{visitors} illustrates how stackful coroutines permit
  you to run two concurrent Boost.Graph traversal algorithms on separate
  stacks, ``pulling'' visitor events from each traversal on demand,
  interleaving calls to the two traversals.\\

  Attempting to use \coawait for this would require rewriting the Boost.Graph
  algorithm of interest.
\item[Connecting output iterator to input iterator] In February 2017, a
  question was asked on the Boost developers' mailing list:\cite{lazyspirit}
  Can a Boost.Spirit\cite{bspirit} Karma\cite{bspiritkarma} generator, which
  takes an output iterator, be engaged lazily? Can consuming code request
  one item at a time?\\

  The answer is straightforward: run the Karma generator in a
  Boost.Coroutine \cpp{pull\_type} asymmetric coroutine. Such a coroutine
  receives, as a parameter, a synthesized \cpp{push\_type} coroutine endpoint
  representing its invoker.\\

  Boost.Coroutine provides an output iterator fa�ade for a \cpp{push\_type}
  coroutine endpoint. This output iterator is passed to the Karma generator.\\

  The input iterator corresponding to the \cpp{pull\_type} coroutine can then
  be dereferenced as desired. Every time a new item is requested, the Karma
  generator is resumed to produce it.
\end{description}

\uabschnitt{Stackful coroutines built on \cpp{callcc()}}
\begin{itemize}
\item Symmetric coroutines map very directly to \cpp{callcc()} functionality.
  Each coroutine is reified as an object. The object contains the
  \cpp{continuation} representing its suspended context -- or, if that
  coroutine is currently running, an invalid \cpp{continuation}.
\item A symmetric coroutine suspends by specifying a particular other
  coroutine object to resume. The implementation calls \resume on that other
  coroutine object's \cpp{continuation}.
\item An asymmetric coroutine ``knows'' its invoker: rather than explicitly
  resuming an arbitrary other coroutine, it \emph{yields,} implicitly resuming its
  invoker. (In just the same way, \cpp{return} implicitly resumes a function's
  caller.)
\item An asymmetric coroutine object could contain a reference to its
  invoker's coroutine object, permitting an anonymous \emph{yield} operation.
\end{itemize}

\abschnitt{Why not propose stackful coroutines instead?}

In fact -- we \emph{did!}\cite{N3708}\citecomma\cite{N3985} We were directed
to bring back a lower-level proposal. That lower-level proposal has evolved to
this present form.
\newpage
\abschnitt{Use case: many small stacks, one deep stack}
Proponents of \coawait frequently describe a particular execution
environment: a 32-bit Windows server process supporting millions of clients in
a transiently-stateful way, preserving some amount of state data across some
number of asynchronous operations.\\

It is pointed out that calls to opaque library, runtime and operating-system
functions may consume arbitrary amounts of stack space. The inability to
predict stack consumption in advance leads the Windows operating system to
allocate a 1MB stack for each kernel thread.\\

Of course, that stack memory is not committed until actually used. Still, it
does present a problem: in a 32-bit process, you quickly run out of address
space. You are constrained to no more than\\

$ \frac{2^{32} - (size\ of\ all\ code) - (size\ of\ all\ other\ data)}{stack\ size} $\\

stacks. Even if you set both \cpp{(size of all code)} and \cpp{(size of all other
data)} to zero -- impossible, in practice -- you can allocate no more than 4096
1MB stacks. 4096 is very much smaller than ``millions.''\\

On another operating system, one could use segmented stacks, but those are not
supported on Windows.\\

In a 64-bit process, the limitation would be actual memory consumption rather
than the address space. But it is suggested that many people still use 32-bit
server processes.\\

When considering userland threads, we might not be quite as conservative as
the Windows operating system. We might decide that we need far less stack
space than 1MB per userland thread. We might be so bold as to suggest 16KB
stacks. But that \emph{still} is constrained to a theoretical maximum of
262144 stacks -- and that's without code or any other data. This falls short
of ``millions'' by at least an order of magnitude.\\

We might be certain that our own code requires very little stack space -- even
less than 16KB. But does our code ever call library functions? runtime
functions? operating-system functions? How much stack space do \emph{they}
consume? This brings us back to the original unanswerable question.\\

Proponents of the \coawait facility explain that since each \coawait function
allocates a separate heap activation frame -- in effect, each has its own tiny
stack -- the thread's main stack is left largely untouched. Calls to opaque
library or runtime or operating-system functions that require arbitrary stack
space consume the thread's main stack, which is presumed to be Big Enough.\\

With \cc, we can use a similar trick. We can construct each new context with a
very small stack: just big enough for the function calls in our own code. We
can set aside one ``big enough'' stack as a common resource, shunting function
calls of unknown depth onto the shared ``big enough'' stack.\\

We assume that opaque functions of this kind will not themselves suspend.
Please note that the \coawait scenario requires the same assumption.\\

\cppf{deepstack}

(With \callcc replaced by \cpp{boost::context::callcc()} and \cont replaced by\\
\cpp{boost::context::continuation}, and with lavish \cpp{std::cout}
annotation, the program above compiles and runs as expected.)
