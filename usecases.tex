\abschnitt{Use cases}

\cc can be used to implement several higher-level abstractions.


\uabschnitt{Asymmetric coroutines: N3708}

is implemented in boost.coroutine2\cite{bcoroutine2} using \cc from
boost.context\cite{bcontext} as building block. Each \cpp{push\_type} and
\cpp{pull\_type} of a coroutine represents a continuation (e.g. a coroutine
consist of two continuations).
\cppf{coroutine}


\uabschnitt{Cooperative multi-tasking:}

boost.fiber\cite{bfiber} provides a framework for micro-/userland-threads
(fibers) scheduled cooperatively. The library implements fibers on behalf of \cc
(boost.context\cite{bcontext}). The API contains classes and functions to manage
and synchronize fibers similar to standard thread support library.\\
Each fiber represents a continuation.
\cppf{fiber}


\uabschnitt{Delimited continuations}

can be implemented via \cc. \cpp{reset} delimits the continuation and
\cpp{shift} reifies the continuation, e.g. the code that follows after
\cpp{shift} returns is passed as continuation to \cpp{shift}.\\

On entry \cpp{1} is written to \cpp{std::cout} at (a). The \cpp{shift} operator
at (b) wraps the continuation, that means the code at (d), and passes it as
argument \cpp{cont} at (b). \cpp{cont()} is called two times, thus (c) is
executed two times before (d) writes \cpp{2} to \cpp{std::cout}.
\cppf{delimited}


\uabschnitt{Backtracking}

or non-deterministic choice is the ability to specify certain
\emph{choice points} in the program used to for finding all (or some) solutions
to some computational problems. The algorithm \emph{backtracks} to a previous
\emph{choice point} as soon as it determines that current execution path does
not complete to a valid solution.\\
Backtracking could be implemented using two continuations, a success
continuation that proceeds with the algorithm and a failure continuation that
backtracks to an previous choice point\cite{Ferguson}.
