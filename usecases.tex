\abschnitt{Use cases}

\cc can be used to implement several higher-level abstractions.


\uabschnitt{Asymmetric coroutines: N3708}

is implemented in boost.coroutine2\cite{bcoroutine2} using \cc from
boost.context\cite{bcontext} as building block. Each \cpp{push\_type} and
\cpp{pull\_type} of a coroutine represents a continuation (e.g. a coroutine
consits of two continuations).
\cppf{coroutine}


\uabschnitt{Cooperative multi-tasking:}

boost.fiber\cite{bfiber} provides a framework for micro-/userland-threads
(fibers) scheduled cooperatively. The library implements fibers on behalf of \cc
(boost.context\cite{bcontext}). The API contains classes and functions to manage
and synchronize fibers similiarly to standard thread support library.\\
Each fiber represents a continuation.
\cppf{fiber}


\uabschnitt{Delimited continuations}

can be implemented via \cc. \cpp{reset} delimites the continuation and
\cpp{shift} reifies the continuation, e.g. the code that follows after
\cpp{shift} returns is passed as continuation to \cpp{shift}. On entry \cpp{1}
is written to \cpp{std::cout} at (a). The \cpp{shift} operator at (b) wraps the
continuation, that means the code at (d), and passes it as argument \cpp{cont}
at (b). \cpp{cont()} is called two times, thus (c) is executed two times before
(d) writes \cpp{2} to \cpp{std::cout}.
\cppf{delimited}


\uabschnitt{Backtracking}

is a general algorithm for finding all (or some) solutions to some computational problems, notably constraint satisfaction problems, that incrementally builds candidates to the solutions, and abandons each partial candidate c ("backtracks") as soon as it determines that c cannot possibly be completed to a valid solution.

\uabschnitt{Non-deterministic choice}
is the ability to specify certain points in the program (called "choice points") for various alternatives for program flow.
