\newpage
\abschnitt{Additional notes}

\uabschnitt{GPU}

\cc as proposed in this paper does not take GPUs into account. Later revisions
will address this issue, once we have an overarching concept of how the various
kinds of ``lightweight execution agents'' should interact.


\uabschnitt{SIMD}

does not interfere with \cc and can be used as usual (\cc triggers the context
switch at its invocation).\\
Of course, depending on the calling convention, some micro-processor registers,
dedicated to SIMD, might be preserved and restored too
\footnote{\emph{MS Windows x64} calling convention}.


\uabschnitt{TLS}

\cc is TLS-agnostic - best practice related to TLS applies to \cc too.\\
As shown in \nameref{mechanism}, \cc only preserves and restores
micro-processor registers at its invocation.


\uabschnitt{Migration between threads}

\cont can be migrated between threads, except for instances of
\cont representing \main or \entryfn of a thread (see \nameref{subsec:main}).
