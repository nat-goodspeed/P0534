\abschnitt{Why \cc is low-level}

\cc is a low-level implementation. \cont has the memory footprint of a pointer
because it aggregates a pointer to the stack of the managed continuation.\\
\newline
Functions expressed in assembler, consit of a \emph{function prologue} at the
beginning of a function and \emph{function epilogue} at the end of the function.
\cppf{function}

The \emph{prologue} (few lines of assembler) that prepares the stack and
registers for use inside the function. The \emph{epilogue} restores the stack
and registers\footnote{callee-saved registers as defined by the calling
convention} to the state they were before the function was called.\\
Between prologue and epilogue the computation and calls to sub-routines are
done.\\
\newline
Like ordinary functions \resume and \resumewith consist of prologue and
epilogue. The only difference to oridnary functions is, that \resume and
\resumewith additionally exchange the stack- and instruction pointer
\footnote{In fact on x86-architecture the instructions- pointer return-address
remains already on the stack. On some RISC-architectures the link-register
has to be preserved on the stack while the context is suspended and is loaded
into the instruction-pointer on resumption.} between proloque and epiloque.\\
The prologue and epilogue of \resume and \resumewith neither consume stack space
nor do they call sub-routines, only the stack-pointer is exchanged.\\
\newline
On modern architectures \callcc takes view CPU cycles (6-8 CPU cycles on Intel E5 2620).
\cppf{implementation}
