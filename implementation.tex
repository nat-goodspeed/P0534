\abschnitt{Why \cc is low-level}

\cc is a low-level implementation. \cont has the memory footprint of a pointer
because it contains only a pointer to the stack of the managed continuation.\\
\newline
The assembly code generated for a function consists of a \emph{function prologue} at the
beginning and \emph{function epilogue} at the end.
\cppf{function}

The \emph{prologue} (a few lines of assembler) prepares the stack and
registers for use inside the function. The \emph{epilogue} restores the stack
and registers\footnote{callee-saved registers as defined by the calling
convention} to the state they were before the function was called.\\
Between prologue and epilogue the computation and calls to sub-routines are
done.\\
\newline
Like ordinary functions, \resume and \resumewith contain prologue and
epilogue. The only difference from ordinary functions is that \resume and
\resumewith additionally exchange the stack- and instruction pointer
\footnote{In fact on x86-architecture the instruction-pointer return-address
is already stored on the stack. On some RISC-architectures the link-register
must be preserved on the stack while the context is suspended, and is loaded
into the instruction-pointer on resumption.} between prologue and epilogue.\\
The prologue and epilogue of \resume and \resumewith neither consume stack space
nor do they call sub-routines; only the stack-pointer is exchanged.
\cppf{implementation}
