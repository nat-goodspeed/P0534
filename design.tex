\newpage
\abschnitt{Design}\label{design}

\uabschnitt{\callcc as a factory function}

Every valid \cont instance is synthesized by the \callcc facility:
\begin{itemize}
\item as a parameter passed into the function called by \callcc or \resumewith
\item as the value returned by \callcc, \resume or \resumewith.
\end{itemize}

This is intentional for consistency with the \cc facility in other
languages.\cite{schemecallcc}\citecomma\cite{rubycallcc}

\uabschnitt{Footprint}

\cont contains only its \bfs{stack pointer} as a member variable. It should
typically be no larger than a pointer.

\uabschnitt{Passing data}\label{subsec:data}

Data may be passed to another context as additional arguments of \callcc and
\resume.\footnote{or returned from the function invoked by \resumewith; see
section \nameref{subsec:resumewith}}\\

With functions \dataavail and \getdata the code can test for data and, if desired,
retrieve the data.\\

Any additional data arguments passed to \callcc or \resume can be retrieved by
the created/resumed context using \getdata. The template arguments
for \getdata must agree with the types passed to \callcc or \resume:

\begin{itemize}
  \item If you call \cpp{std::callcc(fn)} or \resume with no additional arguments,
  then \dataavail will return \cpp{false}: \cpp{fn()} may not call \getdata at
  all.
  \item If you call \cpp{std::callcc(fn, single\_arg)}
  or \cpp{resume(single\_arg)}, then \dataavail will return \cpp{true}, and
  if \cpp{fn()} calls \getdata, its template argument must
  match \cpp{get\_data<decltype(single\_arg)>()}.
  \item If you call \cpp{std::callcc(fn, multiple\_args...)}
  or \cpp{resume(multiple\_args...)}, then \dataavail will return \cpp{true},
  and if \cpp{fn()} calls \getdata, its template arguments must match the
  types of \cpp{multiple\_args...}. Specifically, \cpp{get\_data<Args...>()}
  returns \cpp{std::tuple<Args...>}; that \cpp{std::tuple} must be compatible
  with \cpp{std::make\_tuple(multiple\_args...)}.
\end{itemize}

\cppf{passing_single}

The \cpp{callcc()} call at (0) enters the lambda and passes 1 into the
new context. The value is retrieved as \cpp{j}, as shown by (1). The expression
\cpp{caller.resume(j+1)} at (2) resumes the original context (represented
within the lambda by \cpp{caller}) and transfers back an integer of \cpp{j+1}.
The assignment at (3) sets \cpp{i} to \cpp{j+1}.\\

The call to \cpp{lambda.resume()} at (4) (note that no data is passed) resumes the
lambda, returning from the \cpp{caller.resume(j+1)} call at (2). Here, too,
we replace the \cont instance \cpp{caller} invalidated by the \resume call at (2)
with the new instance returned by that same \resume call.\\

Finally the lambda returns (the updated) \cpp{caller} at (5), terminating its
context.\\

Since the updated \cpp{caller} represents the continuation suspended by the call at
(4), control returns to \main.\\

However, since context \cpp{lambda} has now terminated, the updated \cpp{lambda} is
invalid. Its \opbool returns \cpp{false}; its \cpp{operator\!()} returns
\cpp{true}.\\

Multiple arguments can be transferred into another continuation too.
\cppf{passing_multiple}
\cpp{caller.get\_data<int,int>()} returns a \cpp{std::tuple<int,int>}
containing the values passed by the \callcc call at (0).


\uabschnitt{main() and thread functions}\label{subsec:main}

\main as well as the \entryfn of a thread can be represented by a continuation.
That \cont instance is synthesized when the running context suspends, and is
passed into the newly-resumed context.

Calling \anythread on a \cont instance representing \main, or
the \entryfn of a thread, will return \cpp{false}: such a \cont
must \emph{only} be resumed on its original thread.

\cppf{simple}
The \cpp{callcc()} call at (0) enters the lambda. The \cont\ \cpp{caller} at (1)
represents the execution context of \main. Returning \cpp{caller} at (2) resumes the
original context, switching back to \main.


\uabschnitt{\cc and std::thread}
Any context represented by a valid \cont instance is necessarily suspended.\\

It is valid to resume a \cont instance on any thread -- \emph{except} that
since the operating system is responsible for the stack allocated for \main,
as well as each \cpp{std::thread}, you must not attempt to resume a \cont
instance representing any such context on any thread other
than its own. \anythread tests for this.\\

If, for \cont\ \cpp{c}, \cpp{c.any\_thread()} returns \cpp{false}, it is
only valid to resume \cpp{c} on the thread on which it was initially
launched.


\uabschnitt{Termination}
There are a few different ways to terminate a given context without
terminating the whole process, or engaging undefined behavior.

\begin{itemize}
\item Return a valid continuation from the \entryfn.
\item Call \unwindcont with a valid continuation. This throws a \unwindex
instance that binds that continuation.
\item \bfs{[LEWG: Should we publish the \unwindex constructor that
accepts \cont? Then another supported way would be to construct and
throw \unwindex ``by hand,'' which is what \unwindcont does internally.]}
\item
Call \cpp{std::continuation::resume\_with(std::unwind\_context)}.
This is what \dtor does. Since \unwindcont accepts a \cont, and
since \resumewith synthesizes a \cont and passes it to the subject function,
this terminates the context referenced by the original \cont instance and
switches back to the caller.
\item Engage \dtor: switch to some other context, which will
receive a \cont instance representing the current context. Make that other
context destroy the received \cont instance.
\end{itemize}

When the \entryfn invoked by \callcc returns a valid \cont instance,
the running context is terminated. Control switches to the context
indicated by the returned \cont instance.\\

Returning an invalid \cont instance (\opbool returns \cpp{false}) invokes
undefined behavior.\\

If the \entryfn returns the same \cont instance it was originally
passed (or rather, the most recently updated \cont returned from \callcc or the
previous instance's \resume), control returns to the context that most
recently resumed the running context. However, the \entryfn may return (switch
to) any reachable valid \cont instance.\\

\emph{Calling} \contresume means: ``Please switch to the indicated context; I
am suspending; please resume me later.''\\

\emph{Returning} a particular \cont means: ``Please switch to the indicated
context; and by the way, I am done.''


\uabschnitt{Exceptions}\label{subsec:exceptions}
In general, if an uncaught exception escapes from
the \entryfn, \cpp{std::terminate} is called. There is one
exception: \unwindex. The \callcc facility internally uses \unwindex to clean
up the stack of a suspended context being destroyed. This exception must be
allowed to propagate out of an \entryfn.\\

A correct \entryfn\ \cpp{try / catch} block looks like this:
\cppf{rethrow_unwind}

Of course, if you do not expect the \entryfn or anything it calls to throw
exceptions, you need no \cpp{try / catch} block.\\

If a \resumewith function throws an exception that you expect to catch in the
context's \entryfn, it is good practice to bind into the exception object the
continuation passed to the \resumewith function so that
the \entryfn's \cpp{catch} clause can return that continuation.

\uabschnitt{Inject function into suspended context}\label{subsec:resumewith}
Sometimes it is useful to inject a new function (for instance, to throw an
exception) into a suspended context. For this purpose you may call
\cpp{resume\_with(Fn && fn,Args ... args)}, passing:

\begin{itemize}
  \item the function \cpp{fn()} to execute
  \item additional data \cpp{args} to be retrieved by \cpp{fn()}.
\end{itemize}

Let's say that the context represented by the \cont instance \cpp{ctx} has
suspended in a function \cpp{suspender()}. You intend to inject
function \cpp{fn()} into context \cpp{ctx} as if \cpp{suspender()} had directly
called \cpp{fn()} at the point where it suspended.\\

Like an \entryfn passed to \callcc, \cpp{fn()} must
accept \cpp{std::continuation&&}. However, instead of
returning \cont, \cpp{fn()} must return a type corresponding to the \cpp{args}
passed to the \resumewith call.

\begin{itemize}
  \item If you call \cpp{ctx.resume\_with(fn)} with no additional \cpp{args},
  the return type of \cpp{fn()} is irrelevant: its return value is discarded.
  \item If you call \cpp{ctx.resume\_with(fn, single\_arg)}, then \cpp{fn()}
  must return \cpp{decltype(single\_arg)}.
  \item If you call \cpp{ctx.resume\_with(fn, multiple\_args...)},
  then \cpp{fn()} must return a \cpp{std::tuple} with corresponding types:
  specifically, \cpp{decltype(std::make\_tuple(multiple\_args...))}.
\end{itemize}

Any \cpp{args...} passed to \resumewith can be retrieved within \cpp{fn()}
using \getdata. As with \callcc and \resume, the template arguments
for \getdata must agree with the types passed to \resumewith:

\begin{itemize}
  \item If you call \cpp{ctx.resume\_with(fn)} with no additional \cpp{args},
  then \dataavail will return \cpp{false}: \cpp{fn()} may not call \getdata at
  all.
  \item If you call \cpp{ctx.resume\_with(fn, single\_arg)}, then \dataavail
  will return \cpp{true}, and if \cpp{fn()} calls \getdata, its template
  argument must match \cpp{get\_data<decltype(single\_arg)>()}.
  \item If you call \cpp{ctx.resume\_with(fn, multiple\_args...)},
  then \dataavail will return \cpp{true}, and if \cpp{fn()} calls \getdata,
  its template arguments must match the types of \cpp{multiple\_args...}.
  Specifically, \cpp{get\_data<Args...>()} returns \cpp{std::tuple<Args...>};
  that \cpp{std::tuple} must be compatible
  with \cpp{std::make\_tuple(multiple\_args...)}.
\end{itemize}

The above describes retrieving passed values \emph{within} \cpp{fn()}. The
value returned by \cpp{fn()} becomes available to \cpp{suspender()}.
It is up to \cpp{fn()} whether to return the same value(s) it retrieved from
the \resumewith call.

\begin{itemize}
  \item If you call \cpp{ctx.resume\_with(fn)} with no additional \cpp{args},
  then \dataavail will return \cpp{false} even
  within \cpp{suspender()}: \cpp{suspender()} may not call \getdata at all.
  \item If you call \cpp{ctx.resume\_with(fn, args...)} -- with one or
  more \cpp{args...} -- then \dataavail will return \cpp{true}
  within \cpp{suspender()}. \getdata within \cpp{suspender()} will retrieve
  the value returned by \cpp{fn()}.
\end{itemize}

Note that if you pass one or more \cpp{args...} to \resumewith,
those \cpp{args...} \emph{must} be compatible with the return type
of \cpp{fn()}. However, it is permissible to call \cpp{resume\_with(fn)}
sometimes (\dataavail returns \cpp{false}) but pass compatible data in
other \resumewith calls.\\

Suppose that code running on the program's main context calls \cpp{std::callcc(f)},
thereby entering \cpp{f()}. This is the point at which \cpp{mc} is synthesized
and passed into \cpp{f()}, as illustrated below.\\

Suppose further that after doing some work, \cpp{f()} calls \cpp{mc.resume()},
thereby switching back to the main context. \cpp{f()} remains suspended
in the call to \cpp{mc.resume()}.\\

At this point the main context calls \cpp{f\_ct.resume\_with(g)}
where \cpp{g()} is declared as \cpp{int g(continuation &&)}. \cpp{g()} is
entered in the context of \cpp{f()}. It is as if \cpp{f()}'s call
to \cpp{mc.resume()} directly called \cpp{g()}.\\

Function \cpp{g()} has almost the same range of possibilities as any function
called on \cpp{f()}'s context. Its special invocation matters when control
leaves it in either of two ways:

\begin{enumerate}
  \item If \cpp{g()} throws an exception, that exception unwinds all previous
        stack entries in that context (such as \cpp{f()}'s) as well, back to a
        matching \cpp{catch} clause.\footnote{As stated in \nameref{subsec:exceptions},
        if there is no matching \cpp{catch} clause in that context,
        \cpp{std::terminate()} is called.}
  \item If \cpp{g()} returns, its return value provides data for
        \cpp{f()}'s suspended \cpp{mc.resume()} call.
\end{enumerate}

\cppf{ontop}

Control passes from (0) to (1) to (2), and so on.\\

The \cpp{f\_ct.resume\_with(g, data+1)} call at (8) passes control
to \cpp{g()} on the context of \cpp{f()}.\\

The \cpp{return} statement at (9) causes the \resume call at (6) to return,
executing the assignment at (10). The \cpp{int} returned by \cpp{g()} is
accessed at (11).\\

Finally, \cpp{f()} returns its own \cpp{mc} variable, switching back to the main
context.\\

There is one restriction on a function \cpp{fn()} passed to \resumewith:
neither \cpp{fn()}, nor any function it calls, may perform a context switch
back to the context that called \resumewith -- whether directly, or indirectly
via other contexts. \cpp{fn()} \emph{must} return (or throw an exception)
before \resumewith returns.


\uabschnitt{\callcc immediately enters new context}

\callcc creates a new context and immediately calls its passed \entryfn on
that new context.

This is intentional for consistency with the \cc facility in other
languages.\cite{schemecallcc}\citecomma\cite{rubycallcc}

Moreover, this behavior prevents a problematic usage. Suppose we had
a \cpp{callcc\_deferred()} which would create a new context but immediately
return to its caller. The newly-created context would first be entered by
calling \resume on the returned \cont instance.
\cppf{good_resume}
But now consider this scenario:
\cppf{bad_resume_with}
What should happen here?

\resumewith is supposed to call \cpp{injected\_function()} as if it had been
directly called by \cpp{entry\_function()} -- rather, by the context-switch
operation most recently executed by \cpp{entry\_function()}. But
since \cpp{entry\_function()} has never yet been entered, it hasn't executed
any context-switch operation. Indeed, it does not yet have a stack frame.

Should \cpp{injected\_function()} \emph{become} the \entryfn
for \cpp{newcontext}, displacing \cpp{entry\_function()} entirely?

That would encounter signature problems. The \entryfn invoked
by \callcc \emph{must} return a \cont. Yet the \cpp{fn} passed to \resumewith
returns -- not a \cont -- but arbitrary data to be retrieved by the suspended
context!

To quickly resume the caller's context rather than prioritizing the new
context, the \entryfn passed to \callcc can immediately context-switch back
to its caller by calling \resume on its passed-in \cont:
\cppf{quickly_resume}

A more generic wrapper for that behavior could look something like this:
\cppf{callcc_deferred}
Note that since \cpp{suspend\_immediately()} \emph{has} been entered, it
is perfectly valid for the caller of \cpp{callcc\_deferred()} to
call \resumewith on the returned \cont.

\uabschnitt{Stack destruction}\label{subsec:destruction}
On construction of a context with \callcc a stack is allocated. If the
\entryfn returns, the stack will be destroyed. If the function has not
yet returned and the \nameref{subpara:destructor} of the \cont instance
representing that context is called, the stack will be unwound and destroyed.\\

For this purpose member-function \resumewith is called with \unwindcont
as argument. The execution context will be temporarily
resumed and \unwindcont is invoked. Function \unwindcont throws exception
\unwindex.\footnote{\unwindex binds an instance of \cont that represents the
continuation that called \resumewith.} The exception
is caught by the first frame on the stack: the one created by
\callcc. Control is switched back to the context that called
\dtor and the stack gets deallocated.\\

The StackAllocator's deallocate operation is called on the context that
invoked \dtor.\\

The stack on which \cpp{main()} is executed, as well as the stack implicitly
created by \cpp{std::thread}'s constructor, is allocated by the operating
system. Such stacks are recognized by \cont (\anythread
returns \cpp{false}), and are not deallocated by its destructor.


\uabschnitt{Stack allocators}\label{subsec:stackalloc}
are used to create stacks.\\

Stack allocators might implement arbitrary stack strategies. For instance, a
stack allocator might append a guard page at the end of the stack, or cache
stacks for reuse, or create stacks that grow on demand.\\

Because stack allocators are provided by the implementation, and are only used
as parameters of \callcc, the StackAllocator concept is an implementation detail,
used only by the internal mechanisms of the \cc implementation. Different
implementations might use different StackAllocator concepts.\\

However, when an implementation provides a stack allocator matching one of
the descriptions below, it should use the specified name.\\

Possible types of stack allocators:
\begin{itemize}
    \item \cpp{protected\_fixedsize}: The constructor accepts a \cpp{size\_t}
        parameter. This stack allocator constructs a contiguous stack of
        specified size, appending a guard page at the end to protect against
        overflow. If the guard page is accessed (read or write operation), a
        segmentation fault/access violation is generated by the operating
        system.
    \item \cpp{fixedsize}: The constructor accepts a \cpp{size\_t} parameter.
        This stack allocator constructs a contiguous stack of specified size.
        In contrast to \cpp{protected\_fixedsize}, it does not append a guard
        page. The memory is simply managed by \cpp{std::malloc()}
        and \cpp{std::free()}, avoiding kernel involvement.
    \item \cpp{segmented}: The constructor accepts a \cpp{size\_t} parameter.
        This stack allocator creates a segmented stack\cite{gccsplit} with the
        specified initial size, which grows on demand.\footnote{An
        implementation of the \cpp{segmented} StackAllocator necessarily
        interacts with the C++ runtime. For instance, with gcc, the
        Boost.Context\cite{bcontext} library invokes
        the \cpp{\_\_splitstack\_makecontext()}
        and \cpp{\_\_splitstack\_releasecontext()} intrinsic
        functions.\cite{splitalloc}\citecomma\cite{bctxseg}}
\end{itemize}

It is expected that the StackAllocator's allocation operation will run in the
context of the \callcc call (before control is passed to the new context), and
that the StackAllocator's deallocation operation will run in the context of
the \dtor call (after control returns from the destroyed
context). No special constraints need apply to either operation.
