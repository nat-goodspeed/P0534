\abschnitt{Design}

\call expects a \entryfn (callable that is executed in the new continuation)
with signature \cpp{continuation(continuation &&,Arg ...)}.\\

On return the \entryfn has to specify an \cont to which the execution control is
transferred.\\

If an instance with valid state goes out of scope and the \entryfn has not yet
returned, the stack is traversed  and continuation's stack is deallocated.\\


\uabschnitt{Passing data}\label{subsec:data}

\uabschnitt{Toplevel functions: main() and thread functions}\label{subsec:main}

\main as well as the \entryfn of a thread can be represented by an continuation.
That \cont instance is synthesized when the running context suspends, and is
passed into the newly-resumed continuation.
\cppf{simple}
The \cpp{ctx1()} call at (a) enters the lambda in context ctx1.\\
The \cont\ \cpp{ctx2} at (b) represents the execution context of \main.\\
Returning \cpp{ctx2} at (c) resumes the original context (switch back to
\main).

\uabschnitt{\callcc and std::thread}
Any contiunation represented by a valid \cont instance is necessarily suspended.\\
It is valid to resume a \cont instance on any thread -- \emph{except} that you
must not attempt to resume a \cont instance representing \main, or
the \emph{entry-function} of some other \cpp{std::thread}, on any thread other
than its own.
\cont provides a method to test for this.
If \cpp{std::contiunation<>::any\_thread()} returns \cpp{false}, it is
only valid to resume that
\cont instance on the thread on which it was initially launched.


\uabschnitt{Termination}
When that toplevel callable returns a \cont instance, the continuation is
terminated. Control switches to the continuation indicated by the returned \cont
instance.\\
Returning an invalid \cont instance (\opbool returns \cpp{false}) invokes
undefined behavior.\\
If the toplevel callable returns the same \cont instance it was originally
passed (or rather, the most recently updated instance returned from the
previous instance's \call or \resume), control returns to the context that most
recently resumed the running callable. However, the callable may return (switch
to) any reachable valid \cont instance with the correct type signature.


\uabschnitt{Exceptions}\label{subsec:exceptions}
If an uncaught exception escapes from the \entryfn, \cpp{std::terminate} is called.


\uabschnitt{Invoke function on top of a continuation}
Sometimes it is useful to invoke a new function (for instance, to trigger
unwinding the stack) on top of a continuation. For this purpose you may
pass to \cpp{resume(continuation &&,invoke\_ontop\_arg\_t,Fn &&,Args ...)}:

\begin{itemize}
  \item the special argument \cpp{invoke\_ontop\_arg}
  \item the function to execute
  \item any additional arguments.
\end{itemize}

The function passed in this case must accept as parameters the \cont.
It must return the same set of arguments as the \resume specialization.
\footnote{But in the case of passing no arguments, the return type is simply
\cpp{void}.}\\

For purposes of discussion, consider two \cont instances: \cpp{mc} and \cpp{fc}.
Suppose that code running on the program's main context instantiates \cpp{fc}
with function \cpp{f()} and calls \cpp{calcc( fc)}, thereby entering \cpp{f()}.
This is the point at which \cpp{mc} is synthesized and passed into \cpp{f()}.

Suppose further that after doing some work, \cpp{f()} calls \cpp{resume(mc)},
thereby switching context back to the main context. \cpp{f()} remains
suspended in the call to \cpp{resume(mc)}.

At this point the main context calls \cpp{iresume(fc,invoke\_ontop\_arg, g);}
where \cpp{g()} is declared as \cpp{void g(continuation &);} \cpp{g()} is
entered in the context of \cpp{f()}. It is as if \cpp{f()}'s call to
\cpp{resume(mc)} directly called \cpp{g()}.

Function \cpp{g()} has the same range of possibilities as any function called
on \cpp{f()}'s context. Its special invocation only matters when control
leaves it in either of two ways:

\begin{enumerate}
  \item If \cpp{g()} throws an exception, that exception unwinds all previous
        stack entries in that context (such as \cpp{f()}'s) as well, back to a
        matching \cpp{catch} clause.\footnote{As stated in
        \nameref{subsec:exceptions}, if there is no matching \cpp{catch} clause
        in that context, \cpp{std::terminate()} is called.}
  \item If \cpp{g()} returns, its return value becomes the value returned by
        \cpp{f()}'s suspended \cpp{resume(mc)} call. This is why \cpp{g()}'s
        return type must be the same as that of \resume, rather than that of an
        ordinary toplevel context function.
\end{enumerate}

Consider the following example:

\cppf{ontop}

Control passes from (a) to (b) to (c), and so on.

The \cpp{resume(c,invoke\_ontop\_arg, f2, data+1)} call at (l) passes control
to \cpp{f2()} on the context originally created for \cpp{f1()}.

The \cpp{return} statement at (n) causes the \op call at (i) to return,
executing the assignment at (o). The \cpp{int} returned by \cpp{f2()}
is directly returned to that assignment at (o).

So in this example, the call at (l) synthesizes a \cont instance representing
the main context and updates \cpp{mctx} internally. \cpp{f2()} returns its
return value \emph{-1}. Finally, \cpp{f1()} returns its own \cpp{ctx}
variable, switching back to the main context.


\uabschnitt{Stack destruction}\label{subsec:destruction}
On construction of \cpp{execution\_context} a stack is allocated. If the
toplevel context-function returns, the stack will be destroyed. If the
context-function has not yet returned and the \nameref{subpara:destructor} of
a valid \cpp{execution\_context} instance (\opbool
returns \cpp{true}) is called, the stack will be unwound and
destroyed.\footnote{An implementation is free to unwind the stack without
throwing an exception. However, if an exception is thrown, it should be
named \cpp{std::execution\_context\_unwind}. Portable consumer
code \emph{must} permit \cpp{std::execution\_context\_unwind} exceptions to
propagate, even if all other exceptions are caught with \cpp{catch(...)}.}

The stack on which \cpp{main()} is executed, as well as the stack implicitly
created by \cpp{std::thread}'s constructor, is allocated by the operating
system. Such stacks are recognized by \cont, and are not deallocated by its
destructor.

\uabschnitt{Stack allocators}\label{subsec:stackalloc}
are used to create stacks.\footnote{This concept, along with the \cont
constructor accepting \cpp{std::allocator\_arg\_t}, is an optional part of the
proposal. It might be that implementations can reliably infer the optimal
stack representation.} Stack allocators might implement arbitrary stack
strategies. For instance, a stack allocator might append a guard page at the
end of the stack, or cache stacks for reuse, or create stacks that grow on
demand.\\
Because stack allocators are provided by the implementation, and are only used
as parameters of\\
\cont's constructor, the StackAllocator concept is an implementation detail,
used only by the internal mechanisms of the \cont implementation. Different
implementations might use different StackAllocator concepts.\\
However, when an implementation provides a stack allocator matching one of
the descriptions below, it should use the specified name.\\
Possible types of stack allocators:
\begin{itemize}
    \item \cpp{protected\_fixedsize}: The constructor accepts a \cpp{size\_t}
        parameter. This stack allocator constructs a contiguous stack of
        specified size, appending a guard page at the end to protect against
        overflow. If the guard page is accessed (read or write operation), a
        segmentation fault/access violation is generated by the operating
        system.
    \item \cpp{fixedsize}: The constructor accepts a \cpp{size\_t} parameter.
        This stack allocator constructs a contiguous stack of specified size.
        In contrast to \cpp{protected\_fixedsize}, it does not append a guard
        page. The memory is simply managed by \cpp{std::malloc()}
        and \cpp{std::free()}, avoiding kernel involvement.
    \item \cpp{segmented}: The constructor accepts a \cpp{size\_t} parameter.
        This stack allocator creates a segmented stack with the specified
        initial size, which grows on demand.
\end{itemize}


\uabschnitt{std::continuation}
declaration of class \cont
\cppf{cont}
\paragraph*{member functions}
\subparagraph*{(constructor)}
constructs new continuation\\

\begin{tabular}{ l l }
    \midrule

    \cpp{continuation() noexcept} & (1)\\

    \midrule

    \cpp{continuation(continuation&& other)} & (2)\\

    \midrule

    \cpp{continuation(const continuation& other)=delete} & (3)\\

    \midrule
\end{tabular}

\begin{description}
    \item[1)] This constructor instantiates an invalid \cont. Its \opbool
              returns \cpp{false}; its \cpp{operator\!()} returns \cpp{true}.
    \item[2)] moves underlying state to new \cont
    \item[3)] copy constructor deleted
\end{description}

{\bfseries Notes}
\newline
When a \cont is constructed using either of the constructors accepting
\cpp{fn}, control is \emph{not} immediately passed to \cpp{fn}. Resuming
(entering) \cpp{fn} is performed by calling \cpp{operator()()} on the new
\cont instance.\\

\subparagraph*{(destructor)}\label{subpara:destructor}
destroys an continuation\\

\begin{tabular}{ l l }
    \midrule

    \cpp{\~continuation()} & (1)\\

    \midrule
\end{tabular}

\begin{description}
    \item[1)] destroys a \cont instance. If this instance represents a
              context of execution (\opbool returns \cpp{true}),
              then the context of execution is destroyed too. Specifically,
              the stack is unwound. As noted in \nameref{subsec:destruction},
              an implementation is free to unwind the stack either by
              throwing \cpp{std::execution\_context\_unwind} or by intrinsics
              not requiring \cpp{throw}.\\
\end{description}

\subparagraph*{operator=}
moves the continuation object\\

\begin{tabular}{ l l }
    \midrule

    \cpp{continuation& operator=(continuation&& other)} & (1)\\

    \midrule

    \cpp{continuation& operator=(const continuation& other)=delete} & (2)\\

    \midrule
\end{tabular}

\begin{description}
    \item[1)] assigns the state of \cpp{other} to \cpp{*this} using move semantics
    \item[2)] copy assignment operator deleted
\end{description}

{\bfseries Parameters}
\begin{description}
    \item[other]   another execution context to assign to this object\\
\end{description}

{\bfseries Return value}
\begin{description}
    \item[*this]
\end{description}

\subparagraph*{operator bool}
test whether contiunation is valid\\

\begin{tabular}{ l l }
    \midrule

    \cpp{explicit operator bool() const noexcept} & (1)\\

    \midrule
\end{tabular}

\begin{description}
    \item[1)] returns \cpp{true} if \cpp{*this} represents a context of
              execution, \cpp{false} otherwise.
\end{description}

{\bfseries Notes}
\newline
A \cont instance might not represent a context of execution for any of a
number of reasons.
\begin{itemize}
    \item It might have been default-constructed.
    \item It might have been assigned to another instance, or passed into a
          function.\\
          \cont instances are move-only.
    \item It might already have been resumed (\resume called). Calling \resume
          invalidates the instance.
    \item The toplevel context-function might have voluntarily terminated the
          context by returning.
\end{itemize}
The essential points:
\begin{itemize}
    \item Regardless of the number of \cont declarations, exactly one\\
          \cont instance represents each suspended context.
    \item No \cont instance represents the currently-running context.
\end{itemize}

\subparagraph*{operator!}
test whether contiunation is invalid\\

\begin{tabular}{ l l }
    \midrule

    \cpp{bool operator\!() const noexcept} & (1)\\

    \midrule
\end{tabular}

\begin{description}
    \item[1)] returns \cpp{false} if \cpp{*this} represents a context of
              execution, \cpp{true} otherwise.
\end{description}

{\bfseries Notes}
\newline
See {\bfseries Notes} for \opbool.

\subparagraph*{any\_thread}
test whether suspended contiunation may be resumed on a different thread\\

\begin{tabular}{ l l }
    \midrule

    \cpp{bool any\_thread() const noexcept} & (1)\\

    \midrule
\end{tabular}

\begin{description}
    \item[1)] returns \cpp{false} if \cpp{*this} must be resumed on the same
              thread on which it previously ran, \cpp{true} otherwise
\end{description}

{\bfseries Notes}
\newline
As stated in \nameref{subsec:main}, a \cont instance can represent the initial
context on which the operating system runs \main, or the context created by
the operating system for a new \cpp{std::thread}.

It is not permitted to attempt to resume such a \cont instance on any thread
other than its original thread. \cpp{any\_thread()} allows consumer code to
distinguish this case.

\subparagraph*{(comparisons)}
establish an arbitrary total ordering for \cont instances\\

\begin{tabular}{ l l }
    \midrule

    \cpp{bool operator==(const continuation& other) const noexcept} & (1)\\

    \midrule

    \cpp{bool operator\!=(const continuation& other) const noexcept} & (1)\\

    \midrule

    \cpp{bool operator<(const continuation& other) const noexcept} & (2)\\

    \midrule

    \cpp{bool operator>(const continuation& other) const noexcept} & (2)\\

    \midrule

    \cpp{bool operator<=(const continuation& other) const noexcept} & (2)\\

    \midrule

    \cpp{bool operator>=(const continuation& other) const noexcept} & (2)\\

    \midrule
\end{tabular}

\begin{description}
    \item[1)] Every invalid \cont instance compares equal to every other
              invalid instance. But because the running context is never
              represented by a valid \cont instance, and because every
              suspended context is represented by exactly one valid
              instance, \emph{no valid instance can ever compare equal to any
              other valid instance.}
    \item[2)] These comparisons establish an arbitrary total ordering of \cont
              instances, for example to store in ordered containers. (However,
              key lookup is meaningless, since you cannot construct a search
              key that would compare equal to any entry.) There is no
              significance to the relative order of two instances.
\end{description}

\subparagraph*{swap}
swaps two \cont instances\\

\begin{tabular}{ l l }
    \midrule

    \cpp{void swap(continuation& other) noexcept} & (1)\\

    \midrule
\end{tabular}

\begin{description}
    \item[1)] Exchanges the state of \cpp{*this} with \cpp{other}.\\
\end{description}


\uabschnitt{std::callcc()}
declaration of free function \call
\cppf{callcc}

\subparagraph*{std::callcc()}
create and enter new continuation\\

\begin{tabular}{ l l }
    \midrule

    \cpp{template< typename Fn, typename ...Arg >}\\
    \cpp{continuation callcc( Fn &&, Arg ...)} & (1)\\

    \midrule

    \cpp{template< typename StackAlloc, typename Fn, typename ...Arg >}\\
    \cpp{continuation callcc( std::allocator_arg_t, StackAlloc, Fn &&, Arg ...)} & (2)\\

    \midrule
\end{tabular}

\begin{description}
    \item[1)] suspends the active context, resumes the execution context
    \item[2)] specialization of (1) for \cpp{execution\_context<void>}
    \item[3)] suspends the active context, resumes the execution context but
        executes \cpp{fn(args ...)} in the resumed context (on top of the
        last stack frame)
    \item[4)] specialization of (3) for \cpp{execution\_context<void>}
\end{description}

{\bfseries Parameters}
\begin{description}
    \item[...args] If the toplevel context-function represented
                   by \cpp{*this} has not yet been entered, the arguments
                   you pass are passed to the context-function as its
                   parameters, following the \cont first parameter. \\
                   If the context represented by \cpp{*this} suspended by
                   calling \op, the arguments you pass
                   are constructed into a \contargstup and returned by
                   that suspended \op call. \\
                   See section \nameref{subsec:ectxdata}.\\
    \item[fn]      The \cpp{fn} passed to (3) must accept \cpp{Args...}. It
                   must return \cpp{std::tuple<Args...>} or simply \cpp{Arg} in
                   the case of a single argument.\\
                   The \cpp{fn} passed to (4) must accept and return \cpp{void}.\\
\end{description}

{\bfseries Return value}
\begin{description}
    \item[void]     When called on a \contvoid instance, \op returns
                    a different \contvoid instance. This new instance
                    represents the context that suspended in order to resume
                    the current context. That may or may not be the same
                    context that was previously represented by \cpp{*this},
                    depending on other context switches executed in the
                    meantime.
    \item[tuple]    When called on a \contargs instance, \op returns a\\
                    \contargstup containing a different\\
                    \contargs instance. This new instance represents the
                    context that suspended in order to resume the current
                    context, as above.\\
                    If the context represented by the new \contargs instance
                    suspended by calling \op, the arguments passed to \op are
                    used to populate the rest of the members of the
                    returned \cpp{std::tuple}.\\
                    See section \nameref{subsec:ectxdata}.\\
                    If the context represented by the new \contargs instance
                    voluntarily terminated by returning from its toplevel
                    context-function, the rest of the members of the
                    returned \cpp{std::tuple} are indeterminate.\\
\end{description}

{\bfseries Exceptions}
\begin{description}
    \item[1)] calls \cpp{std::terminate} if an exception escapes toplevel \cpp{fn}\\
\end{description}

{\bfseries Preconditions}
\begin{description}
    \item[1)] \cpp{*this} represents a context of execution (\opbool
              returns \cpp{true})
    \item[2)] \cpp{any\_thread()} returns \cpp{true}, or the running thread is
              the same thread on which \cpp{*this} ran previously.
\end{description}

{\bfseries Postcondition}
\begin{description}
    \item[1)] \cpp{*this} is invalidated (\opbool returns \cpp{false})
\end{description}

{\bfseries Notes}
\newline
The \emph{prologue} preserves the execution context of the calling context as
well as stack parts like \emph{parameter list} and \emph{return
address}.\footnote{required only by some x86 ABIs}
Those data are restored by the \emph{epilogue} if the calling context is
resumed.
\newline
A suspended \cpp{execution\_context} can be destroyed. Its resources will be
cleaned up at that time.
\newline
The returned \cpp{execution\_context} indicates whether the suspended context
has terminated (returned from toplevel context-function) via \opbool. If the
returned \cpp{execution\_context} has terminated, no data are transferred in
the returned tuple.
\newline
Because \op invalidates the instance on which it is called, \emph{no
valid \cont instance ever represents the currently-running context.}
\newline
When calling \op, it is conventional to replace the newly-invalidated 
instance -- the instance on which \op was called -- with the new instance
returned by that \op call. This helps to avoid inadvertent calls to \op on the
old, invalidated instance.
\newline
For any \cont specialization other than \contvoid,\\
when \op returns, it is important to test the returned \contargs instance using
\opbool or \cpp{operator\!()} before referencing any of the \cpp{args...} in the
returned \contargstup. If that context voluntarily terminated by returning
from the toplevel context-function, only the \contargs member of
the \cpp{std::tuple} will be populated. The rest of the members will have
indeterminate values.


\subparagraph*{std::resume()}
resume a continuation\\

\begin{tabular}{ l l }
    \midrule

    \cpp{template< typename ...Arg >}\\
    \cpp{continuation resume( continuation &&, Arg ...)} & (1)\\

    \midrule

    \cpp{template< typename Fn, typename ...Arg >}\\
    \cpp{continuation resume( continuation &&, invoke\_ontop\_arg_t, Fn &&, Arg ...)} & (2)\\

    \midrule
\end{tabular}

\begin{description}
    \item[1)] suspends the active context, resumes the execution context
    \item[2)] specialization of (1) for \cpp{execution\_context<void>}
    \item[3)] suspends the active context, resumes the execution context but
        executes \cpp{fn(args ...)} in the resumed context (on top of the
        last stack frame)
    \item[4)] specialization of (3) for \cpp{execution\_context<void>}
\end{description}

{\bfseries Parameters}
\begin{description}
    \item[...args] If the toplevel context-function represented
                   by \cpp{*this} has not yet been entered, the arguments
                   you pass are passed to the context-function as its
                   parameters, following the \cont first parameter. \\
                   If the context represented by \cpp{*this} suspended by
                   calling \op, the arguments you pass
                   are constructed into a \contargstup and returned by
                   that suspended \op call. \\
                   See section \nameref{subsec:ectxdata}.\\
    \item[fn]      The \cpp{fn} passed to (3) must accept \cpp{Args...}. It
                   must return \cpp{std::tuple<Args...>} or simply \cpp{Arg} in
                   the case of a single argument.\\
                   The \cpp{fn} passed to (4) must accept and return \cpp{void}.\\
\end{description}

{\bfseries Return value}
\begin{description}
    \item[void]     When called on a \contvoid instance, \op returns
                    a different \contvoid instance. This new instance
                    represents the context that suspended in order to resume
                    the current context. That may or may not be the same
                    context that was previously represented by \cpp{*this},
                    depending on other context switches executed in the
                    meantime.
    \item[tuple]    When called on a \contargs instance, \op returns a\\
                    \contargstup containing a different\\
                    \contargs instance. This new instance represents the
                    context that suspended in order to resume the current
                    context, as above.\\
                    If the context represented by the new \contargs instance
                    suspended by calling \op, the arguments passed to \op are
                    used to populate the rest of the members of the
                    returned \cpp{std::tuple}.\\
                    See section \nameref{subsec:ectxdata}.\\
                    If the context represented by the new \contargs instance
                    voluntarily terminated by returning from its toplevel
                    context-function, the rest of the members of the
                    returned \cpp{std::tuple} are indeterminate.\\
\end{description}

{\bfseries Exceptions}
\begin{description}
    \item[1)] calls \cpp{std::terminate} if an exception escapes toplevel \cpp{fn}\\
\end{description}

{\bfseries Preconditions}
\begin{description}
    \item[1)] \cpp{*this} represents a context of execution (\opbool
              returns \cpp{true})
    \item[2)] \cpp{any\_thread()} returns \cpp{true}, or the running thread is
              the same thread on which \cpp{*this} ran previously.
\end{description}

{\bfseries Postcondition}
\begin{description}
    \item[1)] \cpp{*this} is invalidated (\opbool returns \cpp{false})
\end{description}

{\bfseries Notes}
\newline
The \emph{prologue} preserves the execution context of the calling context as
well as stack parts like \emph{parameter list} and \emph{return
address}.\footnote{required only by some x86 ABIs}
Those data are restored by the \emph{epilogue} if the calling context is
resumed.
\newline
A suspended \cpp{execution\_context} can be destroyed. Its resources will be
cleaned up at that time.
\newline
The returned \cpp{execution\_context} indicates whether the suspended context
has terminated (returned from toplevel context-function) via \opbool. If the
returned \cpp{execution\_context} has terminated, no data are transferred in
the returned tuple.
\newline
Because \op invalidates the instance on which it is called, \emph{no
valid \cont instance ever represents the currently-running context.}
\newline
When calling \op, it is conventional to replace the newly-invalidated 
instance -- the instance on which \op was called -- with the new instance
returned by that \op call. This helps to avoid inadvertent calls to \op on the
old, invalidated instance.
\newline
For any \cont specialization other than \contvoid,\\
when \op returns, it is important to test the returned \contargs instance using
\opbool or \cpp{operator\!()} before referencing any of the \cpp{args...} in the
returned \contargstup. If that context voluntarily terminated by returning
from the toplevel context-function, only the \contargs member of
the \cpp{std::tuple} will be populated. The rest of the members will have
indeterminate values.


\subparagraph*{std::transfer\_data<>()}
transfer of data\\
