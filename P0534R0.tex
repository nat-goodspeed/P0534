%//////////////////////////////////////////////////////////////////////////////

\documentclass[paper=A4,pagesize,DIV=15]{scrartcl}

\usepackage[T1]{fontenc}
\usepackage[latin1]{inputenc}
\usepackage[british]{babel}

%\usepackage{fixltx2e}
\usepackage{ellipsis}
\usepackage{ragged2e}
\usepackage[final]{microtype}

\usepackage{palatino}

\usepackage{overcite}
\usepackage{booktabs}
\usepackage{fancyhdr}
\usepackage{listings}
\usepackage{perpage}
\usepackage{rotating}
\usepackage{svg}
\usepackage{tikz}
\usetikzlibrary{arrows,automata}
\usepackage{xcolor}
\usepackage{xspace}
\usepackage[colorlinks=true,
            urlcolor=blue,
            pdftex,
            pdfsubject  = {},
            pdfauthor   = {Oliver Kowalke},
            pdfkeywords = {C++,callcc,call/cc,context,continuation,coroutine,execution,fiber,P0099,P0534},
            pdftitle    = {call/cc: A low-level API for stackful context switching}]{hyperref}

%//////////////////////////////////////////////////////////////////////////////

\setlength{\parindent}{0pt} 
\renewcommand\sfdefault{phv}

\makeatletter
    \renewcommand*\l@subsection{\@dottedtocline{2}{0em}{2.3em}}
    \renewcommand*\l@subsection{\@dottedtocline{3}{0em}{3.2em}}
    \renewcommand{\tableofcontents}{\@starttoc{toc}}
\makeatother

\MakePerPage{footnote}
\renewcommand*{\thefootnote}{\fnsymbol{footnote}}

\newcommand{\pdfimg}[1]{\pdfximage{pics/#1}\pdfrefximage\pdflastximage}
\newcommand{\img}[1]{\mbox{\pdfimg{#1}}}
\newcommand{\imgc}[1]{\begin{center}\img{#1}\end{center}}
\newcommand{\graph}[1]{\input{graphs/#1}}
\newcommand{\graphc}[1]{\begin{center}\graph{#1}\end{center}}

\lstdefinelanguage
   [arm]{Assembler}                     % add a "arm" dialect of Assembler
   {morekeywords={mov,pop,push,subi,stw,mfcr,mflr,mr,lwz,mtcr,mtlr,mtctr,addi,bctr,str}}    % with these extra keywords:
\lstset{
        language=[arm]Assembler,
        numbers=none,
        numberstyle=\tiny,
        numberblanklines=false,
        stepnumber=1,
        numbersep=10pt
}

\newcommand{\cpp}[1]{{\lstinline[
		basicstyle=\ttfamily\small\color{black},
        breakatwhitespace=true,
        breaklines=true,
        captionpos=b,
        commentstyle=\ttfamily\color{gray},
        keywordstyle=\ttfamily\color{blue},
        language={C++},
        morekeywords={co\_await,from,noexcept,resumable,co\_yield},
        showspaces=false,
        showstringspaces=false,
        showtabs=false,
        stringstyle=\ttfamily\color{red}
] !#1!}\xspace}
\newcommand{\cppf}[1]{\lstinputlisting[
		basicstyle=\ttfamily\small\color{black},
        breakatwhitespace=true,
        breaklines=true,
        captionpos=b,
        commentstyle=\ttfamily\color{gray},
        keywordstyle=\ttfamily\color{blue},
        language={C++},
        morekeywords={co\_await,from,noexcept,resumable,co\_yield},
        showspaces=false,
        showstringspaces=false,
        showtabs=false,
        stringstyle=\ttfamily\color{red}
] {code/#1.cpp}}


\newcommand{\asm}[1]{
    \lstinline[
        basicstyle=\ttfamily\color{black},
        keywordstyle=\color{blue},
        commentstyle=\color{red},
        stringstyle=\color{green}
    ] {#1}
}
\newcommand{\asmf}[1]{
    \lstinputlisting[
%        numbers=left,
        basicstyle=\ttfamily\color{black},
        keywordstyle=\color{blue},
        commentstyle=\color{red},
        stringstyle=\color{green}
    ] {code/#1}
}

\newcommand{\call}{\cpp{std::callcc()}}
\newcommand{\resume}{\cpp{std::resume()}}
\newcommand{\cont}{\cpp{std::contiunation}}
\newcommand{\ectx}{\cpp{std::execution\_context<>}}
\newcommand{\main}{\cpp{main()}}

\newcommand{\callcc}{\emph{call-with-current-continuation}}
\newcommand{\cc}{\emph{call/cc}}
\newcommand{\contfn}{\emph{continuation-function}}

\newcommand{\abschnitt}[1]{\addcontentsline{toc}{subsection}{#1}\subsection*{#1}}
\newcommand{\uabschnitt}[1]{\addcontentsline{toc}{subsubsection}{#1}\paragraph*{#1}}


%//////////////////////////////////////////////////////////////////////////////

\begin{document}
\small
\begin{tabbing}
    Document number: \= P0534R0\\
    Date:            \> 2017-01-01\\
    Reply-to:        \> Oliver Kowalke (oliver.kowalke@gmail.com)\\
    Audience:        \> SG1/LEWG\\
\end{tabbing}

\section*{call/cc: `A low-level API for stackful context switching`}

%//////////////////////////////////////////////////////////////////////////////

\tableofcontents

%//////////////////////////////////////////////////////////////////////////////

\paragraph*{Abstract}
This document {\bfseries supersedes P0099R1}.
\newline
This paper proposes a C++ equivalent to the well-known concept of \callcc.\\
In fact, the proposed \etc from P0099R1\cite{P0099R1} already represents a
one-shot continuation, reminding on Scheme's\cite{schemecallcc} and
Ruby's\cite{rubycallcc} \emph{call/cc}. Many other programing languages contain
as similar feature\footnote{for instance see PICO\cite{picocallcc}}.\\
\newline
From this point of view the proposed \cc is an advancement of \ectx.\\
\newline
Benefits are:
\begin{itemize}
    \item   established/well-known concept
    \item   prevent name clashes with \emph{execution-context} in
            executor and network proposals
    \item   relaxed constrains regarding to transfer of data
    \item   eliminating falls usage
\end{itemize}

\abschnitt{Introduction}

\cc (abrevation of 'call with current continuation') is a universal control
operator (well-known from like Scheme, Ruby, Lisp ...) that captures the
current continuation as a first-class object and pass it as an argument to
a function (named \contfn in rest of the text) that is executed in a newly
created continuation.


\uabschnitt{Continuation}

Modern mico-processors are registers machines; the content of processor
registers represent a execution context of the program at a given point in
time.\\
Operating systems simulate parallel execution of programs on a single processor
by switching between programs (context switch) by preserving and restoring the
the content of all registers.\\
Contiunations uses a similar technique - they represent the state of the
execution context and can be suspended and resumed later in order to change the
control flow of a program.\\
This is achieved by preserving and restoring a subset of micro-processor
registers (similar to OS context switching).\\
\newline
Continuations are useful to implement other control structures like
coroutines/(lazy) generators, lightweight threads, cooperative mutlitasking
(fibers), backtracking, non-deterministic choice ...\\
\newline
With first-class contiunations a language can control the order of instructions.
They enable to jump into a function on exact the point were it that has been
exited previously. A contiunation preserves the execution context (e.g. state of
the register machine).
\newline

\con represents a contiunation; it contains the content of preserved
registers and manages the associated stack (allocation/deallocation).
\con s a one-shot continuation - it can be used only once, after applied to
\resume it is invalidated.\\
\newline

\con is  only move-constructible and move-assignable.

As a first-class object \con can be applied to and returned from a function,
assigned to a variable or stored in a container.

A contiunation is continued by appling to function `resume()`.


\uabschnitt{callcc function}

\call is the C++ equivalent to Scheme's \cc operator. It captures the
current continuation (the rest of the computation; code after \call) and
trickers a context switch. The context switch is achived by preserving certain
registers (including instruction and stack pointer), defined by the calling
convention of the ABI, of the current continuation and restoring those
registers of the resumed continuation. The control flow of the resumed
continuation continues.
The current continuation is suspended and passed as argument to the resumed
continuation.

\call expects a \contfn with signature
`'continuation(continuation && c, Arg ...arg)'`. The parameter `c`
represents the current continuation from which this continuation was resumed
(e.g. that has called \call) and `arg` are the data passed to \call.

On return the \contfn of the current continuation has to specify an
\con to which the execution control is transferred after termination
of the current continuation.

If an instance with valid state goes out of scope and the \contfn has
not yet returned, the stack is traversed in order to access the control
structure (address stored at the first stack frame) and continuation's stack is
deallocated.


\uabschnitt{resume function}

\abschnitt{The switch mechanism}\label{mechanism}

Modern \bfs{micro-processors} are \bfs{registers machines}; the content of
processor registers represent a execution context of the program at a given
point in time.\\
\bfs{Operating systems} simulate parallel execution of programs on a single
processor by switching between programs (\bfs{context switch}) by
\bfs{preserving} and \bfs{restoring} the content of \bfs{all registers}.\\

For \cc no all registers have to be preserved because the calling convention,
part of the ABI, that covers how function's arguments and return values are
passed, determines which subset of CPU registers have to be preserved by the
called subroutine (scratch and callee-saved registers).\\

As a consequence a continuation preserves the execution context, e.g. state of
the register machine (including the stack as well as the instruction pointer).

The \bfs{calling convention}\cite{SYSVABI} of \bfs{SYSV ABI} for \bfs{x86\_64}
architecture determines that general purpose registers R12, R13, R14, R15, RBX
and RBP have to be preserved by the sub-routine - the first arguments are passed
to functions via RDI, RSI, RDX, RCX, R8 and R9 and values returned from
functions in RAX, RDX.\\

\asmfn{jump_x86_64.S}

The code fragment, taken from boost.context\cite{bcontext}, shows how the
context switch might be implemented for \bfs{SYSV ABI}/\bfs{x86\_64}.\\
Line (1) prepares the stack of the current context to hold the content of
registers R12-R15, RBX ad RBP. The address of the stack pointer is preserved in
register RAX at line (10) for further use.\\
The \bfs{return address}, e.g. the address of the instruction that has to be
executed after this function returns, is left on the stack. Other architectures
store the return address in a special register (link register) instead on the
stack. In this case the link register must be preserved too.\\
At line (12) the stack pointer gets assigned to the address of the
continuation that as to be resumed - in fact, the continuation represents a
stack address (the stack pointer was passed in RDI as first argument).\\
The return address is loaded into register R8 at line (14); with the indirect
jump at line (28) the \bfs{continuation} is \bfs{resumed}.\\
As required by the calling convention, registers R12-R15, RBX and RBP are
restored at lines (16) - (21).\\
The stack address, preserved in RAX at line (10), of the
\bfs{suspended continuation} is \bfs{returned} as a \bfs{one-shot continuation}.\\

In fact, \cc is an extended function call; while some \bfs{general purpose
registers}, defined by the calling convention, are preserved, additionally the
\bfs{stack pointer} and \bfs{instruction pointer} are preserved and exchanged
too - for code that invokes \cc it behaves like an ordinary function call.

\abschnitt{Call with current continuation}

\cc (abbreviation of 'call with current continuation') is a universal control
operator (well-known from languages like Scheme, Ruby, Lisp ...) that captures
the \currcont (instructions that happen next after \cc returns) as a
\bfs{first-class object} and pass it as an argument to a function that is
executed in the newly created execution context.\\

\call is the C++ equivalent to \cc, preserving the \bfs{call state} and the
\bfs{program state} (variables).\\

When a continuation object is applied to \resume, the existing continuation is
eliminated and the applied continuation is restored in its place, so that the
program flow will continue at the point at which the continuation was captured
and the currnt continuation then becomes the \emph{return value} of the \call
invocation (see \nameref{mechanism}).\\

\cont is a \bfs{one-shot continuation} - it can be used only once, is only
move-constructible and move-assignable.
\cppf{loop}

The \cpp{std::callcc(foo)} call at (a) captures the \currcont, enters function
\cpp{foo()} while passing the captured continuation as argument \cpp{c1}.\\
As long as continuation \cpp{c1} is valid, \cpp{"foo"} is passed to standard
output.\\
The expression \cpp{std::resume(std::move(c1))} at (b) resumes the original
continuation represented within \cpp{foo()} by \cpp{c1} and transfers back the
control of execution to \cpp{main()}. On return from \cpp{std::callcc(foo)},
the assignment at (c) sets \cpp{c2} to the \currcont by (b).\\
The call to \cpp{resume(std::move(c2))} at (d) (the \emph{updated} \cpp{c2})
resumes function \cpp{foo}, returning from the \cpp{resume()} call at (b) and
executing the assignment at (e). Here, too, we replace the \cont instance
\cpp{c1} invalidated by the \resume call at (b) with the new instance
returned by that same \resume call.\\
Function \call captures \currcont and enters the given function immediately,
while \resume returns the control back to the continuation passed as argument.\\
The presented code prints out \cpp{"foo"} and \cpp{"bar"} in a endless loop.\\

In order to transfer data, \call as well as \resume accept arguments, that are
stored on the stack of the \currcont. Function \davail tests if data have been
passed and with \dget the data can be accessed.
\cppf{fibonacci}

The invocation of \call at (a) immediately enters the lambda, passing no data
but the \currcont. The lambda calculates the fibonacci number using local
variables \cpp{a}, \cpp{b} and \cpp{next}. The calculated fibonacci number is
transferred via \resume at (b). The execution control returns and \cpp{c2}
represents the continuation of the lambda. With \dget at (c) the fibonacci
number is transferred to the current context while at (d) the lambda is entered
again in order to compute the next fibonacci number (without passing parameter
to the lambda).

\newpage
\abschnitt{Use cases}

\cc can be used to implement several higher-level abstractions.


\uabschnitt{Asymmetric coroutines: N3708}

is implemented in boost.coroutine2\cite{bcoroutine2} using \cc from
boost.context\cite{bcontext} as a building block. Each \cpp{push\_type} and
\cpp{pull\_type} of a coroutine represents a continuation (i.e. a coroutine
consists of two continuations).
\cppf{coroutine}


\uabschnitt{Cooperative multi-tasking:}

boost.fiber\cite{bfiber} provides a framework for micro-/userland-threads
(fibers) scheduled cooperatively. The library implements fibers using \cc
(boost.context\cite{bcontext}). The API contains classes and functions to manage
and synchronize fibers, similar to the standard thread support library.\\
Each fiber is implemented using a continuation.
\cppf{fiber}


\uabschnitt{Delimited continuations}\label{delimited}

can be implemented via \cc. \cpp{reset} delimits the continuation and
\cpp{shift} reifies the continuation, i.e. the code that follows after
\cpp{shift} returns is passed as a continuation to \cpp{shift}.\\

On entry \cpp{1} is written to \cpp{std::cout} at (0). The \cpp{shift} operator
at (1) wraps the continuation, that means the code at (2), and passes it as
argument \cpp{cont} at (1). \cpp{cont()} is called two times, thus (2) is
executed two times before (3) writes \cpp{2} to \cpp{std::cout}.
\cppf{delimited}


\uabschnitt{Backtracking}

or non-deterministic choice is the ability to specify certain
\emph{choice points} in the program used to for finding all (or some) solutions
to some computational problems. The algorithm \emph{backtracks} to a previous
\emph{choice point} as soon as it determines that the current execution path cannot
reach a valid solution.\\
Backtracking could be implemented using two continuations, a success
continuation that proceeds with the algorithm and a failure continuation that
backtracks to a previous choice point\cite{Ferguson}.

\newpage
\abschnitt{Additional notes}

\uabschnitt{GPU}

\cc as proposed in this paper does not take GPUs into account. Later revisions
will address this issue, once we have an overarching concept of how the various
kinds of ``lightweight execution agents'' should interact.


\uabschnitt{SIMD}

does not interfere with \cc and can be used as usual (\cc triggers the context
switch at its invocation).\\
Of course, depending on the calling convention, some micro-processor registers,
dedicated to SIMD, might be preserved and restored too
\footnote{\emph{MS Windows x64} calling convention}.


\uabschnitt{TLS}

\cc is TLS-agnostic - best practice related to TLS applies to \cc too.\\
As shown in \nameref{mechanism}, \cc only preserves and restores
micro-processor registers at its invocation.


\uabschnitt{Migration between threads}

\cont can be migrated between threads, except for instances of
\cont representing \main or \entryfn of a thread (see \nameref{subsec:main}).

\abschnitt{A. Assembler: shortes list of mnemonics (ARM)}\label{appendixa}

\asmf{jump_arm.S}

\newpage
\abschnitt{B. Assembler: longest list of mnemonics (PPC)}\label{appendixb}

The code is taken from boost.context\cite{bcontext} (architecture: PPC 32bit,
calling convention: SYSV).

\asmf{jump_ppc.S}


%//////////////////////////////////////////////////////////////////////////////

\addcontentsline{toc}{subsection}{References}
\begin{thebibliography}{99}

    \bibitem{P0099R1}
        \href{http://www.open-std.org/jtc1/sc22/wg21/docs/papers/2016/p0099r1.pdf}
        {P0099R1: A low-level API for stackful context switching}

    \bibitem{schemecallcc}
        \href{http://community.schemewiki.org/?call-with-current-continuation}
        {call/cc in Scheme}

    \bibitem{rubycallcc}
        \href{http://gnuu.org/2009/03/21/demystifying-continuations-in-ruby}
        {call/cc in Ruby}

    \bibitem{SYSVABI}
        {System V Application Binary Interface, AMD64 Architecture Processor Supplement,
        Draft Version 0.96}

    \bibitem{N3708}
        \href{http://www.open-std.org/jtc1/sc22/wg21/docs/papers/2013/n3708.pdf}
        {N3708: A proposal to add coroutines to the C++ standard library}

    \bibitem{gccsplit}
        \href{http://gcc.gnu.org/wiki/SplitStacks}
        {Split Stacks / GCC}

    \bibitem{bcontext}
        \href{http://www.boost.org/doc/libs/release/libs/context/doc/html/index.html}
        {Library \emph{Boost.Context}} (\cc available int boost-1.64)

    \bibitem{bcoroutine2}
        \href{http://www.boost.org/doc/libs/release/libs/coroutine2/doc/html/index.html}
        {Library \emph{Boost.Coroutine2}}

    \bibitem{bfiber}
        \href{http://www.boost.org/doc/libs/release/libs/fiber/doc/html/index.html}
        {Library \emph{Boost.Fiber}}

    \bibitem{Ferguson}
        {Darrell Ferguson and Dwight Deugo, Call with Current Continuation Patterns.
        in 8th Conference on Pattern Languages of Programs (PLoP 2001) , 2001}

\end{thebibliography}


%//////////////////////////////////////////////////////////////////////////////

\end{document}
