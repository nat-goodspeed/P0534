\abschnitt{Why \cc should be preferred over \uc or \lj}

\uabschnitt{stack represents the continuation}

In contrast to \uc, \cc uses the stack as storage for the suspended
execution context (the content of the registers, see P0534R0\cite{P0534R0}).

\begin{itemize}
    \item only the target has to be provided at resumption
        (\cpp{swap\_ucontext()} required source and target)
    \item current execution context is already represented by the
        stack to which the stack-pointer points
    \item suspended execution context is passed as continuation (parameter) 
        to the resumed execution context
    \item no need for a global pointer that points to the current execution context
    \item data are transferred through a function call, no need for a global pointer
    \item \main and each thread's \entryfn fit seamlessly into the concept of \cc
        because the stack of \main, or the thread, already represents the continuation of
        that context
\end{itemize}

\uabschnitt{aggregation of stack address}

A instance of \cont contains the stack address of a suspended execution.
\cont:

\begin{itemize}
    \item represents the continuation of a suspended context
    \item prevents accidentally copying the stack
    \item prevents accidentally resuming the running execution
        context
    \item prevents accidentally resuming an execution context that has already
        terminated (computation has finished)
    \item manages lifespan of an explicitly-allocated stack: the stack gets
        deallocated when \cont goes out of scope
    \item context switch and data transfer via one function call
\end{itemize}

Of course a \uc-like standard API would be possible, but in C++ we can do much
better with very little abstraction cost.

\uabschnitt{Disadvantages of \uc}

\begin{itemize}
    \item deprecated since POSIX.1-2004d and removed in POSIX.1-2008
    \item \cpp{makecontext} violates C99 standard (function pointer cast and integer arguments)
    \item \cpp{makecontext} arguments in var-arg list are required to be integers; passing pointers
        is not guaranteed to work (especially on platforms where pointers are larger than integers)
    \item \cpp{swapcontext} calls into the kernel, consuming many CPU cycles (two orders
        of magnitude)
    \item does not prevent accidentally copying the stack
    \item does not prevent accidentally resuming the running execution
        context
    \item does not prevent accidentally resuming an execution context that has already
        terminated (computation has finished)
    \item does not manage lifespan of an explicitly-allocated stack
    \item context switch (\cpp{swapcontext}) does not transfer data (requires global pointer)
\end{itemize}

\uabschnitt{Disadvantages of \lj}

\begin{itemize}
    \item C99 defines \sj / \lj for non-local jumps
    \item \lj is not required to preserve the current stack frame, therefore jumping into a function
        which has exited (by return or by a different \lj higher up the stack) is undefined behavior:
        only \lj up the call stack is allowed
    \item does not prevent accidentally copying the stack
    \item does not prevent accidentally resuming the running execution
        context
    \item does not prevent accidentally resuming an execution context that has already
        terminated (computation has finished)
    \item does not manage lifespan of an explicitly-allocated stack
    \item context switch (\lj) does not transfer data (requires global pointer)
\end{itemize}
