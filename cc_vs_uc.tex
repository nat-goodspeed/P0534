\abschnitt{Why \cc is preferred over \uc}

\uabschnitt{stack represents the continuation}

In contrast to \uc, \cc uses the stack as storage for the suspended
execution context (the content of the registers, see P0534R0\cite{P0534R0}).

\begin{itemize}
    \item only the target has to be provided at resumption
        (\cpp{swap\_ucontext()} required source and target)
    \item current execution context is already represented by the
        stack to which the stack-pointer points to
    \item suspended execution context is passed as continuation (parameter) 
        to the resumed execution context
    \item no need for a global pointer that points to the current execution context
    \item \main and thread's \entryfn do fit seamlessly into the concept of \cc
        because the stack of \main or thread already represents the continuation of \main
        and thread
\end{itemize}

\uabschnitt{aggregation of stack address}

A instance of \cont aggregates the stack address of a suspended execution
- \cont:

\begin{itemize}
    \item represents the continuation of an suspended context
    \item prevents accidentaly copying the stack
    \item prevents accidentaly resuming an already resumed/running execution
        context
    \item prevents accidentaly resuming an execution context that has already
        terminated (compuation has already been finished)
    \item manages lifespan of an explictly-allocated stack, the stack gets
        deallocated if \cont goes out of scope
    \item context switch and data transfer via one function call
\end{itemize}

Ofcourse a \uc-like API would be possible but it would be unsafe and error
prone.
