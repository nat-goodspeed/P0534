\abschnitt{Introduction}

\cc (abrevation of 'call with current continuation') is a universal control
operator (well-known from like Scheme, Ruby, Lisp ...) that captures the
current continuation as a first-class object and pass it as an argument to
a function (named \contfn in rest of the text) that is executed in a newly
created continuation.


\uabschnitt{Continuation}

Modern mico-processors are registers machines; the content of processor
registers represent a execution context of the program at a given point in
time.\\
Operating systems simulate parallel execution of programs on a single processor
by switching between programs (context switch) by preserving and restoring the
the content of all registers.\\
Contiunations uses a similar technique - they represent the state of the
execution context and can be suspended and resumed later in order to change the
control flow of a program.\\
This is achieved by preserving and restoring a subset of micro-processor
registers (similar to OS context switching).\\
\newline
Continuations are useful to implement other control structures like
coroutines/(lazy) generators, lightweight threads, cooperative mutlitasking
(fibers), backtracking, non-deterministic choice ...\\
\newline
With first-class contiunations a language can control the order of instructions.
They enable to jump into a function on exact the point were it that has been
exited previously. A contiunation preserves the execution context (e.g. state of
the register machine).
\newline

\con represents a contiunation; it contains the content of preserved
registers and manages the associated stack (allocation/deallocation).
\con s a one-shot continuation - it can be used only once, after applied to
\resume it is invalidated.\\
\newline

\con is  only move-constructible and move-assignable.

As a first-class object \con can be applied to and returned from a function,
assigned to a variable or stored in a container.

A contiunation is continued by appling to function `resume()`.


\uabschnitt{callcc function}

\call is the C++ equivalent to Scheme's \cc operator. It captures the
current continuation (the rest of the computation; code after \call) and
trickers a context switch. The context switch is achived by preserving certain
registers (including instruction and stack pointer), defined by the calling
convention of the ABI, of the current continuation and restoring those
registers of the resumed continuation. The control flow of the resumed
continuation continues.
The current continuation is suspended and passed as argument to the resumed
continuation.

\call expects a \contfn with signature
`'continuation(continuation && c, Arg ...arg)'`. The parameter `c`
represents the current continuation from which this continuation was resumed
(e.g. that has called \call) and `arg` are the data passed to \call.

On return the \contfn of the current continuation has to specify an
\con to which the execution control is transferred after termination
of the current continuation.

If an instance with valid state goes out of scope and the \contfn has
not yet returned, the stack is traversed in order to access the control
structure (address stored at the first stack frame) and continuation's stack is
deallocated.


\uabschnitt{resume function}
