\setlength{\parindent}{0pt} 
\renewcommand\sfdefault{phv}

\makeatletter
    \renewcommand*\l@subsection{\@dottedtocline{2}{0em}{2.3em}}
    \renewcommand*\l@subsection{\@dottedtocline{3}{0em}{3.2em}}
    \renewcommand{\tableofcontents}{\@starttoc{toc}}
\makeatother

\MakePerPage{footnote}
\renewcommand*{\thefootnote}{\fnsymbol{footnote}}

\newcommand{\pdfimg}[1]{\pdfximage{pics/#1}\pdfrefximage\pdflastximage}
\newcommand{\img}[1]{\mbox{\pdfimg{#1}}}
\newcommand{\imgc}[1]{\begin{center}\img{#1}\end{center}}
\newcommand{\graph}[1]{\input{graphs/#1}}
\newcommand{\graphc}[1]{\begin{center}\graph{#1}\end{center}}
\newcommand{\bfs}[1]{{\bfseries #1}}

\lstdefinelanguage
   [arm]{Assembler}                     % add a "arm" dialect of Assembler
   {morekeywords={}}    % with these extra keywords:
\lstset{
        language=[arm]Assembler,
        numbers=none,
        numberstyle=\tiny,
        numberblanklines=false,
        stepnumber=1,
        numbersep=10pt
}

\newcommand{\cpp}[1]{{\lstinline[
		basicstyle=\ttfamily\small\color{black},
        breakatwhitespace=true,
        breaklines=true,
        captionpos=b,
        commentstyle=\ttfamily\color{gray},
        keywordstyle=\ttfamily\color{blue},
        language={C++},
        morekeywords={co\_await,from,noexcept,resumable,co\_yield},
        showspaces=false,
        showstringspaces=false,
        showtabs=false,
        stringstyle=\ttfamily\color{red}
] !#1!}\xspace}
\newcommand{\cppf}[1]{\lstinputlisting[
		basicstyle=\ttfamily\small\color{black},
        breakatwhitespace=true,
        breaklines=true,
        captionpos=b,
        commentstyle=\ttfamily\color{gray},
        keywordstyle=\ttfamily\color{blue},
        language={C++},
        morekeywords={co\_await,from,noexcept,resumable,co\_yield},
        showspaces=false,
        showstringspaces=false,
        showtabs=false,
        stringstyle=\ttfamily\color{red}
] {code/#1.cpp}}


\newcommand{\asm}[1]{
    \lstinline[
        basicstyle=\ttfamily\color{black},
        keywordstyle=\color{blue},
        commentstyle=\color{red},
        stringstyle=\color{green}
    ] {#1}
}
\newcommand{\asmf}[1]{
    \lstinputlisting[
        basicstyle=\ttfamily\color{black},
        keywordstyle=\color{blue},
        commentstyle=\color{red},
        stringstyle=\color{green}
    ] {code/#1}
}
\newcommand{\asmfn}[1]{
    \lstinputlisting[
        numbers=left,
        basicstyle=\ttfamily\color{black},
        keywordstyle=\color{blue},
        commentstyle=\color{red},
        stringstyle=\color{green}
    ] {code/#1}
}

\newcommand{\call}{\cpp{std::callcc()}\xspace}
\newcommand{\resume}{\cpp{std::resume()}\xspace}
\newcommand{\cont}{\cpp{std::continuation}\xspace}
\newcommand{\ectx}{\cpp{std::execution\_context<>}\xspace}
\newcommand{\main}{\cpp{main()}\xspace}
\newcommand{\shift}{\cpp{shift}\xspace}
\newcommand{\reset}{\cpp{reset}\xspace}
\newcommand{\davail}{\cpp{data\_available()}\xspace}
\newcommand{\dtransfer}{\cpp{transfer\_data()}\xspace}
\newcommand{\opbool}{\cpp{operator bool()}}

\newcommand{\callcc}{\bfs{call-with-current-continuation}\xspace}
\newcommand{\cc}{\bfs{call/cc}\xspace}
\newcommand{\entryfn}{\emph{entry-function}\xspace}
\newcommand{\currcont}{\bfs{current continuation}\xspace}

\newcommand{\abschnitt}[1]{\addcontentsline{toc}{subsection}{#1}\subsection*{#1}}
\newcommand{\uabschnitt}[1]{\paragraph*{#1}}
