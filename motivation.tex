\abschnitt{Motivation}\label{motivation}

\cc is an evolutionary step of \ectx; beside the name clashes with the executor
and network proposals, \ectx has some drawbacks:
\begin{itemize}
    \item The API described in P0099R1 allows calling
        \cpp{operator()(std::invoke\_ontop\_arg,fn &&,Args...)} on a newly
        created \ectx, which results in undefined behaviour: the context
        must be entered at least one time before it is permitted to
        invoke a function on top of the context.
    \item \ectx mandates data transfer in both directions, even when not
        required. For instance, generators must pass (and pay for)
        \bfs{dummy data} in one direction.
    \item \ectx transfers only data of a fixed type (the template
        argument). Implementation experience with boost.coroutine2 and
        boost.fiber shows that implementations of higher-level abstractions
        might require transferring data of different types, or no data at all
        (for instance during initialization phase).
    \item If the data type used by \ectx (template argument) is not default
        constructible and the context-function simply returns (no data
        transferred back) the expression \cpp{auto [ctx2,x2]=ctx1(x1)} is
        invalid: \cpp{x2} can not be default constructed.
\end{itemize}
