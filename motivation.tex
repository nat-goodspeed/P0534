\abschnitt{Motivation}\label{motivation}

\cc is an evolutionary step of \ectx; beside the name clashes with the executor
and network proposals, \ectx has some drawbacks:
\begin{itemize}
    \item The API P0099R1 allows to call
        \cpp{operator()(std::invoke\_ontop\_arg,fn &&,Args...)} on a newly
        created \ectx which results in undefined behaviour (the context is
        required to be entered at least one time before it is permitted to
        invoke a function ontop of the context).
    \item \ectx forces to transfer data in both directions, even if it is not
        required, for instance generators must pass (and pay for)
        \bfs{dummy data} in one direction.
    \item \ectx allows to transferre only data of a fixed type (template
        argument), but the implementation experience of boost.coroutine2 and
        boost.fiber showed that implementations of higher-level abstractions
        might require to transfer data of different types or no data at all
        (for instance during initialization phase).
    \item If the data type used by \ectx (template argument) is not default
        constructible and the context-function simply returns (no data
        transferred back) the expression \cpp{auto [ctx2,y2]=ctx1(x1)} is
        invalid (\cpp{x2} can not be default constructed).
\end{itemize}
