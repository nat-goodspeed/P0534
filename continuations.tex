\abschnitt{Continuations}

A continuation is a abstract concept that represents the context state at a
given point during the execution of a program. As a \bfs{basic, low-level
primitive} it cane be used to implement control structures like coroutines,
(lazy) generators, lightweight threads, cooperative multitasking (fibers),
backtracking, non-deterministic choice. In classic event-driven programs,
organized around a main loop that fetches and dispatches incoming I/O events,
certain asynchronous I/O sequences are logically sequential, and for those the
written and maintained code can look and act sequential while using
continuations.\\

C and C++ already use continuations - if a routine calls a sub-routine, then a
(hidden) continuation (the rest of the computation after the sub-routine call)
is created. This continuation is resumed when the sub-routine returns
(\bfs{return-continuation}).\\

Continuations exposed as \bfs{first-class continuations} can be passed to and
returned from functions, assigned to variables or stored into containers. With
first-class continuations a language can \bfs{control the flow of execution}
via suspending and resuming continuations explicitly (enabling to jump into a
function on exact the point were it has been exited previously).\\
Making the program state visible via a first-class continuations is known as
\bfs{reification}.\\

The continuation of the computation step, derived from the current point in a
program's execution, is called \bfs{current continuation} (\cc captures the
\bfs{current continuation} and passes it as argument of the function invoked by
\cc).\\

Continuations that can be called multiple times are named
\bfs{full continuations}.\\
\bfs{One-shot continuations} can only resumed once (a resumed 
\bfs{one-shot continuation} becomes invalidated); the control is transferred to
a execution context where to continuation is no longer in scope.\\

In contrast to s \cc that captures the \bfs{entire remaining} continuation, the
operators \shift and \reset create a so called \bfs{delimited continuation}. A
delimited continuation represents a slice of the program context. Operator
\reset delimits the continuation, e.g. it determines where the continuation
starts and ends, while \shift \bfs{reifies} the continuation.
