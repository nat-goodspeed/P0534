\abschnitt{Continuations}

A continuation is an abstract concept that represents the context state at a
given point during the execution of a program. That implies that a continuation
represents the remaining steps of a computation.\\

As a \bfs{basic, low-level primitive} it can be used to implement control
structures like coroutines,  generators, lightweight threads, cooperative
multitasking (fibers), backtracking, non-deterministic choice. In classic
event-driven programs, organized around a main loop that fetches and dispatches
incoming I/O events, certain asynchronous I/O sequences are logically
sequential, and for those the written and maintained code can look and act
sequential while using continuations.\\

C and C++ already use implicit continuations: if a routine calls a sub-routine,
then a (hidden) continuation (the remaining steps after the sub-routine call) is
created. This continuation is resumed when the sub-routine returns. For
instance the x86 architecture stores the (hidden) continuation as return address
on the stack\footnote{Other (RISC) architectures use a special micro-processor
register for this purpose.}.\\

Continuations exposed as \bfs{first-class continuations} can be passed to and
returned from functions, assigned to variables or stored into containers. With
first-class continuations, a language can explicitly \bfs{control the flow of
execution} via suspending and resuming continuations, enabling control to pass
into a function at exactly the point where it previously suspended.\\
Making the program state visible via first-class continuations is known as
\bfs{reification}.\\

The continuation of the computation step derived from the current point in a
program's execution is called the \bfs{current continuation}. \cc captures the
\bfs{current continuation} and passes it as the argument of the function invoked by
\cc.\\

Continuations that can be called multiple times are named
\bfs{full continuations}.\\
\bfs{One-shot continuations} can only resumed once (a resumed 
\bfs{one-shot continuation} becomes invalidated); control is transferred to
an execution context where the continuation is no longer in scope.\\
Class \cpp{std::execution_context<>}, proposed in \bfs{P0099R1}\cite{P0099R1}
already represents a \bfs{one-shot-continuation}.\\
\bfs{Full continuations} are \bfs{not} considered in this proposal because of
their nature, problematic in C++. Full continuations would require copies of
the stack (including the variables), which would violate C++'s RAII pattern.\\

In contrast to \cc that captures the \bfs{entire remaining} continuation, the
operators \shift and \reset create a so called \bfs{delimited continuation}. A
delimited continuation represents a slice of the program context. Operator
\reset delimits the continuation, i.e. it determines where the continuation
starts and ends, while \shift \bfs{reifies} the continuation.\\
\bfs{Delimited continuations} are \bfs{not part} of this proposal. However,
delimited continuation functionality can be built on \cc, as shown in
\nameref{delimited}.

